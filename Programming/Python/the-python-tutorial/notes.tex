%-*- coding: UTF-8 -*-
% notes.tex
%
\documentclass[UTF8]{article}
\usepackage{geometry}
\geometry{a4paper, centering, scale=0.8}
\usepackage{minted}
\usepackage{hyperref}
\usepackage{indentfirst} % to indent the first paragraph of a section

\title{The Python Tutorial Notes}
\author{Du Ang \\ \texttt{du2ang233@gmail.com} }
\date{\today}

\begin{document}
\maketitle

\tableofcontents
\newpage

\section{Whetting Your Appetite}
\subsection{Some applications of Python}
\begin{itemize}
    \item Automate works on computer
    \item Help software engineers develop and test more quickly
    \item ...
\end{itemize}

\subsection{The features of Python}
\begin{itemize}
    \item High-level language \\
    High-level more general data types
    \item Allow to split program into modules
    \item Interpreted language
    \begin{itemize}
        \item No necessary compilation and linking
        \item The interpreter can be used interactively
    \end{itemize}
    \item Shorter programs than equivalent C, C++ or Java programs
    \begin{itemize}
        \item high-level data types
        \item statement grouping is done by indentation instead of brackets
        \item no variable or argument declarations are necessary
    \end{itemize}
    \item Extensible
\end{itemize}

\subsection{The origin of the name "Python"}
“Monty Python’s Flying Circus”, the skits on BBC.

\section{Using the Python Interpreter}
\subsection{Invoking the Interpreter}
The Python interpreter is usually installed as \texttt{/usr/local/bin/python}, and usually the
\texttt{/usr/local/bin} has been put in the Unix shell’s search path. Type command \texttt{python}
to the shell to invoke the Python interpreter.

Type an end-of-file character(\texttt{Ctrl-D} on Unix, \texttt{Ctrl-Z} on Windows) to exit the
interpreter. If that doesn’t work, type command \mintinline{Python}{quit()} to the shell.

A second way of starting the interpreter is \mintinline{bash}{python -c 'command [arg] ...'},
which executes the statement(s) in command.

Some Python modules are also useful as scripts. These can be invoked using
\mintinline{bash}{python -m module [arg] ...}, which executes the source file for modules as if you
had spelled out its full name on the command line.

\subsubsection{Argument Passing}
When known to the interpreter, the script name and additional arguments thereafter are turned into
a list of strings and assigned to the \texttt{argv} variable in the \texttt{sys} module. You can
access this list by executing \texttt{import sys}. The length of the list is at least one; when no
script and no arguments are given, \texttt{sys.argv[0]} is an empty string. When the script name is
given as \texttt{'-'} (meaning standard input), \texttt{sys.argv[0]} is set to \texttt{'-'}. When
\texttt{-c} command is used, \texttt{sys.argv[0]} is set to \texttt{'-c'}. When \texttt{-m} module
is used, \texttt{sys.argv[0]} is set to the full name of the located module. Options found after
\texttt{-c} command or \texttt{-m} module are not consumed by the Python interpreter’s option
processing but left in \texttt{sys.argv} for the command or module to handle.

\subsubsection{Interactive Mode}
\begin{itemize}
    \item \emph{primary prompt}: \texttt{>>>}
    \begin{minted}{bash}
    Python 2.7.10 (default, Feb  7 2017, 00:08:15)
    Type "help", "copyright", "credits" or "license" for more information.
    >>>
    \end{minted}
    \item \emph{secondary prompt}: \texttt{...}
    \begin{minted}{bash}
    >>> the_world_is_flat = 1
    >>> if the_world_is_flat:
    ...     print "Be careful not to fall off!"
    ...
    Be careful not to fall off!
    \end{minted}
\end{itemize}

\subsection{The Interpreter and Its Environment}
\subsubsection{Source Code Encoding}
By default, Python source files are treated as encoded in UTF-8.

To declare an encoding other than the default one, a special comment line should be added as the
first line of the file. The syntax is as follows:
\begin{minted}{Python}
    # -*- coding: encoding -*-
\end{minted}
where \emph{encoding} is one of the valid \texttt{codes} supported by Python.

\section{An Informal Introduction to Python}
In the following examples, input and output are distinguished by the presence or absence of prompts
(\texttt{>>>} and \texttt{...}).

Comments in Python start with the hash character \texttt{\#}, and extend to the end of the physical
line. A hash character in a string literal is just a hash character.

\begin{minted}{Python}
    # this is the first comment
    spam = 1    # and this the second comment
                # and now a third!
    text = "# This is not a comment because it's inside quotes."
\end{minted}

\subsection{Using Python as a Calculator}
\subsubsection{Numbers}
\begin{minted}{bash}
    >>> 2 + 2
    4
    >>> 50 - 5 * 6
    20
    >>> (50 - 5.0 * 6) / 4
    5.0
    >>> 8 / 5.0
    1.6
\end{minted}

The integer numbers(e.g. \texttt{2}, \texttt{4}, \texttt{20}) have type \texttt{int}, the ones with
a fractional part(e.g. \texttt{5.0}, \texttt{1.6}) have type \texttt{float}.

The return type of a division (\texttt{/}) operation depends on its operands. If both operands are
of type \texttt{int}, floor division is performed and an \texttt{int} is returned. If either
operand is a float, classic division is performed and a float is returned. The \texttt{//} operator
is also provided for doing floor divsion no matter what the operands are. The remainder can be
calculated with the \texttt{\%} operator:
\begin{minted}{bash}
    >>> 17 / 3    # int / int -> int
    5
    >>> 17 / 3.0  # int / float -> float
    5.666666666666667
    >>> 17 // 3.0 # explicit floor division discards the fractional part
    5.0
    >>> 5 * 3 + 2 # result * divisor + remainder
    17
\end{minted}

With Python, it is possible to use \texttt{**} operator to calculate powers:
\begin{minted}{bash}
    >>> 5 ** 2  # 5 squared
    25
    >>> 2 ** 7  # 2 to the power of 7
    128
\end{minted}

The equal sign (\texttt{=}) is used to assign a value to a variable. Afterwards, no results is
displayed before the next interactive prompt:
\begin{minted}{bash}
    >>> width = 20
    >>> height = 5 * 9
    >>> width * height
    900
\end{minted}

If a variable is not "defined" (assigned a value), trying to use it will give you an error:
\begin{minted}{bash}
    >>> n   # try to access an undefined variable
    Traceback (most recent call last):
    File "<stdin>", line 1, in <module>
    NameError: name 'n' is not defined
\end{minted}

There is full support for floating point; operators with mixed type operands convert the integer
operand to floating point:
\begin{minted}{bash}
    >>> 3 * 3.75 / 1.5
    7.5
    >>> 7.0 / 2
    3.5
\end{minted}

In interactive mode, the last printed expression is assigned to the variable \texttt{\_}.
\begin{minted}{bash}
    >>> tax = 12.5 / 100
    >>> price = 100.50
    >>> price * tax
    12.5625
    >>> price + _
    113.0625
    >>> round(_, 2)
    113.06
\end{minted}

The variable should be treated as read-only by the user. Don't explicitly assign a value to it ---
you would create an independent local variable with the same name masking the built-in variable with
its magic behavior.

In addition to \texttt{int} and \texttt{float}, Python supports other types of numbers, such as
\texttt{Decimal} and \texttt{Fraction}. Python also has built-in support for complex numbers, and
uses the \texttt{j} or \texttt{J} suffix to indicate the imaginary part (e.g. \texttt{3 + 5j}).

\subsubsection{Strings}
Python can also manipulate strigns, which should be enclosed in single quotes (\texttt{'...'}) or
double quotes (\texttt{"..."}) with the same result. \texttt{\textbackslash} can be used to escape
quotes.
\begin{minted}{bash}
    >>> 'spam eggs'  # single quotes
    'spam eggs'
    >>> 'doesn\'t'   # use \' to escape the single quote...
    "doesn't"
    >>> "doesn't"    # ... or use double quotes instead
    "doesn't"
    >>> '"Yes," he said.'
    '"Yes," he said.'
    >>> "\"Yes,\" he said."
    '"Yes," he said.'
    >>> '"Isn\'t," she said.'
    '"Isn\'t," she said.'
\end{minted}

The \texttt{print} statement produces a more readable output, by omitting the enclosing quotes and
by printing escaped and special characters:
\begin{minted}{bash}
    >>> '"Isn\'t," she said.'
    '"Isn\'t," she said.'
    >>> print '"Isn\'t," she said.'
    "Isn't," she said.
    >>> s = 'First line. \nSecond line.'  # \n means newline
    >>> s       # without print, \n is included in the output
    'First line. \nSecond line.'
    >>> print s # with print, \n produces a new line
    First line.
    Second line.
\end{minted}

If you don't want characters prefaced by \texttt{\textbackslash} to be interpreted as special
characters, you can use \emph{raw strings} by adding an \texttt{r} before the first quote:
\begin{minted}{bash}
    >>> print 'C:\some\name'  # here \n means newline!
    C:\some
    ame
    >>> print r'C:\some\name' # note the r before the quote
    C:\some\name
\end{minted}

String literals can span multiple lines. One way is using triple-quotes: \texttt{"""..."""} or
\texttt{'''...'''}. End of lines are automatically included in the string, but it's possible to
prevent this by adding a \texttt{\textbackslash} at the end of the line:
\begin{minted}{bash}
    >>> print """
    ... Usage: thingy [OPTIONS]
    ...     -h              Display this usage message
    ...     -H hostname     Hostname to connect to
    ... """

    Usage: thingy [OPTIONS]
        -h          Display this usage message
        -H hostname Hostname to connect to

    >>> print """\  # prevent the initial newline
    ... Usage: thingy [OPTIONS]
    ...     -h              Display this usage message
    ...     -H hostname     Hostname to connect to
    ... """
    Usage: thingy [OPTIONS]
        -h          Display this usage message
        -H hostname Hostname to connect to

\end{minted}

Strings can be concatenated (glued together) with \texttt{+} operator, and repeated with \texttt{*}:
\begin{minted}{bash}
    >>> # 3 times 'un', followed by 'ium'
    ... 3 * 'un' + 'ium'
    'unununium'
\end{minted}

Two or more string literals next to each other are automatically concatenated:
\begin{minted}{bash}
    >>> 'Py' 'thon'
    'Python'
\end{minted}

This only works with two literals though, not with variables or expressions. If you want to
concatenate variables or a variable and a literal, use \texttt{+}.
\begin{minted}{bash}
    >>> prefix = 'Py'
    >>> prefix 'thon'  # can't concatenate a variable and string literal
      File "<stdin>", line 1
        prefix 'thon'  # can't concatenate a variable and string literal
                    ^
    SyntaxError: invalid syntax

    >>> prefix + 'thon'
    'Python'
\end{minted}

This feature is particularly useful when you want to break long strings:
\begin{minted}{bash}
    >>> text = ('Put several string within parentheses '
    ...         'to have them joined together.')
    >>> text
    'Put several string within parentheses to have them joined together.'
\end{minted}

Strings can be indexed(subscripted), with the first character having index 0. There is no separate
character type; a character is simple a string a size one:
\begin{minted}{bash}
    >>> word  = 'Python'
    >>> word[0]     # character in position 0
    'P'
    >>> word[5]     # character in position 5
    'n'
\end{minted}

Indices may also be negative nubmers, to start counting from the right:
\begin{minted}{bash}
    >>> word[-1]    # last character
    'n'
    >>> word[-2]    # second-last character
    'o'
    >>> word[-6]
    'P'
\end{minted}
Note that since -0 is the same as 0, negative indices start from -1.

In addition to indexing, \emph{slicing} is also supported. While indexing is used to obtain
individual characters, \emph{slicing} allows you to obtain a substring:
\begin{minted}{bash}
    >>> word[0:2]   # characters from position 0 (included) to 2 (excluded)
    'Py'
    >>> word[2:5]   # characters from position 2 (included) to 5 (excluded)
    'tho'
\end{minted}

Since indices have useful defaults; an omitted first index defaults to zero, an ommited second
index defaults to the size of the string being sliced.
\begin{minted}{bash}
    >>> word[:2]    # character from the beginning to position 2 (excluded)
    'Py'
    >>> word[4:]    # character from the position 4 (included) to the end
    'on'
    >>> word[-2:]   # character from the second-last (included) to the end
    'on'
\end{minted}

Note how the start is always included, and the end always excluded. This makes sure that
\texttt{s[:i] + s[i:]} is always equal to \texttt{s}:
\begin{minted}{bash}
    >>> word[:2] + word[2:]
    'Python'
    >>> word[:4] + word[4:]
    'Python'
\end{minted}

One way to remember how slices work is to think of the indices as pointing \emph{between}
characters, with the left edge of the first character numbered 0. Then the right edge of the last
character of a string of n characters has index \emph{n}, for example:
\begin{minted}{text}
    +---+---+---+---+---+---+
    | P | y | t | h | o | n |
    +---+---+---+---+---+---+
    0   1   2   3   4   5   6
    -6  -5  -4  -3  -2  -1
\end{minted}

Attempting to use an index that is too large will result in an error:
\begin{minted}{bash}
    >>> word[42]
    Traceback (most recent call last):
      File "<stdin>", line 1, in <module>
    IndexError: string index out of range
\end{minted}

However, out of range slice indices are handled gracefully when used for slicing:
\begin{minted}{bash}
    >>> word[4:42]
    'on'
    >>> word[42:]
    ''
\end{minted}

Python strings cannot be changed --- they are immutable. Therefore, assigning to a indexed position
in the string results in an error:
\begin{minted}{bash}
>>> word[0] = 'J'
    Traceback (most recent call last):
      File "<stdin>", line 1, in <module>
    TypeError: 'str' object does not support item assignment

    >>> word[2:] = 'py'
    Traceback (most recent call last):
      File "<stdin>", line 1, in <module>
    TypeError: 'str' object does not support item assignment
\end{minted}

If you need a different string, you should create a new one:
\begin{minted}{bash}
    >>> 'J' + word[1:]
    'Jython'
    >>> word[:2] + 'py'
    'Pypy'
\end{minted}

The built-in fucntion \texttt{len()} returns the length of a string:
\begin{minted}{bash}
    >>> s = 'supercalifragilisticexpialidocious'
    >>> len(s)
    34
\end{minted}

\subsubsection{Unicode Strings}
Unicode has the advantege of providing one ordinal for every character in every script used in
modern and ancient texts.

Creating Unicode strings in Python is just as simple as creating normal strings:
\begin{minted}{bash}
    >>> u'Hello World!'
    u'Hello World!'
\end{minted}

The small \texttt{'u'} in front of the quote indicates that a Unicode string is supposed to be
created. If you want to include special characters in the string, you can do so by using the
Python \emph{Unicode-Escape} encoding. The following example shows how:
\begin{minted}{bash}
    >>> u'Hello\u0020World!'
    u'Hello World!'
\end{minted}
The escape sequence \texttt{\textbackslash u0020} indicates to insert the Unicode character with
the ordinal value 0x0020(the space character) at the given position.

\subsubsection{Lists}
Python knows a number of compound data types, used to group together other values. The most
versatile is the \emph{list}, which can be written as a list of comma-separated values (items)
between square brackets. Lists might contain items of different types, but usually the items all
have the same type.

\begin{minted}{bash}
    >>> squares = [1, 4, 9, 16, 25]
    >>> squares
    [1, 4, 9, 16, 25]
\end{minted}

Like strings (and all other built-in sequence type), lists can be indexed and sliced:
\begin{minted}{bash}
    >>> squares[0]      # indexing returns the item
    1
    >>> squares[-1]
    25
    >>> squares[-3:]    # slicing returns a new list
    [9, 16, 25]
\end{minted}

All slice operations return a new list containing the requested elements. This means that the
following slice returns a new (shallow) copy of the list:
\begin{minted}{bash}
    >>> squares[:]
    [1, 4, 9, 16, 25]
\end{minted}

Lists also supports operations like concatenation:
\begin{minted}{bash}
    >>> squares + [36, 49, 64, 81, 100]
    [1, 4, 9, 16, 25, 36, 49, 64, 81, 100]
\end{minted}

Unlike strings, which are immutable, lists are a mutable type, i.e. it is possible to change their
content:
\begin{minted}{bash}
    >>> cubes = [1, 8, 27, 65, 125] # something's wrong here
    >>> 4 ** 3  # the cube of 4 is 64, not 65!
    64
    >>> cubes[3] = 64   # replace the wrong value
    >>> cubes
    [1, 8, 27, 64, 125]
\end{minted}

You can also add new items at the end of the list, by using the \texttt{append()} method:
\begin{minted}{bash}
    >>> cubes.append(216)           # add the cube of 6
    >>> cubes.append(7 ** 3)        # add the cube of 7
    >>> cubes
    [1, 8, 27, 64, 125, 216, 343]
\end{minted}

Assignment to slices is also possible, and this can even change the size of the list or clear it
entirely:
\begin{minted}{bash}
    >>> letters = ['a', 'b', 'c', 'd', 'e', 'f', 'g']
    >>> letters
    ['a', 'b', 'c', 'd', 'e', 'f', 'g']
    >>> # replace some values
    ... letters[2:5] = ['C', 'D', 'E']
    >>> letters
    ['a', 'b', 'C', 'D', 'E', 'f', 'g']
    >>> # now remove them
    ... letters[2:5] = []
    >>> letters
    ['a', 'b', 'f', 'g']
    >>> # clear the list by replacing all the elements with an empty list
    ... letters[:] = []
    >>> letters
    []
\end{minted}

The built-in function \texttt{len()} also applies to lists:
\begin{minted}{bash}
    >>> letters = ['a', 'b', 'c', 'd']
    >>> len(letters)
    4
\end{minted}

It is possible to nest lists (create lists containing other lists), for example:
\begin{minted}{bash}
    >>> a = ['a', 'b', 'c']
    >>> n = [1, 2, 3]
    >>> x = [a, n]
    >>> x
    [['a', 'b', 'c'], [1, 2, 3]]
    >>> x[0]
    ['a', 'b', 'c']
    >>> x[0][1]
    'b'
    >>> x[1]
    [1, 2, 3]
\end{minted}

\subsection{First Steps Towards Programming}
We can write an initial sub-sequence of the \emph{Fibonacci} series as follows:
\begin{minted}{bash}
    >>> # Fibonacci series:
    ... # the sum of two elements defines the next
    ... a, b = 0, 1     # multiple assignment
    >>> while b < 10:   # loop as long as the condition remains true
    ...     print b
    ...     a, b = b, a + b
    ...
    1
    1
    2
    3
    5
    8
\end{minted}

In Python, like in C, any non-zero integer value is true; zero is false. The condition of the
\texttt{while} loop may also be a string or list value, in fact any sequence; anything with a
non-zero length is true, empty sequences are false. The standard comparision operators are written
the same as in C: \texttt{<} (less than), \texttt{>} (greater than), \texttt{==} (equal to),
\texttt{<=} (less than or equal to), \texttt{>=} (greater than or equal to) and \texttt{!=} (not
equal to).

The \emph{body} of loop is \emph{indented}: indentation is Python's way of grouping statements.
At the interactive prompt, you have to type a tab or space(s) for each indented line.

The \texttt{print} statement writes the value of the expression(s) it is given. It differs from
just writing the expression you want to write in the way it handles multiple expressions and
strings. Strings are pointer without quotes, and a space is inserted between items, so you can
format things nicely, like this:
\begin{minted}{bash}
    >>> i = 256 * 256
    >>> print 'The value of i is', i
    The value of i is 65536
\end{minted}

A trailing comma avoids the newline after the output:
\begin{minted}{bash}
    >>> a, b = 0, 1
    >>> while b < 1000:
    ...     print b,
    ...     a, b = b, a + b
    ...
    1 1 2 3 5 8 13 21 34 55 89 144 233 377 610 987
\end{minted}

\section{More Control Flow Tools}
\subsection{\texttt{if} Statements}
\begin{minted}{bash}
    >>> x = int(raw_input("Please enter an integer: "))
    Please enter an integer: 42
    >>> if x < 0:
    ...     x = 0
    ...     print 'Negative changed to zero'
    ... elif x == 0:
    ...     print 'Zero'
    ... elif x == 1:
    ...     print 'Single'
    ... else:
    ...     print 'More'
    ...
    More
\end{minted}

\subsection{\texttt{for} Statements}
The \texttt{for} statement in Python differs a bit from what you may be used to in C or Pascal.
Rather than always iterating over an arithmetic progression of numbers (like in Pascal), or giving
the user the ability to define both the iteration step and halting condition (as C), Python's
\texttt{for} statement iterates over the items of any sequence (a list or a string), in the order
that they appear in the sequence. For example:
\begin{minted}{bash}
    >>> # Measure some strings:
    ... words = ['cat', 'window', 'defenestrate']
    >>> for w in words:
    ...     print w, len(w)
    ...
    cat 3
    window 6
    defenestrate 12
\end{minted}

If you need to modify the sequence you are iterating over while inside the loop, it is recommended
that you first make a copy. Iterating over a sequence does not implicitly make a copy. The slice
notation makes this especially convenient:
\begin{minted}{bash}
    >>> for w in words[:]:
    ...     if len(w) > 6:
    ...         words.insert(0, w)
    ...
    >>> words
    ['defenestrate', 'cat', 'window', 'defenestrate']
\end{minted}

\subsection{The \texttt{range()} function}
If you do need to iterate over a sequence of numbers, the built-in function \texttt{range()} comes
in handy. It generates lists containing arithmetic progressions:
\begin{minted}{bash}
    >>> range(10)
    [0, 1, 2, 3, 4, 5, 6, 7, 8, 9]
    >>> range(5, 10)    # the given end point is never part of the generated list
    [5, 6, 7, 8, 9]
    >>> range(0, 10, 3) # '3' here is the increment('step'), it can be negative
    [0, 3, 6, 9]
    >>> range(-10, -100, -30)
    [-10, -40, -70]
\end{minted}

To iterate over the indices of a sequence, you can combine \texttt{range()} and \texttt{len()} as
follows:
\begin{minted}{bash}
    >>> a = ['Mary', 'had', 'a', 'little', 'lamb']
    >>> for i in range(len(a)):
    ...     print i, a[i]
    ...
    0 Mary
    1 had
    2 a
    3 little
    4 lamb
\end{minted}

\subsection{\texttt{break} and \texttt{continue} Statements, and \texttt{else} Clauses on Loops}
The \texttt{break()} statement, like in C, breaks out of the innermost enclosing \texttt{for} or
\texttt{while} loop.

Loop statements may have an \texttt{else} clause; it is exected when the loop terminates through
exhaustion of the list (with \texttt{for}) or when the condition becomes false (with
\texttt{while}), but not when the loop is terminated by a \texttt{break} statement. This is
exemplified by the following loop, which searches for prime numbers:
\begin{minted}{bash}
    >>> for n in range(2, 10):
    ...     for x in range(2, n):
    ...         if n % x == 0:
    ...             print n, 'equals', x, '*', n / x
    ...             break
    ...     else:
    ...         # loop fell through without finding a factor
    ...         print n, 'is a prime number'
    ...
    2 is a prime number
    3 is a prime number
    4 equals 2 * 2
    5 is a prime number
    6 equals 2 * 3
    7 is a prime number
    8 equals 2 * 4
    9 equals 3 * 3
\end{minted}

When used with a loop, the \texttt{else} clause has more in common with the \texttt{else} clause of
a \texttt{try} statement than it does that of \texttt{if} statements: a \texttt{try} statement's
\texttt{else} clause runs when no exception occurs, and a loop's \texttt{else} clause runs when no
\texttt{break} occurs.

The \texttt{continue} statement, also borrowed from C, continues with the next iteration of the
loop:
\begin{minted}{bash}
    >>> for num in range(2, 10):
    ...     if num % 2 == 0:
    ...         print "Found an even number", num
    ...         continue
    ...     print "Found a number", num
    ...
    Found an even number 2
    Found a number 3
    Found an even number 4
    Found a number 5
    Found an even number 6
    Found a number 7
    Found an even number 8
    Found a number 9
\end{minted}

\subsection{\texttt{pass} Statements}
The \texttt{pass} statement does nothing. It can be used when a statement is required syntactically
but the program requires no action. For example:
\begin{minted}{bash}
    >>> while True:
    ...     pass    # Busy-wait for keyboard interrupt (Ctrl + C)
    ...
    ^CTraceback (most recent call last):
    File "<stdin>", line 1, in <module>
    KeyboardInterrupt
\end{minted}

This is commonly used for creating minimal classes:
\begin{minted}{bash}
    >>> class MyEmpthClass:
    ...     pass
    ...
\end{minted}

Another place \texttt{pass} can be used is as a place-holder for a function or conditional body
when you are working on new code, allowing you to keep thinking at a more abstract level. The
\texttt{pass} is silently ignored:
\begin{minted}{bash}
    >>> def initlog(*args):
    ...     pass    # Remember to implement this!
    ...
\end{minted}

\subsection{Defining Functions}
We can create a function that writes the Fibonacci series to an arbitrary boundary:
\begin{minted}{bash}
    >>> def fib(n):     # write Fibonacci series up to n
    ...     """Print a Fibonacci series up to n."""
    ...     a, b = 0, 1
    ...     while a < n:
    ...         print a,
    ...         a, b = b, a + b
    ...
    >>> # Now call the function we just defined:
    ... fib(2000)
    0 1 1 2 3 5 8 13 21 34 55 89 144 233 377 610 987 1597
\end{minted}

The keyword \texttt{def} introduces a function \emph{definition}. It must be followed by the
function name and the parenthesized list of formal parameters. The statements that form the body of
the function start at the next line, and must be indented.

The first statement of the function body can optionally be a string literal; this string literal is
the function's documentation string, or \emph{docstring}. There are tools which use docstrings to
automatically produce online or printed documentation, or to let the user interactively browse
through code; it's a good practice to include docstrings in code that you write, so make a habit of
it.

The \texttt{execution} of a function introduces a new symbol table used for the local variables of
the function. More precisely, all variable assignments in a function store the value in the local
symbol table; whereas variable references first look in the local symbol table, then in the local
symbol tables of enclosing functions, then in the global symbol table, and finally in the table of
built-in names. Thus, global variables cannot be directly assigned a value within a function
(unless named in a \texttt{global} statement), although they may be referenced.

The actual parameters (arguments) to a function call are introduced in the local symbol table of
the called function when it is called; thus, arguments are passed using \emph{call by value} (where
the \emph{value} is always an object \emph{reference}, not the value of the object). When a
function calls another function, a new local symbol table is created for that call.

A function definition introduces the function name in the current symbol table. The value of the
function name has a type that is recognized by the interpreter as a userdefined function. This
value can be assigned to another name which can be used as a function. This serves as a general
renaming mechanism:
\begin{minted}{bash}
    >>> fib
    <function fib at 0x106cd8140>
    >>> f = fib
    >>> f(100)
    0 1 1 2 3 5 8 13 21 34 55 89
\end{minted}

Coming from other languages, you might object that \texttt{fib} is not a function but a procedure
since it doesn't return a value. In fact, even functions without a \texttt{return} statement do
return a value, albeit a rather boring one. This value is called \texttt{None} (It's a built-in
name). Writing the value \texttt{None} is normally suppressed by the interpreter if it would be the
only value written. You can see if you really want to using \texttt{print}:
\begin{minted}{bash}
    >>> fib(0)
    >>> print fib(0)
    None
\end{minted}

It is simple to write a function that returns a list of the numbers of the Fibonacci series,
instead of printing it:
\begin{minted}{bash}
    >>> def fib2(n):
    ...     """Return a list containing the Fibonacci series up to n."""
    ...     result = []
    ...     a, b = 0, 1
    ...     while a < n:
    ...         result.append(a)
    ...         a, b = b, a + b
    ...     return result
    ...
    >>> f100 = fib2(100)
    >>> f100
    [0, 1, 1, 2, 3, 5, 8, 13, 21, 34, 55, 89]
\end{minted}
This example, as usual, demonstrates some new Python features:
\begin{itemize}
    \item The \texttt{return} statement returns with a value from a function. \texttt{return}
    without an expression argument returns \texttt{None}. Falling off the end of a function also
    returns \texttt{None}.
    \item The statement \texttt{result.append(a)} calls a method of the list object \texttt{result}.
    A method is a function that 'belongs' to an object is named \texttt{obj.methodname}, where
    \texttt{obj} is some object (this may be an expression), and \texttt{methodname} is the name of
    a method that is defined by the object's type. \\
    The method \texttt{append()} shown in the example is defined for list objects; it adds a new
    element at the end of the list. In this example it is equivalent to
    \texttt{result = result + [a]}, but more efficient.
\end{itemize}

\subsection{More on Defining Functions}
It is also possible to define functions with a variable number of arguments. There are three forms,
which can be comined.
\subsubsection{Default Arguments Values}
The most useful form is to specify a default value for one or more arguments. The default arguments
can't be followed by non-default arguments. This creates a function that can be called with fewer
arguments than it is defined to allow. For example:
\begin{minted}{bash}
    >>> def ask_ok(prompt, retries = 4, complaint = "Yes or no, please!"):
    ...     while True:
    ...         ok = raw_input(prompt)
    ...         if ok in ('y', 'ye', 'yes'):
    ...             return True
    ...         if ok in ('n', 'no', 'nop', 'nope'):
    ...             return False
    ...         retries = retries - 1
    ...         if retries < 0:
    ...             raise IOError('refusenik user')
    ...         print complaint
    ...
\end{minted}
This function can be called in several ways:
\begin{itemize}
    \item giving only the madatory argument: \\
    \texttt{ask\_ok('Do you really want to quit?')}
    \item giving one of the optional arguments: \\
    \texttt{ask\_ok('OK to overwrite the file?', 2)}
    \item or even giving all arguments: \\
    \texttt{ask\_ok('OK to overwrite the file?', 2, 'Come on, only yes or no!')}
\end{itemize}

The keyword \texttt{in} can test whether or not a sequence contains a certain value.

The default values are evaluated at the point of function definition in the \emph{defining} scope.
In other words, default parameter values are always evaluated when, and only when, the \texttt{def}
statement they belong to is executed, so that
\begin{minted}{bash}
    >>> i = 5
    >>> def f(arg = i):
    ...     print arg
    ...
    >>> f()
    5
    >>> i = 6
    >>> f()
    5
\end{minted}

\textbf{Important warnings: The default value is evaluated only once.}
\footnote{http://effbot.org/zone/default-values.htm} This makes a difference when
the default is a mutable object
\footnote{https://codehabitude.com/2013/12/24/python-objects-mutable-vs-immutable/}
such as a list, dictionary, or instances of most classes. For example, the following function
accumulates the arguments passed to it on subsequent calls:
\begin{minted}{bash}
>>> def f(a, L = []):
    ...     L.append(a)
    ...     return L
    ...
    >>> print f(1)
    [1]
    >>> print f(2)
    [1, 2]
    >>> print f(3)
    [1, 2, 3]
\end{minted}

If you don't want the default to be shared between subsequent calls, you can write the function
like this instead:
\begin{minted}{bash}
    >>> def f(a, L = None):
    ...     if L is None:
    ...         L = []
    ...     L.append(a)
    ...     return L
    ...
    >>> print f(1)
    [1]
    >>> print f(2)
    [2]
    >>> print f(3)
    [3]
\end{minted}

\subsubsection{Keyword Arguments}
Functions can also be called using keyword arguments of the form \texttt{kwarg = value}.

\begin{minted}{bash}
    >>> def parrot(voltage, state = 'a stiff', action = 'voom', type = 'Norwegian Blue'):
    ...     print "-- This parrot wouldn't", action,
    ...     print "if you put", voltage, "volts through it."
    ...     print "-- Lovely plumage, the", type
    ...     print "-- It's", state, "!"
    ...
\end{minted}

This function can be called in any of the following ways:
\begin{minted}{bash}
    parrot(1000)                                          # 1 positional argument
    parrot(voltage=1000)                                  # 1 keyword argument
    parrot(voltage=1000000, action='VOOOOOM')             # 2 keyword arguments
    parrot(action='VOOOOOM', voltage=1000000)             # 2 keyword arguments
    parrot('a million', 'bereft of life', 'jump')         # 3 positional arguments
    parrot('a thousand', state='pushing up the daisies')  # 1 positional, 1 keyword
\end{minted}
but all the following calls would be invalid:
\begin{minted}{bash}
    parrot()                     # required argument missing
    parrot(voltage=5.0, 'dead')  # non-keyword argument after a keyword argument
    parrot(110, voltage=220)     # duplicate value for the same argument
    parrot(actor='John Cleese')  # unknown keyword argument
\end{minted}

When a final formal parameter of the form \texttt{**name} is present, it receives a dictionary
containing all keyword arguments except for those corresponding to a formal parameter. This may be
contained with a formal parameter of the form \texttt{*name} which receives a tuple containing the
positional arguments beyond the formal parameter list. (\texttt{*name} must occur before
\texttt{**name}.) For example, if we define a function like this:
\begin{minted}{bash}
    >>> def cheeseshop(kind, *arguments, **keywords):
    ...     print "-- Do you have any", kind + "?"
    ...     print "-- I'm sorry, we're all out of", kind
    ...     for arg in arguments:
    ...             print arg
    ...     print "-" * 40
    ...     keys = sorted(keywords.keys())
    ...     for kw in keys:
    ...             print kw, ":", keywords[kw]
    ...
    >>> cheeseshop("Limburger", "It's very runny, sir.",
    ...             "It's really very, VERY runny, sir.",
    ...             shopkeeper = 'Michael Palin',
    ...             client = "John Cleese",
    ...             sketch = "Cheese Shop Sketch")
    -- Do you have any Limburger?
    -- I'm sorry, we're all out of Limburger
    It's very runny, sir.
    It's really very, VERY runny, sir.
    ----------------------------------------
    client : John Cleese
    shopkeeper : Michael Palin
    sketch : Cheese Shop Sketch
\end{minted}

Note that the list of keyword argument names is created by sorting the result of the keywords
dictionary's \texttt{keys()} method before printing its contents; if this is not done, the order in
which the arguments are printed is undefined.

\subsubsection{Unpacking Arugument Lists}
The reverse situation occurs when the arguments are already in a list or tuple but need to be
unpacked for a function call requiring separate positional arguments. For instances, the built-in
\texttt{range()} function expects separate \emph{start} and \emph{stop} arguments. If they are not
available separately, write the function call with the \texttt{*}-operator to unpack the arguments
out of a list or tuple:
\begin{minted}{bash}
    >>> range(3, 6)
    [3, 4, 5]
    >>> args = [3, 6]
    >>> range(*args)
    [3, 4, 5]
\end{minted}

In the same fashion, dictionaries can deliver keyword arguments with the \texttt{**}-operator:
\begin{minted}{bash}
    >>> def parrot(voltage, state = 'a stiff', action = 'voom'):
    ...     print "-- This parrot wouldn't", action,
    ...     print "if you put", voltage, "volts through it.",
    ...     print "E's", state + "!"
    ...
    >>> d = {"voltage" : "four million", "state" : "blledin' demised", "action" : "VOOM"}
    >>> parrot(**d)
    -- This parrot wouldn't VOOM if you put four million volts through it. E's blledin' demised!
\end{minted}

\subsubsection{Lambda Expressions}
Small anonymous functions can be created with the \texttt{lambda} keyword. This function returns
the sum of its two arguments: \mintinline{Python}{lambda a, b : a + b}. Lambda functions can be
used wherever function objects are required. They are syntactically restricted to a single
expression. Semantically, they are just syntactic sugar for a normal function definition. Like
nested function definitions, lambda functions can reference variables from the containing scope:
\begin{minted}{bash}
    >>> def make_incrementor(n):
    ...     return lambda x : x + n     # return a function
    ...
    >>> f = make_incrementor(42)
    >>> f(0)
    42
    >>> f(1)
    43
\end{minted}

The above example uses a lambda expression to return a function. Another use is to pass a small
function as an argument:
\begin{minted}{bash}
    >>> pairs = [(1, 'one'), (2, 'two'), (3, 'three'), (4, 'four')]
    >>> pairs.sort(key = lambda pair : pair[1])
    >>> pairs
    [(4, 'four'), (1, 'one'), (3, 'three'), (2, 'two')]
\end{minted}

\subsubsection{Documentation Strings}
There are emerging conventions about the content and formatting of documentation strings.

The first line should always be a short, concise summary of the object's purpose. For brevity, it
should not explicitly state the object's name or type, since these are available by other means
(except if the name happens to be a verb describing a function's operation). This line should begin
with a capital letter and end with a period.

If there are more lines in the documentation string, the seocnd line should be blank, visually
separating the summary from the rest of the description. The following lines should be one or more
paragraphs describing the object's calling conventions, its side effects, etc.

The Python parser does not strip indentation from multi-line string literals in Python, so tools
that process documentation have to strip indentation if desired. This is done using the following
convention. The first non-blank line \emph{after} the first line of the string determines the
amount of indentation for the entire documentation string. Whitespace ``equivalent'' to this
indentation is then stripped from the start of all lines of the string.

Here is an example of a multi-line docstring:
\begin{minted}{bash}
    >>> def my_function():
    ...     """Do nothing, but document it.
    ...
    ...     No, really, it doesn't do anything.
    ...     """
    ...     pass
    ...
    >>> print my_function.__doc__
    Do nothing, but document it.

            No, really, it doesn't do anything.

\end{minted}

\section{Intermezzo: Coding Style}
Making it easy for others to read your code is always a good a idea, and adopting a nice coding
style helps tremendously for that.

For Python, PEP 8 \footnote{https://www.python.org/dev/peps/pep-0008} has emerged as the style
guide that most projects adhere to; it promotes a very readable and eye-pleasing coding style.
Every Python developer should read it at some point; here are the most important points extracted:
\begin{itemize}
    \item Use 4-space indentation, and no tabs.
    \item Wrap lines so that they don't exceed 79 characters.
    \item Use blank lines to separate functions and classes, and larger blocks of code inside
    functions.
    \item When possible, put comments on a line of their own.
    \item Use docstrings.
    \item Use spaces around operators and after commas, but not directly inside bracketing
    constructs: \texttt{a = f(1, 2) + g(3, 4)}.
    \item Name your classes and functions consistently; the convention is to use \texttt{CamelCase}
    for classes and \texttt{lower\_case\_with\_underscores} for functions and methods. Always use
    \texttt{self} as the name for the first method argument.
    \item Don't use fancy encodings if your code is meant to be used in international environments.
    Plain ASCII works best in any case.
\end{itemize}

\section{Data Structures}
\subsection{More on Lists}
All of the methods of list objects:
\begin{itemize}
    \item \mintinline{Python}{list.append(x)} \\
    Add an item to the end of the list; equivalent to \texttt{a[len(a):] = x}.
    \item \mintinline{Python}{list.extend(L)} \\
    Extend the list by appending all the items in the given list; equivalent to
    \texttt{a[len(a):] = L}
    \item \mintinline{Python}{list.insert(i, x)} \\
    Insert an item at a given position. The first argument is the index of the element before which
    to insert, so \texttt{a.insert(0, x)} inserts at the front of the list, and
    \texttt{a.insert(len(a), x)} is equivalent to \texttt{a.append(x)}.
    \item \mintinline{Python}{list.remove(x)} \\
    Remove the first item from the list whose value is \texttt{x}.
    \item \mintinline{Python}{list.pop([i])} \\
    Remove the item at the given position in the list, and return it. If no index is specified,
    \texttt{a.pop()} removes and returns the last item in the list. (The square brackets around the
    \texttt{i} in the method signature denote that the parameter is optional, not that you should
    type square brackets at that position. You will see this notation frequently in the Python
    Library Reference.)
    \item \mintinline{Python}{list.index(x)} \\
    Return the index in the list of the first item whose value is \texttt{x}. It is an error if
    there is no such item.
    \item \mintinline{Python}{list.count(x)} \\
    Return the number of times \texttt{x} appears in the list.
    \item \mintinline{Python}{list.sort(cmp = None, key = None, reverse = False)} \\
    Sort the items of the list in place (the arguments can be used for sort customization, see
    \texttt{sorted} \footnote{https://docs.python.org/2.7/library/functions.html\#sorted} for their
    explanation).
    \item \mintinline{Python}{list.reverse()} \\
    Reverse the elements of the list, in place.
\end{itemize}

An example that uses most of the list methods:
\begin{minted}{bash}
    >>> a = [66.25, 333, 333, 1, 1234.5]
    >>> print a.count(333), a.count(66.25), a.count('x')
    2 1 0
    >>> a.insert(2, -1)
    >>> a.append(333)
    >>> a
    [66.25, 333, -1, 333, 1, 1234.5, 333]
    >>> a.index(333)
    1
    >>> a.remove(333)
    >>> a
    [66.25, -1, 333, 1, 1234.5, 333]
    >>> a.reverse()
    >>> a
    [333, 1234.5, 1, 333, -1, 66.25]
    >>> a.sort()
    >>> a
    [-1, 1, 66.25, 333, 333, 1234.5]
    >>> a.pop()
    1234.5
    >>> a
    [-1, 1, 66.25, 333, 333]
\end{minted}

You might have noticed that methods like \texttt{insert}, \texttt{remove} or \texttt{sort} that
only modify the list have no return value printed --- they return the default \texttt{None}. This
is a design principle for all mutable data structures in Python.

\subsubsection{Using Lists as Stacks}
The list methods make it very easy to use a list as a stack, where the last element added is the
first element retrieved (``last-in, first-out''). To add an item to the top of the stack, use
\texttt{append()}. To retrieve an item from the top of the stack, use \texttt{pop()} without an
explicit index. For example:
\begin{minted}{bash}
    >>> stack = [3, 4, 5]
    >>> stack.append(6)
    >>> stack.append(7)
    >>> stack
    [3, 4, 5, 6, 7]
    >>> stack.pop()
    7
    >>> stack
    [3, 4, 5, 6]
    >>> stack.pop()
    6
    >>> stack.pop()
    5
    >>> stack
    [3, 4]
\end{minted}

\subsubsection{Using Lists as Queues}
It is also possible to use a list as a queue, where the first element added is the first element
retrieved (``first-in, first-out''); however, lists are not efficient for this purpose. While
appends and pops from the end of list are fast, doing inserts or pops from the beginning of a list
is slow (because all the other elements have to be shifted by one).

To implement a queue, use \texttt{collections.deque} which was designed to have fast appends and
pops from both ends. For example:
\begin{minted}{bash}
>>> from collections import deque
    >>> queue = deque(["Eric", "John", "Michael"])
    >>> queue.append("Terry")       # Terry arrives
    >>> queue.append("Graham")      # Graham arrives
    >>> queue.popleft()             # The first to arrive now leaves
    'Eric'
    >>> queue.popleft()             # The second to arrive now leaves
    'John'
    >>> queue                       # Remaining queue in order of arrival
    deque(['Michael', 'Terry', 'Graham'])
\end{minted}

\subsubsection{Functional Programming Tools}
There are three built-in functions that are very useful when used with lists: \texttt{filter()},
\texttt{map()}, and \texttt{reduce()}.

\texttt{filter(function, sequence)} returns a sequence consisting of those item from the sequence
for which \texttt{function(item)} is true. If \texttt{sequence} is a \texttt{str}, \texttt{unicode}
or \texttt{tuple}, the result will be of the same type; otherwise, it is always a \texttt{list}.
For example, to compute a sequence of numbers divisible by 3 or 5:
\begin{minted}{bash}
    >>> def f(x) : return x % 3 == 0 or x % 5 == 0
    ...
    >>> filter(f, range(2, 25))
    [3, 5, 6, 9, 10, 12, 15, 18, 20, 21, 24]
\end{minted}

\texttt{map(function, sequence)} calls function(item) for each of the sequence's items and returns
a list of teh return values. For example, to compute some cubes:
\begin{minted}{bash}
    >>> def cube(x) : return x * x * x
    ...
    >>> map(cube, range(1, 11))
    [1, 8, 27, 64, 125, 216, 343, 512, 729, 1000]
\end{minted}

More than one sequence may be passed; the function must then have as many arguments as there are
sequences and is called with the corresponding item from each sequence (or \texttt{None} if some
sequence is shorter than another). For example:
\begin{minted}{bash}
    >>> seq = range(8)
    >>> def add(x, y) : return x + y
    ...
    >>> map(add, seq, seq)
    [0, 2, 4, 6, 8, 10, 12, 14]
\end{minted}

\texttt{reduce(function, sequence)} returns a single value constructed by calling the binary
function \texttt{function} on the first two items of teh sequence, then on the result and the next
items, and so on. For examples, to compute the sum of the numbers 1 through 10:
\begin{minted}{bash}
    >>> def add(x, y) : return x + y
    ...
    >>> reduce(add, range(1, 11))
    55
    >>> reduce(add, range(1,2))
    1
    >>> reduce(add, range(1,1))
    Traceback (most recent call last):
      File "<stdin>", line 1, in <module>
    TypeError: reduce() of empty sequence with no initial value
\end{minted}

A third argument can be passed to indicate the starting value. In this case the starting value is
returned for an empty sequence, and the function is first applied to the starting value and the
first sequence item, then to the result and the next item, and so on. For example,
\begin{minted}{bash}
    >>> def sum(seq):
    ...     def add(x, y) : return x + y
    ...     return reduce(add, seq, 0)
    ...
    >>> sum(range(1, 11))
    55
    >>> sum([])
    0
\end{minted}

Don't use this example's definition of \texttt{sum()}: since summing nubmers is such a common need,
a built-in function \texttt{sum(sequence)} is already provided, and works exactly like this.

\subsubsection{List Comprehensions}
List comprehensions provide a concise way to create lists. Common applications are to make new
lists where each element is teh result of some operations applied to each member of another
sequence or iterable, or to create a subsequence of those elements that satisfy a certain condition.

For example, assume we want to create a list of squares, like:
\begin{minted}{bash}
>>> squares = []
>>> for x in range(10):
...     squares.append(x**2)
...
>>> squares
[0, 1, 4, 9, 16, 25, 36, 49, 64, 81]
\end{minted}

We can obtain the same result with:
\begin{minted}{bash}
    >>> squares = [x**2 for x in range(10)]
    >>> squares
    [0, 1, 4, 9, 16, 25, 36, 49, 64, 81]
\end{minted}
This is also equivalent to \mintinline{Python}{squares = map(lambda x : x ** 2, range(10))}, but
it's more concise and readable.

A list comprehension consists of brackets containing an expression followed by a \texttt{for}
clause, then zero or more \texttt{for} or \texttt{if} clauses. The result will be a new list
resulting from evaluating the expression in the context of the \texttt{for} and \texttt{if} clauses
which follow it. For example, this listcomp combines the elements of two lists if they are not
equal:
\begin{minted}{bash}
    >>> [(x, y) for x in [1, 2, 3] for y in [3, 1, 4] if x != y]
    [(1, 3), (1, 4), (2, 3), (2, 1), (2, 4), (3, 1), (3, 4)]
\end{minted}
and it's equivalent to:
\begin{minted}{bash}
    >>> combs = []
    >>> for x in [1, 2, 3]:
    ...     for y in [3, 1, 4]:
    ...         if x != y:
    ...             combs.append((x, y))
    ...
    >>> combs
    [(1, 3), (1, 4), (2, 3), (2, 1), (2, 4), (3, 1), (3, 4)]
\end{minted}

If the expression is a tuple (e.g. the \texttt{(x, y)} in the previous example), it must be
parenthesized.
\begin{minted}{bash}
    >>> vec = [-4, -2, 0, 2, 4]
    >>> # create a new list with the values doubled
    ... [x*2 for x in vec]
    [-8, -4, 0, 4, 8]
    >>> # filter the list to exclude negative numbers
    ... [x for x in vec if x >= 0]
    [0, 2, 4]
    >>> # apply a function to all the elements
    ... [abs(x) for x in vec]
    [4, 2, 0, 2, 4]
    >>> # call a method on each element
    ... freshfruit = ['  banana', '  loganberry ', 'passion fruit   ']
    >>> [weapon.strip() for weapon in freshfruit]
    ['banana', 'loganberry', 'passion fruit']
    >>> # create a list of 2-tuples like (nubmer, square)
    ... [(x, x**2) for x in range(6)]
    [(0, 0), (1, 1), (2, 4), (3, 9), (4, 16), (5, 25)]
    >>> # the tuple must be parenthesized, otherwise an error is raised
    ... [x, x**2 for x in range(6)]
      File "<stdin>", line 2
        [x, x**2 for x in range(6)]
                   ^
    SyntaxError: invalid syntax
    >>> # flatten a list using a listcomp with two 'for'
    ... vec = [[1, 2, 3], [4, 5, 6], [7, 8, 9]]
    >>> [num for elem in vec for num in elem]
    [1, 2, 3, 4, 5, 6, 7, 8, 9]
\end{minted}

List comprehension can contain complex expressions and nested functions:
\begin{minted}{bash}
    >>> from math import pi
    >>> [str(round(pi, i)) for i in range(1, 6)]
    ['3.1', '3.14', '3.142', '3.1416', '3.14159']
\end{minted}

The initial expression in a list comprehension can be any arbitrary expression, including another
list comprehension.

Consider the following example of a 3$\times$4 matrix implemented as a list of 3 lists of length 4:
\begin{minted}{bash}
    >>> matrix = [
    ...     [1, 2, 3, 4],
    ...     [5, 6, 7, 8],
    ...     [9, 10, 11, 12]
    ... ]
\end{minted}

The following list comprehension will transpose rows and columns:
\begin{minted}{bash}
    >>> [[row[i] for row in matrix] for i in range(4)]
    [[1, 5, 9], [2, 6, 10], [3, 7, 11], [4, 8, 12]]
\end{minted}

As we saw in the previous section, the nested listcomp is evaluated in the context of the
\texttt{for} that follows it, so this example is equivalent to:
\begin{minted}{bash}
    >>> transposed = []
    >>> for i in range(4):
    ...     transposed.append([row[i] for row in matrix])
    ...
    >>> transposed
    [[1, 5, 9], [2, 6, 10], [3, 7, 11], [4, 8, 12]]
\end{minted}

which, in turn, is the same as:
\begin{minted}{bash}
    >>> transposed = []
    >>> for i in range(4):
    ...     # the following 3 lines implement the nested listcomp
    ...     transposed_row = []
    ...     for row in matrix:
    ...         transposed_row.append(row[i])
    ...     transposed.append(transposed_row)
    ...
    >>> transposed
    [[1, 5, 9], [2, 6, 10], [3, 7, 11], [4, 8, 12]]
\end{minted}

In the real world, you should prefer built-in functions to complex flow statements. The
\texttt{zip()} function would do a great job for this use case:
\begin{minted}{bash}
    >>> zip(*matrix)  # use * to unpack the arguments from matrix
    [(1, 5, 9), (2, 6, 10), (3, 7, 11), (4, 8, 12)]
\end{minted}

\subsection{The \texttt{del} Statement}
There is a way to remove an item from a list given its index instead of its value: the \texttt{del}
statement. This differs from the \texttt{pop()} method which returns a value. The \texttt{del}
statement can also be used to remove slices from a list or clear the entire list (which we did
earlier by assignment of an empty list to the slice). For example:
\begin{minted}{bash}
    >>> a = [-1, 1, 66.25, 333, 333, 1234.5]
    >>> del a[0]
    >>> a
    [1, 66.25, 333, 333, 1234.5]
    >>> del a[2:4]
    >>> a
    [1, 66.25, 1234.5]
    >>> del a[:]
    >>> a
    []
\end{minted}

\texttt{del} can also be used to delete entire variables. Referencing the name \texttt{a} hereafter
is an error (at least until another value is assigned to it):
\begin{minted}{bash}
    >>> del a
    >>> a
    Traceback (most recent call last):
      File "<stdin>", line 1, in <module>
    NameError: name 'a' is not defined
\end{minted}

\subsection{Tuples and Sequences}
A tuple consists of a number of values separated by commas, for instance:
\begin{minted}{bash}
    >>> t = 12345, 54321, 'hello!'
    >>> t[0]
    12345
    >>> t
    (12345, 54321, 'hello!')
    >>> # Tuples may be nested
    ... u = t, (1, 2, 3, 4, 5)
    >>> u
    ((12345, 54321, 'hello!'), (1, 2, 3, 4, 5))
    >>> # Tuples are immutable
    ... t[0] = 88888
    Traceback (most recent call last):
      File "<stdin>", line 2, in <module>
    TypeError: 'tuple' object does not support item assignment
    >>> # but they can contain mutable objects
    ... v = ([1, 2, 3], [3, 2, 1])
    >>> v
    ([1, 2, 3], [3, 2, 1])
\end{minted}

As you see, on output tuples are always enclosed in parentheses, so that nested tuples are
interpreted correctly; they may be input with or without surrounding parentheses, although often
parentheses are necessary anyway (if the tuple is part of a larger expression). It is not possible
to assign to the individual items of a tuple, however it is possible to create tuples which contain
mutable objects, such as lists.

Though tuples may seem similar to lists, they are often used in different situations and for
different purposes. Tuples are immutable, and usually contain a heterogeneous sequence of elements
that are accessed via unpacking or indexing (or even by attribute in the case of
\texttt{namedtuples}). Lists are mutable, and their elements are usually homogeneous and are
accessed by iterating over the list.

A special problem is the construction of tuples containing 0 or 1 items: the syntax has some extra
quirks to accommodate these. Empty tuples are constructed by an empty pair of parentheses; a tuple
with one item is constructed by following a value with a comma (it is not sufficient to enclose a
single value in parentheses). Ugly, but effective. For example:
\begin{minted}{bash}
    >>> empty = ()
    >>> singleton = 'hello',    # <-- note trailing comma
    >>> len(singleton)
    1
    >>> singleton
    ('hello',)
    >>> singleton = 'hello'     # the type of singleton here is 'str'
    >>> len(singleton)
    5
    >>> singleton
    'hello'
\end{minted}

The statement \texttt{t = 12345, 54321, 'hello!'} is an example of \emph{tuple packing}: the values
\texttt{12345}, \texttt{54321} and \texttt{'hello!'} are packed together in a tuple. The reverse
operation is also possible:
\begin{minted}{bash}
    >>> x, y, z = t
    >>> x
    12345
    >>> y
    54321
    >>> z
    'hello!'
\end{minted}

This is called, appropriately enough, \emph{sequence unpacking} and works for any sequence on the
right-hand side. Sequence unpacking requires the list of variables on the left to have the same
number of elements as the length of the sequence. Note that multiple assignment is really just a
combination of tuple packing and sequence unpacking.

\subsection{Sets}
Python also includes a data type for \emph{sets}. A set is an unsorted collection with no duplicate
elements. Basic uses include membership testing and eliminating duplicate entries. Set objects also
support mathematical operations like union, intersection, difference, and symmetric difference.

Curly braces or the \texttt{set()} function can be used to create sets. Note: to create an empty
set you have to use \texttt{set()}, not \texttt{\{\}}; the latter creates an empty dictionary.

Here is a brief demonstration:
    \begin{minted}{bash}
    >>> basket = ['apple', 'orange', 'apple', 'pear', 'orange', 'banana']
    >>> fruit = set(basket)     # create a set without duplicates
    >>> fruit
    set(['orange', 'pear', 'apple', 'banana'])
    >>> 'orange' in fruit       # fast membership testing
    True
    >>> 'crabgrass' in fruit
    False

    >>> # Demonstrate set operations on unique letters from two words
    ...
    >>> a = set('abracadabra')
    >>> b = set('alacazam')
    >>> a
    set(['a', 'r', 'b', 'c', 'd'])
    >>> b
    set(['a', 'c', 'z', 'm', 'l'])
    >>> a - b
    set(['r', 'b', 'd'])
    >>> a | b
    set(['a', 'c', 'b', 'd', 'm', 'l', 'r', 'z'])
    >>> a & b
    set(['a', 'c'])
    >>> a ^ b
    set(['b', 'd', 'm', 'l', 'r', 'z'])
\end{minted}

Similarly to list comprehensions, set comprehensions are also supported:
\begin{minted}{bash}
    >>> a = {x for x in 'abracadabra' if x not in 'abc'}
    >>> a
    set(['r', 'd'])
\end{minted}








\end{document}
