%-*- coding: UTF-8 -*-
% notes.tex
%
\documentclass[UTF8]{article}
\usepackage{geometry}
\geometry{a4paper, centering, scale=0.8}
\usepackage{minted}
\usepackage{hyperref}
\usepackage{indentfirst} % to indent the first paragraph of a section

\title{The Python Tutorial Notes}
\author{Du Ang \\ \texttt{du2ang233@gmail.com} }
\date{\today}

\begin{document}
\maketitle

\tableofcontents
\newpage

\section{Whetting Your Appetite}
\subsection{Some applications of Python}
\begin{itemize}
    \item Automate works on computer
    \item Help software engineers develop and test more quickly
    \item ...
\end{itemize}

\subsection{The features of Python}
\begin{itemize}
    \item High-level language \\
    High-level more general data types
    \item Allow to split program into modules
    \item Interpreted language
    \begin{itemize}
        \item No necessary compilation and linking
        \item The interpreter can be used interactively
    \end{itemize}
    \item Shorter programs than equivalent C, C++ or Java programs
    \begin{itemize}
        \item high-level data types
        \item statement grouping is done by indentation instead of brackets
        \item no variable or argument declarations are necessary
    \end{itemize}
    \item Extensible
\end{itemize}

\subsection{The origin of the name "Python"}
“Monty Python’s Flying Circus”, the skits on BBC.

\section{Using the Python Interpreter}
\subsection{Invoking the Interpreter}
The Python interpreter is usually installed as \texttt{/usr/local/bin/python}, and usually the
\texttt{/usr/local/bin} has been put in the Unix shell’s search path. Type command \texttt{python}
to the shell to invoke the Python interpreter.

Type an end-of-file character(\texttt{Ctrl-D} on Unix, \texttt{Ctrl-Z} on Windows) to exit the
interpreter. If that doesn’t work, type command \mintinline{Python}{quit()} to the shell.

A second way of starting the interpreter is \mintinline{bash}{python -c 'command [arg] ...'},
which executes the statement(s) in command.

Some Python modules are also useful as scripts. These can be invoked using
\mintinline{bash}{python -m module [arg] ...}, which executes the source file for modules as if you
had spelled out its full name on the command line.

\subsubsection{Argument Passing}
When known to the interpreter, the script name and additional arguments thereafter are turned into
a list of strings and assigned to the \texttt{argv} variable in the \texttt{sys} module. You can
access this list by executing \texttt{import sys}. The length of the list is at least one; when no
script and no arguments are given, \texttt{sys.argv[0]} is an empty string. When the script name is
given as \texttt{'-'} (meaning standard input), \texttt{sys.argv[0]} is set to \texttt{'-'}. When
\texttt{-c} command is used, \texttt{sys.argv[0]} is set to \texttt{'-c'}. When \texttt{-m} module
is used, \texttt{sys.argv[0]} is set to the full name of the located module. Options found after
\texttt{-c} command or \texttt{-m} module are not consumed by the Python interpreter’s option
processing but left in \texttt{sys.argv} for the command or module to handle.

\subsubsection{Interactive Mode}
\begin{itemize}
    \item \emph{primary prompt}: \texttt{>>>}
    \begin{minted}{bash}
    Python 2.7.10 (default, Feb  7 2017, 00:08:15)
    Type "help", "copyright", "credits" or "license" for more information.
    >>>
    \end{minted}
    \item \emph{secondary prompt}: \texttt{...}
    \begin{minted}{bash}
    >>> the_world_is_flat = 1
    >>> if the_world_is_flat:
    ...     print "Be careful not to fall off!"
    ...
    Be careful not to fall off!
    \end{minted}
\end{itemize}

\subsection{The Interpreter and Its Environment}
\subsubsection{Source Code Encoding}
By default, Python source files are treated as encoded in UTF-8.

To declare an encoding other than the default one, a special comment line should be added as the
first line of the file. The syntax is as follows:
\begin{minted}{Python}
    # -*- coding: encoding -*-
\end{minted}
where \emph{encoding} is one of the valid \texttt{codes} supported by Python.

\section{An Informal Introduction to Python}
In the following examples, input and output are distinguished by the presence or absence of prompts
(\texttt{>>>} and \texttt{...}).

Comments in Python start with the hash character \texttt{\#}, and extend to the end of the physical
line. A hash character in a string literal is just a hash character.

\begin{minted}{Python}
    # this is the first comment
    spam = 1    # and this the second comment
                # and now a third!
    text = "# This is not a comment because it's inside quotes."
\end{minted}

\subsection{Using Python as a Calculator}
\subsubsection{Numbers}
\begin{minted}{bash}
    >>> 2 + 2
    4
    >>> 50 - 5 * 6
    20
    >>> (50 - 5.0 * 6) / 4
    5.0
    >>> 8 / 5.0
    1.6
\end{minted}

The integer numbers(e.g. \texttt{2}, \texttt{4}, \texttt{20}) have type \texttt{int}, the ones with
a fractional part(e.g. \texttt{5.0}, \texttt{1.6}) have type \texttt{float}.

The return type of a division (\texttt{/}) operation depends on its operands. If both operands are
of type \texttt{int}, floor division is performed and an \texttt{int} is returned. If either
operand is a float, classic division is performed and a float is returned. The \texttt{//} operator
is also provided for doing floor divsion no matter what the operands are. The remainder can be
calculated with the \texttt{\%} operator:
\begin{minted}{bash}
    >>> 17 / 3    # int / int -> int
    5
    >>> 17 / 3.0  # int / float -> float
    5.666666666666667
    >>> 17 // 3.0 # explicit floor division discards the fractional part
    5.0
    >>> 5 * 3 + 2 # result * divisor + remainder
    17
\end{minted}

With Python, it is possible to use \texttt{**} operator to calculate powers:
\begin{minted}{bash}
    >>> 5 ** 2  # 5 squared
    25
    >>> 2 ** 7  # 2 to the power of 7
    128
\end{minted}

The equal sign (\texttt{=}) is used to assign a value to a variable. Afterwards, no results is
displayed before the next interactive prompt:
\begin{minted}{bash}
    >>> width = 20
    >>> height = 5 * 9
    >>> width * height
    900
\end{minted}

If a variable is not "defined" (assigned a value), trying to use it will give you an error:
\begin{minted}{bash}
    >>> n   # try to access an undefined variable
    Traceback (most recent call last):
    File "<stdin>", line 1, in <module>
    NameError: name 'n' is not defined
\end{minted}

There is full support for floating point; operators with mixed type operands convert the integer
operand to floating point:
\begin{minted}{bash}
    >>> 3 * 3.75 / 1.5
    7.5
    >>> 7.0 / 2
    3.5
\end{minted}

In interactive mode, the last printed expression is assigned to the variable \texttt{\_}.
\begin{minted}{bash}
    >>> tax = 12.5 / 100
    >>> price = 100.50
    >>> price * tax
    12.5625
    >>> price + _
    113.0625
    >>> round(_, 2)
    113.06
\end{minted}

The variable should be treated as read-only by the user. Don't explicitly assign a value to it ---
you would create an independent local variable with the same name masking the built-in variable with
its magic behavior.

In addition to \texttt{int} and \texttt{float}, Python supports other types of numbers, such as
\texttt{Decimal} and \texttt{Fraction}. Python also has built-in support for complex numbers, and
uses the \texttt{j} or \texttt{J} suffix to indicate the imaginary part (e.g. \texttt{3 + 5j}).

\subsubsection{Strings}
Python can also manipulate strigns, which should be enclosed in single quotes (\texttt{'...'}) or
double quotes (\texttt{"..."}) with the same result. \texttt{\textbackslash} can be used to escape
quotes.
\begin{minted}{bash}
    >>> 'spam eggs'  # single quotes
    'spam eggs'
    >>> 'doesn\'t'   # use \' to escape the single quote...
    "doesn't"
    >>> "doesn't"    # ... or use double quotes instead
    "doesn't"
    >>> '"Yes," he said.'
    '"Yes," he said.'
    >>> "\"Yes,\" he said."
    '"Yes," he said.'
    >>> '"Isn\'t," she said.'
    '"Isn\'t," she said.'
\end{minted}

The \texttt{print} statement produces a more readable output, by omitting the enclosing quotes and
by printing escaped and special characters:
\begin{minted}{bash}
    >>> '"Isn\'t," she said.'
    '"Isn\'t," she said.'
    >>> print '"Isn\'t," she said.'
    "Isn't," she said.
    >>> s = 'First line. \nSecond line.'  # \n means newline
    >>> s       # without print, \n is included in the output
    'First line. \nSecond line.'
    >>> print s # with print, \n produces a new line
    First line.
    Second line.
\end{minted}

If you don't want characters prefaced by \texttt{\textbackslash} to be interpreted as special
characters, you can use \emph{raw strings} by adding an \texttt{r} before the first quote:
\begin{minted}{bash}
    >>> print 'C:\some\name'  # here \n means newline!
    C:\some
    ame
    >>> print r'C:\some\name' # note the r before the quote
    C:\some\name
\end{minted}

String literals can span multiple lines. One way is using triple-quotes: \texttt{"""..."""} or
\texttt{'''...'''}. End of lines are automatically included in the string, but it's possible to
prevent this by adding a \texttt{\textbackslash} at the end of the line:
\begin{minted}{bash}
    >>> print """
    ... Usage: thingy [OPTIONS]
    ...     -h              Display this usage message
    ...     -H hostname     Hostname to connect to
    ... """

    Usage: thingy [OPTIONS]
        -h          Display this usage message
        -H hostname Hostname to connect to

    >>> print """\  # prevent the initial newline
    ... Usage: thingy [OPTIONS]
    ...     -h              Display this usage message
    ...     -H hostname     Hostname to connect to
    ... """
    Usage: thingy [OPTIONS]
        -h          Display this usage message
        -H hostname Hostname to connect to

\end{minted}

Strings can be concatenated (glued together) with \texttt{+} operator, and repeated with \texttt{*}:
\begin{minted}{bash}
    >>> # 3 times 'un', followed by 'ium'
    ... 3 * 'un' + 'ium'
    'unununium'
\end{minted}

Two or more string literals next to each other are automatically concatenated:
\begin{minted}{bash}
    >>> 'Py' 'thon'
    'Python'
\end{minted}

This only works with two literals though, not with variables or expressions. If you want to
concatenate variables or a variable and a literal, use \texttt{+}.
\begin{minted}{bash}
    >>> prefix = 'Py'
    >>> prefix 'thon'  # can't concatenate a variable and string literal
      File "<stdin>", line 1
        prefix 'thon'  # can't concatenate a variable and string literal
                    ^
    SyntaxError: invalid syntax

    >>> prefix + 'thon'
    'Python'
\end{minted}

This feature is particularly useful when you want to break long strings:
\begin{minted}{bash}
    >>> text = ('Put several string within parentheses '
    ...         'to have them joined together.')
    >>> text
    'Put several string within parentheses to have them joined together.'
\end{minted}

Strings can be indexed(subscripted), with the first character having index 0. There is no separate
character type; a character is simple a string a size one:
\begin{minted}{bash}
    >>> word  = 'Python'
    >>> word[0]     # character in position 0
    'P'
    >>> word[5]     # character in position 5
    'n'
\end{minted}

Indices may also be negative nubmers, to start counting from the right:
\begin{minted}{bash}
    >>> word[-1]    # last character
    'n'
    >>> word[-2]    # second-last character
    'o'
    >>> word[-6]
    'P'
\end{minted}
Note that since -0 is the same as 0, negative indices start from -1.

In addition to indexing, \emph{slicing} is also supported. While indexing is used to obtain
individual characters, \emph{slicing} allows you to obtain a substring:
\begin{minted}{bash}
    >>> word[0:2]   # characters from position 0 (included) to 2 (excluded)
    'Py'
    >>> word[2:5]   # characters from position 2 (included) to 5 (excluded)
    'tho'
\end{minted}

Since indices have useful defaults; an omitted first index defaults to zero, an ommited second
index defaults to the size of the string being sliced.
\begin{minted}{bash}
    >>> word[:2]    # character from the beginning to position 2 (excluded)
    'Py'
    >>> word[4:]    # character from the position 4 (included) to the end
    'on'
    >>> word[-2:]   # character from the second-last (included) to the end
    'on'
\end{minted}

Note how the start is always included, and the end always excluded. This makes sure that
\texttt{s[:i] + s[i:]} is always equal to \texttt{s}:
\begin{minted}{bash}
    >>> word[:2] + word[2:]
    'Python'
    >>> word[:4] + word[4:]
    'Python'
\end{minted}

One way to remember how slices work is to think of the indices as pointing \emph{between}
characters, with the left edge of the first character numbered 0. Then the right edge of the last
character of a string of n characters has index \emph{n}, for example:
\begin{minted}{text}
    +---+---+---+---+---+---+
    | P | y | t | h | o | n |
    +---+---+---+---+---+---+
    0   1   2   3   4   5   6
    -6  -5  -4  -3  -2  -1
\end{minted}

Attempting to use an index that is too large will result in an error:
\begin{minted}{bash}
    >>> word[42]
    Traceback (most recent call last):
      File "<stdin>", line 1, in <module>
    IndexError: string index out of range
\end{minted}

However, out of range slice indices are handled gracefully when used for slicing:
\begin{minted}{bash}
    >>> word[4:42]
    'on'
    >>> word[42:]
    ''
\end{minted}

Python strings cannot be changed --- they are immutable. Therefore, assigning to a indexed position
in the string results in an error:
\begin{minted}{bash}
>>> word[0] = 'J'
    Traceback (most recent call last):
      File "<stdin>", line 1, in <module>
    TypeError: 'str' object does not support item assignment

    >>> word[2:] = 'py'
    Traceback (most recent call last):
      File "<stdin>", line 1, in <module>
    TypeError: 'str' object does not support item assignment
\end{minted}

If you need a different string, you should create a new one:
\begin{minted}{bash}
    >>> 'J' + word[1:]
    'Jython'
    >>> word[:2] + 'py'
    'Pypy'
\end{minted}

The built-in fucntion \texttt{len()} returns the length of a string:
\begin{minted}{bash}
    >>> s = 'supercalifragilisticexpialidocious'
    >>> len(s)
    34
\end{minted}

\subsubsection{Unicode Strings}
Unicode has the advantege of providing one ordinal for every character in every script used in
modern and ancient texts.

Creating Unicode strings in Python is just as simple as creating normal strings:
\begin{minted}{bash}
    >>> u'Hello World!'
    u'Hello World!'
\end{minted}

The small \texttt{'u'} in front of the quote indicates that a Unicode string is supposed to be
created. If you want to include special characters in the string, you can do so by using the
Python \emph{Unicode-Escape} encoding. The following example shows how:
\begin{minted}{bash}
    >>> u'Hello\u0020World!'
    u'Hello World!'
\end{minted}
The escape sequence \texttt{\textbackslash u0020} indicates to insert the Unicode character with
the ordinal value 0x0020(the space character) at the given position.

\subsubsection{Lists}
Python knows a number of compound data types, used to group together other values. The most
versatile is the \emph{list}, which can be written as a list of comma-separated values (items)
between square brackets. Lists might contain items of different types, but usually the items all
have the same type.

\begin{minted}{bash}
    >>> squares = [1, 4, 9, 16, 25]
    >>> squares
    [1, 4, 9, 16, 25]
\end{minted}

Like strings (and all other built-in sequence type), lists can be indexed and sliced:
\begin{minted}{bash}
    >>> squares[0]      # indexing returns the item
    1
    >>> squares[-1]
    25
    >>> squares[-3:]    # slicing returns a new list
    [9, 16, 25]
\end{minted}

All slice operations return a new list containing the requested elements. This means that the
following slice returns a new (shallow) copy of the list:
\begin{minted}{bash}
    >>> squares[:]
    [1, 4, 9, 16, 25]
\end{minted}

Lists also supports operations like concatenation:
\begin{minted}{bash}
    >>> squares + [36, 49, 64, 81, 100]
    [1, 4, 9, 16, 25, 36, 49, 64, 81, 100]
\end{minted}

Unlike strings, which are immutable, lists are a mutable type, i.e. it is possible to change their
content:
\begin{minted}{bash}
    >>> cubes = [1, 8, 27, 65, 125] # something's wrong here
    >>> 4 ** 3  # the cube of 4 is 64, not 65!
    64
    >>> cubes[3] = 64   # replace the wrong value
    >>> cubes
    [1, 8, 27, 64, 125]
\end{minted}

You can also add new items at the end of the list, by using the \texttt{append()} method:
\begin{minted}{bash}
    >>> cubes.append(216)           # add the cube of 6
    >>> cubes.append(7 ** 3)        # add the cube of 7
    >>> cubes
    [1, 8, 27, 64, 125, 216, 343]
\end{minted}

Assignment to slices is also possible, and this can even change the size of the list or clear it
entirely:
\begin{minted}{bash}
    >>> letters = ['a', 'b', 'c', 'd', 'e', 'f', 'g']
    >>> letters
    ['a', 'b', 'c', 'd', 'e', 'f', 'g']
    >>> # replace some values
    ... letters[2:5] = ['C', 'D', 'E']
    >>> letters
    ['a', 'b', 'C', 'D', 'E', 'f', 'g']
    >>> # now remove them
    ... letters[2:5] = []
    >>> letters
    ['a', 'b', 'f', 'g']
    >>> # clear the list by replacing all the elements with an empty list
    ... letters[:] = []
    >>> letters
    []
\end{minted}

The built-in function \texttt{len()} also applies to lists:
\begin{minted}{bash}
    >>> letters = ['a', 'b', 'c', 'd']
    >>> len(letters)
    4
\end{minted}

It is possible to nest lists (create lists containing other lists), for example:
\begin{minted}{bash}
    >>> a = ['a', 'b', 'c']
    >>> n = [1, 2, 3]
    >>> x = [a, n]
    >>> x
    [['a', 'b', 'c'], [1, 2, 3]]
    >>> x[0]
    ['a', 'b', 'c']
    >>> x[0][1]
    'b'
    >>> x[1]
    [1, 2, 3]
\end{minted}

\subsection{First Steps Towards Programming}
We can write an initial sub-sequence of the \emph{Fibonacci} series as follows:
\begin{minted}{bash}
    >>> # Fibonacci series:
    ... # the sum of two elements defines the next
    ... a, b = 0, 1     # multiple assignment
    >>> while b < 10:   # loop as long as the condition remains true
    ...     print b
    ...     a, b = b, a + b
    ...
    1
    1
    2
    3
    5
    8
\end{minted}

In Python, like in C, any non-zero integer value is true; zero is false. The condition of the
\texttt{while} loop may also be a string or list value, in fact any sequence; anything with a
non-zero length is true, empty sequences are false. The standard comparision operators are written
the same as in C: \texttt{<} (less than), \texttt{>} (greater than), \texttt{==} (equal to),
\texttt{<=} (less than or equal to), \texttt{>=} (greater than or equal to) and \texttt{!=} (not
equal to).

The \emph{body} of loop is \emph{indented}: indentation is Python's way of grouping statements.
At the interactive prompt, you have to type a tab or space(s) for each indented line.

The \texttt{print} statement writes the value of the expression(s) it is given. It differs from
just writing the expression you want to write in the way it handles multiple expressions and
strings. Strings are pointer without quotes, and a space is inserted between items, so you can
format things nicely, like this:
\begin{minted}{bash}
    >>> i = 256 * 256
    >>> print 'The value of i is', i
    The value of i is 65536
\end{minted}

A trailing comma avoids the newline after the output:
\begin{minted}{bash}
    >>> a, b = 0, 1
    >>> while b < 1000:
    ...     print b,
    ...     a, b = b, a + b
    ...
    1 1 2 3 5 8 13 21 34 55 89 144 233 377 610 987
\end{minted}

\section{More Control Flow Tools}
\subsection{\texttt{if} Statements}
\begin{minted}{bash}
    >>> x = int(raw_input("Please enter an integer: "))
    Please enter an integer: 42
    >>> if x < 0:
    ...     x = 0
    ...     print 'Negative changed to zero'
    ... elif x == 0:
    ...     print 'Zero'
    ... elif x == 1:
    ...     print 'Single'
    ... else:
    ...     print 'More'
    ...
    More
\end{minted}

\subsection{\texttt{for} Statements}
The \texttt{for} statement in Python differs a bit from what you may be used to in C or Pascal.
Rather than always iterating over an arithmetic progression of numbers (like in Pascal), or giving
the user the ability to define both the iteration step and halting condition (as C), Python's
\texttt{for} statement iterates over the items of any sequence (a list or a string), in the order
that they appear in the sequence. For example:
\begin{minted}{bash}
    >>> # Measure some strings:
    ... words = ['cat', 'window', 'defenestrate']
    >>> for w in words:
    ...     print w, len(w)
    ...
    cat 3
    window 6
    defenestrate 12
\end{minted}

If you need to modify the sequence you are iterating over while inside the loop, it is recommended
that you first make a copy. Iterating over a sequence does not implicitly make a copy. The slice
notation makes this especially convenient:
\begin{minted}{bash}
    >>> for w in words[:]:
    ...     if len(w) > 6:
    ...         words.insert(0, w)
    ...
    >>> words
    ['defenestrate', 'cat', 'window', 'defenestrate']
\end{minted}

\subsection{The \texttt{range()} function}
If you do need to iterate over a sequence of numbers, the built-in function \texttt{range()} comes
in handy. It generates lists containing arithmetic progressions:
\begin{minted}{bash}
    >>> range(10)
    [0, 1, 2, 3, 4, 5, 6, 7, 8, 9]
    >>> range(5, 10)    # the given end point is never part of the generated list
    [5, 6, 7, 8, 9]
    >>> range(0, 10, 3) # '3' here is the increment('step'), it can be negative
    [0, 3, 6, 9]
    >>> range(-10, -100, -30)
    [-10, -40, -70]
\end{minted}

To iterate over the indices of a sequence, you can combine \texttt{range()} and \texttt{len()} as
follows:
\begin{minted}{bash}
    >>> a = ['Mary', 'had', 'a', 'little', 'lamb']
    >>> for i in range(len(a)):
    ...     print i, a[i]
    ...
    0 Mary
    1 had
    2 a
    3 little
    4 lamb
\end{minted}

\subsection{\texttt{break} and \texttt{continue} Statements, and \texttt{else} Clauses on Loops}
The \texttt{break()} statement, like in C, breaks out of the innermost enclosing \texttt{for} or
\texttt{while} loop.

Loop statements may have an \texttt{else} clause; it is exected when the loop terminates through
exhaustion of the list (with \texttt{for}) or when the condition becomes false (with
\texttt{while}), but not when the loop is terminated by a \texttt{break} statement. This is
exemplified by the following loop, which searches for prime numbers:
\begin{minted}{bash}
    >>> for n in range(2, 10):
    ...     for x in range(2, n):
    ...         if n % x == 0:
    ...             print n, 'equals', x, '*', n / x
    ...             break
    ...     else:
    ...         # loop fell through without finding a factor
    ...         print n, 'is a prime number'
    ...
    2 is a prime number
    3 is a prime number
    4 equals 2 * 2
    5 is a prime number
    6 equals 2 * 3
    7 is a prime number
    8 equals 2 * 4
    9 equals 3 * 3
\end{minted}

When used with a loop, the \texttt{else} clause has more in common with the \texttt{else} clause of
a \texttt{try} statement than it does that of \texttt{if} statements: a \texttt{try} statement's
\texttt{else} clause runs when no exception occurs, and a loop's \texttt{else} clause runs when no
\texttt{break} occurs.

The \texttt{continue} statement, also borrowed from C, continues with the next iteration of the
loop:
\begin{minted}{bash}
    >>> for num in range(2, 10):
    ...     if num % 2 == 0:
    ...         print "Found an even number", num
    ...         continue
    ...     print "Found a number", num
    ...
    Found an even number 2
    Found a number 3
    Found an even number 4
    Found a number 5
    Found an even number 6
    Found a number 7
    Found an even number 8
    Found a number 9
\end{minted}

\subsection{\texttt{pass} Statements}
The \texttt{pass} statement does nothing. It can be used when a statement is required syntactically
but the program requires no action. For example:
\begin{minted}{bash}
    >>> while True:
    ...     pass    # Busy-wait for keyboard interrupt (Ctrl + C)
    ...
    ^CTraceback (most recent call last):
    File "<stdin>", line 1, in <module>
    KeyboardInterrupt
\end{minted}

This is commonly used for creating minimal classes:
\begin{minted}{bash}
    >>> class MyEmpthClass:
    ...     pass
    ...
\end{minted}

Another place \texttt{pass} can be used is as a place-holder for a function or conditional body
when you are working on new code, allowing you to keep thinking at a more abstract level. The
\texttt{pass} is silently ignored:
\begin{minted}{bash}
    >>> def initlog(*args):
    ...     pass    # Remember to implement this!
    ...
\end{minted}

\subsection{Defining Functions}
We can create a function that writes the Fibonacci series to an arbitrary boundary:
\begin{minted}{bash}
    >>> def fib(n):     # write Fibonacci series up to n
    ...     """Print a Fibonacci series up to n."""
    ...     a, b = 0, 1
    ...     while a < n:
    ...         print a,
    ...         a, b = b, a + b
    ...
    >>> # Now call the function we just defined:
    ... fib(2000)
    0 1 1 2 3 5 8 13 21 34 55 89 144 233 377 610 987 1597
\end{minted}

The keyword \texttt{def} introduces a function \emph{definition}. It must be followed by the
function name and the parenthesized list of formal parameters. The statements that form the body of
the function start at the next line, and must be indented.

The first statement of the function body can optionally be a string literal; this string literal is
the function's documentation string, or \emph{docstring}. There are tools which use docstrings to
automatically produce online or printed documentation, or to let the user interactively browse
through code; it's a good practice to include docstrings in code that you write, so make a habit of
it.

The \texttt{execution} of a function introduces a new symbol table used for the local variables of
the function. More precisely, all variable assignments in a function store the value in the local
symbol table; whereas variable references first look in the local symbol table, then in the local
symbol tables of enclosing functions, then in the global symbol table, and finally in the table of
built-in names. Thus, global variables cannot be directly assigned a value within a function
(unless named in a \texttt{global} statement), although they may be referenced.

The actual parameters (arguments) to a function call are introduced in the local symbol table of
the called function when it is called; thus, arguments are passed using \emph{call by value} (where
the \emph{value} is always an object \emph{reference}, not the value of the object). When a
function calls another function, a new local symbol table is created for that call.

A function definition introduces the function name in the current symbol table. The value of the
function name has a type that is recognized by the interpreter as a userdefined function. This
value can be assigned to another name which can be used as a function. This serves as a general
renaming mechanism:
\begin{minted}{bash}
    >>> fib
    <function fib at 0x106cd8140>
    >>> f = fib
    >>> f(100)
    0 1 1 2 3 5 8 13 21 34 55 89
\end{minted}

Coming from other languages, you might object that \texttt{fib} is not a function but a procedure
since it doesn't return a value. In fact, even functions without a \texttt{return} statement do
return a value, albeit a rather boring one. This value is called \texttt{None} (It's a built-in
name). Writing the value \texttt{None} is normally suppressed by the interpreter if it would be the
only value written. You can see if you really want to using \texttt{print}:
\begin{minted}{bash}
    >>> fib(0)
    >>> print fib(0)
    None
\end{minted}

It is simple to write a function that returns a list of the numbers of the Fibonacci series,
instead of printing it:
\begin{minted}{bash}
    >>> def fib2(n):
    ...     """Return a list containing the Fibonacci series up to n."""
    ...     result = []
    ...     a, b = 0, 1
    ...     while a < n:
    ...         result.append(a)
    ...         a, b = b, a + b
    ...     return result
    ...
    >>> f100 = fib2(100)
    >>> f100
    [0, 1, 1, 2, 3, 5, 8, 13, 21, 34, 55, 89]
\end{minted}
This example, as usual, demonstrates some new Python features:
\begin{itemize}
    \item The \texttt{return} statement returns with a value from a function. \texttt{return}
    without an expression argument returns \texttt{None}. Falling off the end of a function also
    returns \texttt{None}.
    \item The statement \texttt{result.append(a)} calls a method of the list object \texttt{result}.
    A method is a function that 'belongs' to an object is named \texttt{obj.methodname}, where
    \texttt{obj} is some object (this may be an expression), and \texttt{methodname} is the name of
    a method that is defined by the object's type. \\
    The method \texttt{append()} shown in the example is defined for list objects; it adds a new
    element at the end of the list. In this example it is equivalent to
    \texttt{result = result + [a]}, but more efficient.
\end{itemize}




\end{document}
