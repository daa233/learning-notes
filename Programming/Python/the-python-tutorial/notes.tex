%-*- coding: UTF-8 -*-
% notes.tex
%
\documentclass[UTF8]{article}
\usepackage{geometry}
\geometry{a4paper, centering, scale=0.8}
\usepackage{minted}
\usepackage{hyperref}
\usepackage{indentfirst} % to indent the first paragraph of a section

\title{The Python Tutorial \\
        Learning Notes}
\author{Du Ang \\ \texttt{du2ang233@gmail.com} }
\date{\today}

\begin{document}
\maketitle

\tableofcontents
\newpage

\section{Whetting Your Appetite}
\subsection{Some applications of Python}
\begin{itemize}
    \item Automate works on computer
    \item Help software engineers develop and test more quickly
    \item ...
\end{itemize}

\subsection{The features of Python}
\begin{itemize}
    \item High-level language \\
    High-level more general data types
    \item Allow to split program into modules
    \item Interpreted language
    \begin{itemize}
        \item No necessary compilation and linking
        \item The interpreter can be used interactively
    \end{itemize}
    \item Shorter programs than equivalent C, C++ or Java programs
    \begin{itemize}
        \item high-level data types
        \item statement grouping is done by indentation instead of brackets
        \item no variable or argument declarations are necessary
    \end{itemize}
    \item Extensible
\end{itemize}

\subsection{The origin of the name "Python"}
“Monty Python’s Flying Circus”, the skits on BBC.

\section{Using the Python Interpreter}
\subsection{Invoking the Interpreter}
The Python interpreter is usually installed as \texttt{/usr/local/bin/python}, and usually the
\texttt{/usr/local/bin} has been put in the Unix shell’s search path. Type command \texttt{python}
to the shell to invoke the Python interpreter.

Type an end-of-file character(\texttt{Ctrl-D} on Unix, \texttt{Ctrl-Z} on Windows) to exit the
interpreter. If that doesn’t work, type command \mintinline{Python}{quit()} to the shell.

A second way of starting the interpreter is \mintinline{bash}{python -c 'command [arg] ...'},
which executes the statement(s) in command.

Some Python modules are also useful as scripts. These can be invoked using
\mintinline{bash}{python -m module [arg] ...}, which executes the source file for modules as if you
had spelled out its full name on the command line.

\subsubsection{Argument Passing}
When known to the interpreter, the script name and additional arguments thereafter are turned into
a list of strings and assigned to the \texttt{argv} variable in the \texttt{sys} module. You can
access this list by executing \texttt{import sys}. The length of the list is at least one; when no
script and no arguments are given, \texttt{sys.argv[0]} is an empty string. When the script name is
given as \texttt{'-'} (meaning standard input), \texttt{sys.argv[0]} is set to \texttt{'-'}. When
\texttt{-c} command is used, \texttt{sys.argv[0]} is set to \texttt{'-c'}. When \texttt{-m} module
is used, \texttt{sys.argv[0]} is set to the full name of the located module. Options found after
\texttt{-c} command or \texttt{-m} module are not consumed by the Python interpreter’s option
processing but left in \texttt{sys.argv} for the command or module to handle.

\subsubsection{Interactive Mode}
\begin{itemize}
    \item \emph{primary prompt}: \texttt{>>>}
    \begin{minted}{bash}
    Python 2.7.10 (default, Feb  7 2017, 00:08:15)
    Type "help", "copyright", "credits" or "license" for more information.
    >>>
    \end{minted}
    \item \emph{secondary prompt}: \texttt{...}
    \begin{minted}{bash}
    >>> the_world_is_flat = 1
    >>> if the_world_is_flat:
    ...     print "Be careful not to fall off!"
    ...
    Be careful not to fall off!
    \end{minted}
\end{itemize}

\subsection{The Interpreter and Its Environment}
\subsubsection{Source Code Encoding}
By default, Python source files are treated as encoded in UTF-8.

To declare an encoding other than the default one, a special comment line should be added as the
first line of the file. The syntax is as follows:
\begin{minted}{python}
    # -*- coding: encoding -*-
\end{minted}
where \emph{encoding} is one of the valid \texttt{codes} supported by Python.

\section{An Informal Introduction to Python}
In the following examples, input and output are distinguished by the presence or absence of prompts
(\texttt{>>>} and \texttt{...}).

Comments in Python start with the hash character \texttt{\#}, and extend to the end of the physical
line. A hash character in a string literal is just a hash character.

\begin{minted}{python}
    # this is the first comment
    spam = 1    # and this the second comment
                # and now a third!
    text = "# This is not a comment because it's inside quotes."
\end{minted}

\subsection{Using Python as a Calculator}
\subsubsection{Numbers}
\begin{minted}{bash}
    >>> 2 + 2
    4
    >>> 50 - 5 * 6
    20
    >>> (50 - 5.0 * 6) / 4
    5.0
    >>> 8 / 5.0
    1.6
\end{minted}

The integer numbers(e.g. \texttt{2}, \texttt{4}, \texttt{20}) have type \texttt{int}, the ones with
a fractional part(e.g. \texttt{5.0}, \texttt{1.6}) have type \texttt{float}.

The return type of a division (\texttt{/}) operation depends on its operands. If both operands are
of type \texttt{int}, floor division is performed and an \texttt{int} is returned. If either
operand is a float, classic division is performed and a float is returned. The \texttt{//} operator
is also provided for doing floor divsion no matter what the operands are. The remainder can be
calculated with the \texttt{\%} operator:
\begin{minted}{bash}
    >>> 17 / 3    # int / int -> int
    5
    >>> 17 / 3.0  # int / float -> float
    5.666666666666667
    >>> 17 // 3.0 # explicit floor division discards the fractional part
    5.0
    >>> 5 * 3 + 2 # result * divisor + remainder
    17
\end{minted}

With Python, it is possible to use \texttt{**} operator to calculate powers:
\begin{minted}{bash}
    >>> 5 ** 2  # 5 squared
    25
    >>> 2 ** 7  # 2 to the power of 7
    128
\end{minted}

The equal sign (\texttt{=}) is used to assign a value to a variable. Afterwards, no results is
displayed before the next interactive prompt:
\begin{minted}{bash}
    >>> width = 20
    >>> height = 5 * 9
    >>> width * height
    900
\end{minted}

If a variable is not "defined" (assigned a value), trying to use it will give you an error:
\begin{minted}{bash}
    >>> n   # try to access an undefined variable
    Traceback (most recent call last):
    File "<stdin>", line 1, in <module>
    NameError: name 'n' is not defined
\end{minted}

There is full support for floating point; operators with mixed type operands convert the integer
operand to floating point:
\begin{minted}{bash}
    >>> 3 * 3.75 / 1.5
    7.5
    >>> 7.0 / 2
    3.5
\end{minted}

In interactive mode, the last printed expression is assigned to the variable \texttt{\_}.
\begin{minted}{bash}
    >>> tax = 12.5 / 100
    >>> price = 100.50
    >>> price * tax
    12.5625
    >>> price + _
    113.0625
    >>> round(_, 2)
    113.06
\end{minted}

The variable should be treated as read-only by the user. Don't explicitly assign a value to it ---
you would create an independent local variable with the same name masking the built-in variable with
its magic behavior.

In addition to \texttt{int} and \texttt{float}, Python supports other types of numbers, such as
\texttt{Decimal} and \texttt{Fraction}. Python also has built-in support for complex numbers, and
uses the \texttt{j} or \texttt{J} suffix to indicate the imaginary part (e.g. \texttt{3 + 5j}).

\subsubsection{Strings}
Python can also manipulate strigns, which should be enclosed in single quotes (\texttt{'...'}) or
double quotes (\texttt{"..."}) with the same result. \texttt{\textbackslash} can be used to escape
quotes.
\begin{minted}{bash}
    >>> 'spam eggs'  # single quotes
    'spam eggs'
    >>> 'doesn\'t'   # use \' to escape the single quote...
    "doesn't"
    >>> "doesn't"    # ... or use double quotes instead
    "doesn't"
    >>> '"Yes," he said.'
    '"Yes," he said.'
    >>> "\"Yes,\" he said."
    '"Yes," he said.'
    >>> '"Isn\'t," she said.'
    '"Isn\'t," she said.'
\end{minted}

The \texttt{print} statement produces a more readable output, by omitting the enclosing quotes and
by printing escaped and special characters:
\begin{minted}{bash}
    >>> '"Isn\'t," she said.'
    '"Isn\'t," she said.'
    >>> print '"Isn\'t," she said.'
    "Isn't," she said.
    >>> s = 'First line. \nSecond line.'  # \n means newline
    >>> s       # without print, \n is included in the output
    'First line. \nSecond line.'
    >>> print s # with print, \n produces a new line
    First line.
    Second line.
\end{minted}

If you don't want characters prefaced by \texttt{\textbackslash} to be interpreted as special
characters, you can use \emph{raw strings} by adding an \texttt{r} before the first quote:
\begin{minted}{bash}
    >>> print 'C:\some\name'  # here \n means newline!
    C:\some
    ame
    >>> print r'C:\some\name' # note the r before the quote
    C:\some\name
\end{minted}

String literals can span multiple lines. One way is using triple-quotes: \texttt{"""..."""} or
\texttt{'''...'''}. End of lines are automatically included in the string, but it's possible to
prevent this by adding a \texttt{\textbackslash} at the end of the line:
\begin{minted}{bash}
    >>> print """
    ... Usage: thingy [OPTIONS]
    ...     -h              Display this usage message
    ...     -H hostname     Hostname to connect to
    ... """

    Usage: thingy [OPTIONS]
        -h          Display this usage message
        -H hostname Hostname to connect to

    >>> print """\  # prevent the initial newline
    ... Usage: thingy [OPTIONS]
    ...     -h              Display this usage message
    ...     -H hostname     Hostname to connect to
    ... """
    Usage: thingy [OPTIONS]
        -h          Display this usage message
        -H hostname Hostname to connect to

\end{minted}

Strings can be concatenated (glued together) with \texttt{+} operator, and repeated with \texttt{*}:
\begin{minted}{bash}
    >>> # 3 times 'un', followed by 'ium'
    ... 3 * 'un' + 'ium'
    'unununium'
\end{minted}

Two or more string literals next to each other are automatically concatenated:
\begin{minted}{bash}
    >>> 'Py' 'thon'
    'Python'
\end{minted}

This only works with two literals though, not with variables or expressions. If you want to
concatenate variables or a variable and a literal, use \texttt{+}.
\begin{minted}{bash}
    >>> prefix = 'Py'
    >>> prefix 'thon'  # can't concatenate a variable and string literal
      File "<stdin>", line 1
        prefix 'thon'  # can't concatenate a variable and string literal
                    ^
    SyntaxError: invalid syntax

    >>> prefix + 'thon'
    'Python'
\end{minted}

This feature is particularly useful when you want to break long strings:
\begin{minted}{bash}
    >>> text = ('Put several string within parentheses '
    ...         'to have them joined together.')
    >>> text
    'Put several string within parentheses to have them joined together.'
\end{minted}

Strings can be indexed(subscripted), with the first character having index 0. There is no separate
character type; a character is simple a string a size one:
\begin{minted}{bash}
    >>> word  = 'Python'
    >>> word[0]     # character in position 0
    'P'
    >>> word[5]     # character in position 5
    'n'
\end{minted}

Indices may also be negative nubmers, to start counting from the right:
\begin{minted}{bash}
    >>> word[-1]    # last character
    'n'
    >>> word[-2]    # second-last character
    'o'
    >>> word[-6]
    'P'
\end{minted}
Note that since -0 is the same as 0, negative indices start from -1.

In addition to indexing, \emph{slicing} is also supported. While indexing is used to obtain
individual characters, \emph{slicing} allows you to obtain a substring:
\begin{minted}{bash}
    >>> word[0:2]   # characters from position 0 (included) to 2 (excluded)
    'Py'
    >>> word[2:5]   # characters from position 2 (included) to 5 (excluded)
    'tho'
\end{minted}

Since indices have useful defaults; an omitted first index defaults to zero, an ommited second
index defaults to the size of the string being sliced.
\begin{minted}{bash}
    >>> word[:2]    # character from the beginning to position 2 (excluded)
    'Py'
    >>> word[4:]    # character from the position 4 (included) to the end
    'on'
    >>> word[-2:]   # character from the second-last (included) to the end
    'on'
\end{minted}

Note how the start is always included, and the end always excluded. This makes sure that
\texttt{s[:i] + s[i:]} is always equal to \texttt{s}:
\begin{minted}{bash}
    >>> word[:2] + word[2:]
    'Python'
    >>> word[:4] + word[4:]
    'Python'
\end{minted}

One way to remember how slices work is to think of the indices as pointing \emph{between}
characters, with the left edge of the first character numbered 0. Then the right edge of the last
character of a string of n characters has index \emph{n}, for example:
\begin{minted}{text}
    +---+---+---+---+---+---+
    | P | y | t | h | o | n |
    +---+---+---+---+---+---+
    0   1   2   3   4   5   6
    -6  -5  -4  -3  -2  -1
\end{minted}

Attempting to use an index that is too large will result in an error:
\begin{minted}{bash}
    >>> word[42]
    Traceback (most recent call last):
      File "<stdin>", line 1, in <module>
    IndexError: string index out of range
\end{minted}

However, out of range slice indices are handled gracefully when used for slicing:
\begin{minted}{bash}
    >>> word[4:42]
    'on'
    >>> word[42:]
    ''
\end{minted}

Python strings cannot be changed --- they are immutable. Therefore, assigning to a indexed position
in the string results in an error:
\begin{minted}{bash}
>>> word[0] = 'J'
    Traceback (most recent call last):
      File "<stdin>", line 1, in <module>
    TypeError: 'str' object does not support item assignment

    >>> word[2:] = 'py'
    Traceback (most recent call last):
      File "<stdin>", line 1, in <module>
    TypeError: 'str' object does not support item assignment
\end{minted}

If you need a different string, you should create a new one:
\begin{minted}{bash}
    >>> 'J' + word[1:]
    'Jython'
    >>> word[:2] + 'py'
    'Pypy'
\end{minted}

The built-in fucntion \texttt{len()} returns the length of a string:
\begin{minted}{bash}
    >>> s = 'supercalifragilisticexpialidocious'
    >>> len(s)
    34
\end{minted}

\subsubsection{Unicode Strings}
Unicode has the advantage of providing one ordinal for every character in every script used in
modern and ancient texts.

Creating Unicode strings in Python is just as simple as creating normal strings:
\begin{minted}{bash}
    >>> u'Hello World!'
    u'Hello World!'
\end{minted}

The small \texttt{'u'} in front of the quote indicates that a Unicode string is supposed to be
created. If you want to include special characters in the string, you can do so by using the
Python \emph{Unicode-Escape} encoding. The following example shows how:
\begin{minted}{bash}
    >>> u'Hello\u0020World!'
    u'Hello World!'
\end{minted}
The escape sequence \texttt{\textbackslash u0020} indicates to insert the Unicode character with
the ordinal value 0x0020(the space character) at the given position.

\subsubsection{Lists}
Python knows a number of compound data types, used to group together other values. The most
versatile is the \emph{list}, which can be written as a list of comma-separated values (items)
between square brackets. Lists might contain items of different types, but usually the items all
have the same type.

\begin{minted}{bash}
    >>> squares = [1, 4, 9, 16, 25]
    >>> squares
    [1, 4, 9, 16, 25]
\end{minted}

Like strings (and all other built-in sequence type), lists can be indexed and sliced:
\begin{minted}{bash}
    >>> squares[0]      # indexing returns the item
    1
    >>> squares[-1]
    25
    >>> squares[-3:]    # slicing returns a new list
    [9, 16, 25]
\end{minted}

All slice operations return a new list containing the requested elements. This means that the
following slice returns a new (shallow) copy of the list:
\begin{minted}{bash}
    >>> squares[:]
    [1, 4, 9, 16, 25]
\end{minted}

Lists also supports operations like concatenation:
\begin{minted}{bash}
    >>> squares + [36, 49, 64, 81, 100]
    [1, 4, 9, 16, 25, 36, 49, 64, 81, 100]
\end{minted}

Unlike strings, which are immutable, lists are a mutable type, i.e. it is possible to change their
content:
\begin{minted}{bash}
    >>> cubes = [1, 8, 27, 65, 125] # something's wrong here
    >>> 4 ** 3  # the cube of 4 is 64, not 65!
    64
    >>> cubes[3] = 64   # replace the wrong value
    >>> cubes
    [1, 8, 27, 64, 125]
\end{minted}

You can also add new items at the end of the list, by using the \texttt{append()} method:
\begin{minted}{bash}
    >>> cubes.append(216)           # add the cube of 6
    >>> cubes.append(7 ** 3)        # add the cube of 7
    >>> cubes
    [1, 8, 27, 64, 125, 216, 343]
\end{minted}

Assignment to slices is also possible, and this can even change the size of the list or clear it
entirely:
\begin{minted}{bash}
    >>> letters = ['a', 'b', 'c', 'd', 'e', 'f', 'g']
    >>> letters
    ['a', 'b', 'c', 'd', 'e', 'f', 'g']
    >>> # replace some values
    ... letters[2:5] = ['C', 'D', 'E']
    >>> letters
    ['a', 'b', 'C', 'D', 'E', 'f', 'g']
    >>> # now remove them
    ... letters[2:5] = []
    >>> letters
    ['a', 'b', 'f', 'g']
    >>> # clear the list by replacing all the elements with an empty list
    ... letters[:] = []
    >>> letters
    []
\end{minted}

The built-in function \texttt{len()} also applies to lists:
\begin{minted}{bash}
    >>> letters = ['a', 'b', 'c', 'd']
    >>> len(letters)
    4
\end{minted}

It is possible to nest lists (create lists containing other lists), for example:
\begin{minted}{bash}
    >>> a = ['a', 'b', 'c']
    >>> n = [1, 2, 3]
    >>> x = [a, n]
    >>> x
    [['a', 'b', 'c'], [1, 2, 3]]
    >>> x[0]
    ['a', 'b', 'c']
    >>> x[0][1]
    'b'
    >>> x[1]
    [1, 2, 3]
\end{minted}

\subsection{First Steps Towards Programming}
We can write an initial sub-sequence of the \emph{Fibonacci} series as follows:
\begin{minted}{bash}
    >>> # Fibonacci series:
    ... # the sum of two elements defines the next
    ... a, b = 0, 1     # multiple assignment
    >>> while b < 10:   # loop as long as the condition remains true
    ...     print b
    ...     a, b = b, a + b
    ...
    1
    1
    2
    3
    5
    8
\end{minted}

In Python, like in C, any non-zero integer value is true; zero is false. The condition of the
\texttt{while} loop may also be a string or list value, in fact any sequence; anything with a
non-zero length is true, empty sequences are false. The standard comparison operators are written
the same as in C: \texttt{<} (less than), \texttt{>} (greater than), \texttt{==} (equal to),
\texttt{<=} (less than or equal to), \texttt{>=} (greater than or equal to) and \texttt{!=} (not
equal to).

The \emph{body} of loop is \emph{indented}: indentation is Python's way of grouping statements.
At the interactive prompt, you have to type a tab or space(s) for each indented line.

The \texttt{print} statement writes the value of the expression(s) it is given. It differs from
just writing the expression you want to write in the way it handles multiple expressions and
strings. Strings are pointer without quotes, and a space is inserted between items, so you can
format things nicely, like this:
\begin{minted}{bash}
    >>> i = 256 * 256
    >>> print 'The value of i is', i
    The value of i is 65536
\end{minted}

A trailing comma avoids the newline after the output:
\begin{minted}{bash}
    >>> a, b = 0, 1
    >>> while b < 1000:
    ...     print b,
    ...     a, b = b, a + b
    ...
    1 1 2 3 5 8 13 21 34 55 89 144 233 377 610 987
\end{minted}

\section{More Control Flow Tools}
\subsection{\texttt{if} Statements}
\begin{minted}{bash}
    >>> x = int(raw_input("Please enter an integer: "))
    Please enter an integer: 42
    >>> if x < 0:
    ...     x = 0
    ...     print 'Negative changed to zero'
    ... elif x == 0:
    ...     print 'Zero'
    ... elif x == 1:
    ...     print 'Single'
    ... else:
    ...     print 'More'
    ...
    More
\end{minted}

\subsection{\texttt{for} Statements}
The \texttt{for} statement in Python differs a bit from what you may be used to in C or Pascal.
Rather than always iterating over an arithmetic progression of numbers (like in Pascal), or giving
the user the ability to define both the iteration step and halting condition (as C), Python's
\texttt{for} statement iterates over the items of any sequence (a list or a string), in the order
that they appear in the sequence. For example:
\begin{minted}{bash}
    >>> # Measure some strings:
    ... words = ['cat', 'window', 'defenestrate']
    >>> for w in words:
    ...     print w, len(w)
    ...
    cat 3
    window 6
    defenestrate 12
\end{minted}

If you need to modify the sequence you are iterating over while inside the loop, it is recommended
that you first make a copy. Iterating over a sequence does not implicitly make a copy. The slice
notation makes this especially convenient:
\begin{minted}{bash}
    >>> for w in words[:]:
    ...     if len(w) > 6:
    ...         words.insert(0, w)
    ...
    >>> words
    ['defenestrate', 'cat', 'window', 'defenestrate']
\end{minted}

\subsection{The \texttt{range()} function}
If you do need to iterate over a sequence of numbers, the built-in function \texttt{range()} comes
in handy. It generates lists containing arithmetic progressions:
\begin{minted}{bash}
    >>> range(10)
    [0, 1, 2, 3, 4, 5, 6, 7, 8, 9]
    >>> range(5, 10)    # the given end point is never part of the generated list
    [5, 6, 7, 8, 9]
    >>> range(0, 10, 3) # '3' here is the increment('step'), it can be negative
    [0, 3, 6, 9]
    >>> range(-10, -100, -30)
    [-10, -40, -70]
\end{minted}

To iterate over the indices of a sequence, you can combine \texttt{range()} and \texttt{len()} as
follows:
\begin{minted}{bash}
    >>> a = ['Mary', 'had', 'a', 'little', 'lamb']
    >>> for i in range(len(a)):
    ...     print i, a[i]
    ...
    0 Mary
    1 had
    2 a
    3 little
    4 lamb
\end{minted}

\subsection{\texttt{break} and \texttt{continue} Statements, and \texttt{else} Clauses on Loops}
The \texttt{break()} statement, like in C, breaks out of the innermost enclosing \texttt{for} or
\texttt{while} loop.

Loop statements may have an \texttt{else} clause; it is exected when the loop terminates through
exhaustion of the list (with \texttt{for}) or when the condition becomes false (with
\texttt{while}), but not when the loop is terminated by a \texttt{break} statement. This is
exemplified by the following loop, which searches for prime numbers:
\begin{minted}{bash}
    >>> for n in range(2, 10):
    ...     for x in range(2, n):
    ...         if n % x == 0:
    ...             print n, 'equals', x, '*', n / x
    ...             break
    ...     else:
    ...         # loop fell through without finding a factor
    ...         print n, 'is a prime number'
    ...
    2 is a prime number
    3 is a prime number
    4 equals 2 * 2
    5 is a prime number
    6 equals 2 * 3
    7 is a prime number
    8 equals 2 * 4
    9 equals 3 * 3
\end{minted}

When used with a loop, the \texttt{else} clause has more in common with the \texttt{else} clause of
a \texttt{try} statement than it does that of \texttt{if} statements: a \texttt{try} statement's
\texttt{else} clause runs when no exception occurs, and a loop's \texttt{else} clause runs when no
\texttt{break} occurs.

The \texttt{continue} statement, also borrowed from C, continues with the next iteration of the
loop:
\begin{minted}{bash}
    >>> for num in range(2, 10):
    ...     if num % 2 == 0:
    ...         print "Found an even number", num
    ...         continue
    ...     print "Found a number", num
    ...
    Found an even number 2
    Found a number 3
    Found an even number 4
    Found a number 5
    Found an even number 6
    Found a number 7
    Found an even number 8
    Found a number 9
\end{minted}

\subsection{\texttt{pass} Statements}
The \texttt{pass} statement does nothing. It can be used when a statement is required syntactically
but the program requires no action. For example:
\begin{minted}{bash}
    >>> while True:
    ...     pass    # Busy-wait for keyboard interrupt (Ctrl + C)
    ...
    ^CTraceback (most recent call last):
    File "<stdin>", line 1, in <module>
    KeyboardInterrupt
\end{minted}

This is commonly used for creating minimal classes:
\begin{minted}{bash}
    >>> class MyEmpthClass:
    ...     pass
    ...
\end{minted}

Another place \texttt{pass} can be used is as a place-holder for a function or conditional body
when you are working on new code, allowing you to keep thinking at a more abstract level. The
\texttt{pass} is silently ignored:
\begin{minted}{bash}
    >>> def initlog(*args):
    ...     pass    # Remember to implement this!
    ...
\end{minted}

\subsection{Defining Functions}
We can create a function that writes the Fibonacci series to an arbitrary boundary:
\begin{minted}{bash}
    >>> def fib(n):     # write Fibonacci series up to n
    ...     """Print a Fibonacci series up to n."""
    ...     a, b = 0, 1
    ...     while a < n:
    ...         print a,
    ...         a, b = b, a + b
    ...
    >>> # Now call the function we just defined:
    ... fib(2000)
    0 1 1 2 3 5 8 13 21 34 55 89 144 233 377 610 987 1597
\end{minted}

The keyword \texttt{def} introduces a function \emph{definition}. It must be followed by the
function name and the parenthesized list of formal parameters. The statements that form the body of
the function start at the next line, and must be indented.

The first statement of the function body can optionally be a string literal; this string literal is
the function's documentation string, or \emph{docstring}. There are tools which use docstrings to
automatically produce online or printed documentation, or to let the user interactively browse
through code; it's a good practice to include docstrings in code that you write, so make a habit of
it.

The \texttt{execution} of a function introduces a new symbol table used for the local variables of
the function. More precisely, all variable assignments in a function store the value in the local
symbol table; whereas variable references first look in the local symbol table, then in the local
symbol tables of enclosing functions, then in the global symbol table, and finally in the table of
built-in names. Thus, global variables cannot be directly assigned a value within a function
(unless named in a \texttt{global} statement), although they may be referenced.

The actual parameters (arguments) to a function call are introduced in the local symbol table of
the called function when it is called; thus, arguments are passed using \emph{call by value} (where
the \emph{value} is always an object \emph{reference}, not the value of the object). When a
function calls another function, a new local symbol table is created for that call.

A function definition introduces the function name in the current symbol table. The value of the
function name has a type that is recognized by the interpreter as a userdefined function. This
value can be assigned to another name which can be used as a function. This serves as a general
renaming mechanism:
\begin{minted}{bash}
    >>> fib
    <function fib at 0x106cd8140>
    >>> f = fib
    >>> f(100)
    0 1 1 2 3 5 8 13 21 34 55 89
\end{minted}

Coming from other languages, you might object that \texttt{fib} is not a function but a procedure
since it doesn't return a value. In fact, even functions without a \texttt{return} statement do
return a value, albeit a rather boring one. This value is called \texttt{None} (It's a built-in
name). Writing the value \texttt{None} is normally suppressed by the interpreter if it would be the
only value written. You can see if you really want to using \texttt{print}:
\begin{minted}{bash}
    >>> fib(0)
    >>> print fib(0)
    None
\end{minted}

It is simple to write a function that returns a list of the numbers of the Fibonacci series,
instead of printing it:
\begin{minted}{bash}
    >>> def fib2(n):
    ...     """Return a list containing the Fibonacci series up to n."""
    ...     result = []
    ...     a, b = 0, 1
    ...     while a < n:
    ...         result.append(a)
    ...         a, b = b, a + b
    ...     return result
    ...
    >>> f100 = fib2(100)
    >>> f100
    [0, 1, 1, 2, 3, 5, 8, 13, 21, 34, 55, 89]
\end{minted}
This example, as usual, demonstrates some new Python features:
\begin{itemize}
    \item The \texttt{return} statement returns with a value from a function. \texttt{return}
    without an expression argument returns \texttt{None}. Falling off the end of a function also
    returns \texttt{None}.
    \item The statement \texttt{result.append(a)} calls a method of the list object \texttt{result}.
    A method is a function that 'belongs' to an object is named \texttt{obj.methodname}, where
    \texttt{obj} is some object (this may be an expression), and \texttt{methodname} is the name of
    a method that is defined by the object's type. \\
    The method \texttt{append()} shown in the example is defined for list objects; it adds a new
    element at the end of the list. In this example it is equivalent to
    \texttt{result = result + [a]}, but more efficient.
\end{itemize}

\subsection{More on Defining Functions}
It is also possible to define functions with a variable number of arguments. There are three forms,
which can be comined.
\subsubsection{Default Arguments Values}
The most useful form is to specify a default value for one or more arguments. The default arguments
can't be followed by non-default arguments. This creates a function that can be called with fewer
arguments than it is defined to allow. For example:
\begin{minted}{bash}
    >>> def ask_ok(prompt, retries = 4, complaint = "Yes or no, please!"):
    ...     while True:
    ...         ok = raw_input(prompt)
    ...         if ok in ('y', 'ye', 'yes'):
    ...             return True
    ...         if ok in ('n', 'no', 'nop', 'nope'):
    ...             return False
    ...         retries = retries - 1
    ...         if retries < 0:
    ...             raise IOError('refusenik user')
    ...         print complaint
    ...
\end{minted}
This function can be called in several ways:
\begin{itemize}
    \item giving only the madatory argument: \\
    \texttt{ask\_ok('Do you really want to quit?')}
    \item giving one of the optional arguments: \\
    \texttt{ask\_ok('OK to overwrite the file?', 2)}
    \item or even giving all arguments: \\
    \texttt{ask\_ok('OK to overwrite the file?', 2, 'Come on, only yes or no!')}
\end{itemize}

The keyword \texttt{in} can test whether or not a sequence contains a certain value.

The default values are evaluated at the point of function definition in the \emph{defining} scope.
In other words, default parameter values are always evaluated when, and only when, the \texttt{def}
statement they belong to is executed, so that
\begin{minted}{bash}
    >>> i = 5
    >>> def f(arg = i):
    ...     print arg
    ...
    >>> f()
    5
    >>> i = 6
    >>> f()
    5
\end{minted}

\textbf{Important warnings: The default value is evaluated only once.}
\footnote{http://effbot.org/zone/default-values.htm} This makes a difference when
the default is a mutable object
\footnote{https://codehabitude.com/2013/12/24/python-objects-mutable-vs-immutable/}
such as a list, dictionary, or instances of most classes. For example, the following function
accumulates the arguments passed to it on subsequent calls:
\begin{minted}{bash}
>>> def f(a, L = []):
    ...     L.append(a)
    ...     return L
    ...
    >>> print f(1)
    [1]
    >>> print f(2)
    [1, 2]
    >>> print f(3)
    [1, 2, 3]
\end{minted}

If you don't want the default to be shared between subsequent calls, you can write the function
like this instead:
\begin{minted}{bash}
    >>> def f(a, L = None):
    ...     if L is None:
    ...         L = []
    ...     L.append(a)
    ...     return L
    ...
    >>> print f(1)
    [1]
    >>> print f(2)
    [2]
    >>> print f(3)
    [3]
\end{minted}

\subsubsection{Keyword Arguments}
Functions can also be called using keyword arguments of the form \texttt{kwarg = value}.

\begin{minted}{bash}
    >>> def parrot(voltage, state = 'a stiff', action = 'voom', type = 'Norwegian Blue'):
    ...     print "-- This parrot wouldn't", action,
    ...     print "if you put", voltage, "volts through it."
    ...     print "-- Lovely plumage, the", type
    ...     print "-- It's", state, "!"
    ...
\end{minted}

This function can be called in any of the following ways:
\begin{minted}{bash}
    parrot(1000)                                          # 1 positional argument
    parrot(voltage=1000)                                  # 1 keyword argument
    parrot(voltage=1000000, action='VOOOOOM')             # 2 keyword arguments
    parrot(action='VOOOOOM', voltage=1000000)             # 2 keyword arguments
    parrot('a million', 'bereft of life', 'jump')         # 3 positional arguments
    parrot('a thousand', state='pushing up the daisies')  # 1 positional, 1 keyword
\end{minted}
but all the following calls would be invalid:
\begin{minted}{bash}
    parrot()                     # required argument missing
    parrot(voltage=5.0, 'dead')  # non-keyword argument after a keyword argument
    parrot(110, voltage=220)     # duplicate value for the same argument
    parrot(actor='John Cleese')  # unknown keyword argument
\end{minted}

When a final formal parameter of the form \texttt{**name} is present, it receives a dictionary
containing all keyword arguments except for those corresponding to a formal parameter. This may be
contained with a formal parameter of the form \texttt{*name} which receives a tuple containing the
positional arguments beyond the formal parameter list. (\texttt{*name} must occur before
\texttt{**name}.) For example, if we define a function like this:
\begin{minted}{bash}
    >>> def cheeseshop(kind, *arguments, **keywords):
    ...     print "-- Do you have any", kind + "?"
    ...     print "-- I'm sorry, we're all out of", kind
    ...     for arg in arguments:
    ...             print arg
    ...     print "-" * 40
    ...     keys = sorted(keywords.keys())
    ...     for kw in keys:
    ...             print kw, ":", keywords[kw]
    ...
    >>> cheeseshop("Limburger", "It's very runny, sir.",
    ...             "It's really very, VERY runny, sir.",
    ...             shopkeeper = 'Michael Palin',
    ...             client = "John Cleese",
    ...             sketch = "Cheese Shop Sketch")
    -- Do you have any Limburger?
    -- I'm sorry, we're all out of Limburger
    It's very runny, sir.
    It's really very, VERY runny, sir.
    ----------------------------------------
    client : John Cleese
    shopkeeper : Michael Palin
    sketch : Cheese Shop Sketch
\end{minted}

Note that the list of keyword argument names is created by sorting the result of the keywords
dictionary's \texttt{keys()} method before printing its contents; if this is not done, the order in
which the arguments are printed is undefined.

\subsubsection{Unpacking Arugument Lists}
The reverse situation occurs when the arguments are already in a list or tuple but need to be
unpacked for a function call requiring separate positional arguments. For instances, the built-in
\texttt{range()} function expects separate \emph{start} and \emph{stop} arguments. If they are not
available separately, write the function call with the \texttt{*}-operator to unpack the arguments
out of a list or tuple:
\begin{minted}{bash}
    >>> range(3, 6)
    [3, 4, 5]
    >>> args = [3, 6]
    >>> range(*args)
    [3, 4, 5]
\end{minted}

In the same fashion, dictionaries can deliver keyword arguments with the \texttt{**}-operator:
\begin{minted}{bash}
    >>> def parrot(voltage, state = 'a stiff', action = 'voom'):
    ...     print "-- This parrot wouldn't", action,
    ...     print "if you put", voltage, "volts through it.",
    ...     print "E's", state + "!"
    ...
    >>> d = {"voltage" : "four million", "state" : "blledin' demised", "action" : "VOOM"}
    >>> parrot(**d)
    -- This parrot wouldn't VOOM if you put four million volts through it. E's blledin' demised!
\end{minted}

\subsubsection{Lambda Expressions}
Small anonymous functions can be created with the \texttt{lambda} keyword. This function returns
the sum of its two arguments: \mintinline{Python}{lambda a, b : a + b}. Lambda functions can be
used wherever function objects are required. They are syntactically restricted to a single
expression. Semantically, they are just syntactic sugar for a normal function definition. Like
nested function definitions, lambda functions can reference variables from the containing scope:
\begin{minted}{bash}
    >>> def make_incrementor(n):
    ...     return lambda x : x + n     # return a function
    ...
    >>> f = make_incrementor(42)
    >>> f(0)
    42
    >>> f(1)
    43
\end{minted}

The above example uses a lambda expression to return a function. Another use is to pass a small
function as an argument:
\begin{minted}{bash}
    >>> pairs = [(1, 'one'), (2, 'two'), (3, 'three'), (4, 'four')]
    >>> pairs.sort(key = lambda pair : pair[1])
    >>> pairs
    [(4, 'four'), (1, 'one'), (3, 'three'), (2, 'two')]
\end{minted}

\subsubsection{Documentation Strings}
There are emerging conventions about the content and formatting of documentation strings.

The first line should always be a short, concise summary of the object's purpose. For brevity, it
should not explicitly state the object's name or type, since these are available by other means
(except if the name happens to be a verb describing a function's operation). This line should begin
with a capital letter and end with a period.

If there are more lines in the documentation string, the seocnd line should be blank, visually
separating the summary from the rest of the description. The following lines should be one or more
paragraphs describing the object's calling conventions, its side effects, etc.

The Python parser does not strip indentation from multi-line string literals in Python, so tools
that process documentation have to strip indentation if desired. This is done using the following
convention. The first non-blank line \emph{after} the first line of the string determines the
amount of indentation for the entire documentation string. Whitespace ``equivalent'' to this
indentation is then stripped from the start of all lines of the string.

Here is an example of a multi-line docstring:
\begin{minted}{bash}
    >>> def my_function():
    ...     """Do nothing, but document it.
    ...
    ...     No, really, it doesn't do anything.
    ...     """
    ...     pass
    ...
    >>> print my_function.__doc__
    Do nothing, but document it.

            No, really, it doesn't do anything.

\end{minted}

\subsection{Intermezzo: Coding Style}
Making it easy for others to read your code is always a good a idea, and adopting a nice coding
style helps tremendously for that.

For Python, PEP 8 \footnote{https://www.python.org/dev/peps/pep-0008} has emerged as the style
guide that most projects adhere to; it promotes a very readable and eye-pleasing coding style.
Every Python developer should read it at some point; here are the most important points extracted:
\begin{itemize}
    \item Use 4-space indentation, and no tabs.
    \item Wrap lines so that they don't exceed 79 characters.
    \item Use blank lines to separate functions and classes, and larger blocks of code inside
    functions.
    \item When possible, put comments on a line of their own.
    \item Use docstrings.
    \item Use spaces around operators and after commas, but not directly inside bracketing
    constructs: \texttt{a = f(1, 2) + g(3, 4)}.
    \item Name your classes and functions consistently; the convention is to use \texttt{CamelCase}
    for classes and \texttt{lower\_case\_with\_underscores} for functions and methods. Always use
    \texttt{self} as the name for the first method argument.
    \item Don't use fancy encodings if your code is meant to be used in international environments.
    Plain ASCII works best in any case.
\end{itemize}

\section{Data Structures}
\subsection{More on Lists}
All of the methods of list objects:
\begin{itemize}
    \item \mintinline{Python}{list.append(x)} \\
    Add an item to the end of the list; equivalent to \texttt{a[len(a):] = x}.
    \item \mintinline{Python}{list.extend(L)} \\
    Extend the list by appending all the items in the given list; equivalent to
    \texttt{a[len(a):] = L}
    \item \mintinline{Python}{list.insert(i, x)} \\
    Insert an item at a given position. The first argument is the index of the element before which
    to insert, so \texttt{a.insert(0, x)} inserts at the front of the list, and
    \texttt{a.insert(len(a), x)} is equivalent to \texttt{a.append(x)}.
    \item \mintinline{Python}{list.remove(x)} \\
    Remove the first item from the list whose value is \texttt{x}.
    \item \mintinline{Python}{list.pop([i])} \\
    Remove the item at the given position in the list, and return it. If no index is specified,
    \texttt{a.pop()} removes and returns the last item in the list. (The square brackets around the
    \texttt{i} in the method signature denote that the parameter is optional, not that you should
    type square brackets at that position. You will see this notation frequently in the Python
    Library Reference.)
    \item \mintinline{Python}{list.index(x)} \\
    Return the index in the list of the first item whose value is \texttt{x}. It is an error if
    there is no such item.
    \item \mintinline{Python}{list.count(x)} \\
    Return the number of times \texttt{x} appears in the list.
    \item \mintinline{Python}{list.sort(cmp = None, key = None, reverse = False)} \\
    Sort the items of the list in place (the arguments can be used for sort customization, see
    \texttt{sorted} \footnote{https://docs.python.org/2.7/library/functions.html\#sorted} for their
    explanation).
    \item \mintinline{Python}{list.reverse()} \\
    Reverse the elements of the list, in place.
\end{itemize}

An example that uses most of the list methods:
\begin{minted}{bash}
    >>> a = [66.25, 333, 333, 1, 1234.5]
    >>> print a.count(333), a.count(66.25), a.count('x')
    2 1 0
    >>> a.insert(2, -1)
    >>> a.append(333)
    >>> a
    [66.25, 333, -1, 333, 1, 1234.5, 333]
    >>> a.index(333)
    1
    >>> a.remove(333)
    >>> a
    [66.25, -1, 333, 1, 1234.5, 333]
    >>> a.reverse()
    >>> a
    [333, 1234.5, 1, 333, -1, 66.25]
    >>> a.sort()
    >>> a
    [-1, 1, 66.25, 333, 333, 1234.5]
    >>> a.pop()
    1234.5
    >>> a
    [-1, 1, 66.25, 333, 333]
\end{minted}

You might have noticed that methods like \texttt{insert}, \texttt{remove} or \texttt{sort} that
only modify the list have no return value printed --- they return the default \texttt{None}. This
is a design principle for all mutable data structures in Python.

\subsubsection{Using Lists as Stacks}
The list methods make it very easy to use a list as a stack, where the last element added is the
first element retrieved (``last-in, first-out''). To add an item to the top of the stack, use
\texttt{append()}. To retrieve an item from the top of the stack, use \texttt{pop()} without an
explicit index. For example:
\begin{minted}{bash}
    >>> stack = [3, 4, 5]
    >>> stack.append(6)
    >>> stack.append(7)
    >>> stack
    [3, 4, 5, 6, 7]
    >>> stack.pop()
    7
    >>> stack
    [3, 4, 5, 6]
    >>> stack.pop()
    6
    >>> stack.pop()
    5
    >>> stack
    [3, 4]
\end{minted}

\subsubsection{Using Lists as Queues}
It is also possible to use a list as a queue, where the first element added is the first element
retrieved (``first-in, first-out''); however, lists are not efficient for this purpose. While
appends and pops from the end of list are fast, doing inserts or pops from the beginning of a list
is slow (because all the other elements have to be shifted by one).

To implement a queue, use \texttt{collections.deque} which was designed to have fast appends and
pops from both ends. For example:
\begin{minted}{bash}
>>> from collections import deque
    >>> queue = deque(["Eric", "John", "Michael"])
    >>> queue.append("Terry")       # Terry arrives
    >>> queue.append("Graham")      # Graham arrives
    >>> queue.popleft()             # The first to arrive now leaves
    'Eric'
    >>> queue.popleft()             # The second to arrive now leaves
    'John'
    >>> queue                       # Remaining queue in order of arrival
    deque(['Michael', 'Terry', 'Graham'])
\end{minted}

\subsubsection{Functional Programming Tools}
There are three built-in functions that are very useful when used with lists: \texttt{filter()},
\texttt{map()}, and \texttt{reduce()}.

\texttt{filter(function, sequence)} returns a sequence consisting of those item from the sequence
for which \texttt{function(item)} is true. If \texttt{sequence} is a \texttt{str}, \texttt{unicode}
or \texttt{tuple}, the result will be of the same type; otherwise, it is always a \texttt{list}.
For example, to compute a sequence of numbers divisible by 3 or 5:
\begin{minted}{bash}
    >>> def f(x) : return x % 3 == 0 or x % 5 == 0
    ...
    >>> filter(f, range(2, 25))
    [3, 5, 6, 9, 10, 12, 15, 18, 20, 21, 24]
\end{minted}

\texttt{map(function, sequence)} calls function(item) for each of the sequence's items and returns
a list of teh return values. For example, to compute some cubes:
\begin{minted}{bash}
    >>> def cube(x) : return x * x * x
    ...
    >>> map(cube, range(1, 11))
    [1, 8, 27, 64, 125, 216, 343, 512, 729, 1000]
\end{minted}

More than one sequence may be passed; the function must then have as many arguments as there are
sequences and is called with the corresponding item from each sequence (or \texttt{None} if some
sequence is shorter than another). For example:
\begin{minted}{bash}
    >>> seq = range(8)
    >>> def add(x, y) : return x + y
    ...
    >>> map(add, seq, seq)
    [0, 2, 4, 6, 8, 10, 12, 14]
\end{minted}

\texttt{reduce(function, sequence)} returns a single value constructed by calling the binary
function \texttt{function} on the first two items of teh sequence, then on the result and the next
items, and so on. For examples, to compute the sum of the numbers 1 through 10:
\begin{minted}{bash}
    >>> def add(x, y) : return x + y
    ...
    >>> reduce(add, range(1, 11))
    55
    >>> reduce(add, range(1,2))
    1
    >>> reduce(add, range(1,1))
    Traceback (most recent call last):
      File "<stdin>", line 1, in <module>
    TypeError: reduce() of empty sequence with no initial value
\end{minted}

A third argument can be passed to indicate the starting value. In this case the starting value is
returned for an empty sequence, and the function is first applied to the starting value and the
first sequence item, then to the result and the next item, and so on. For example,
\begin{minted}{bash}
    >>> def sum(seq):
    ...     def add(x, y) : return x + y
    ...     return reduce(add, seq, 0)
    ...
    >>> sum(range(1, 11))
    55
    >>> sum([])
    0
\end{minted}

Don't use this example's definition of \texttt{sum()}: since summing nubmers is such a common need,
a built-in function \texttt{sum(sequence)} is already provided, and works exactly like this.

\subsubsection{List Comprehensions}
List comprehensions provide a concise way to create lists. Common applications are to make new
lists where each element is teh result of some operations applied to each member of another
sequence or iterable, or to create a subsequence of those elements that satisfy a certain condition.

For example, assume we want to create a list of squares, like:
\begin{minted}{bash}
>>> squares = []
>>> for x in range(10):
...     squares.append(x**2)
...
>>> squares
[0, 1, 4, 9, 16, 25, 36, 49, 64, 81]
\end{minted}

We can obtain the same result with:
\begin{minted}{bash}
    >>> squares = [x**2 for x in range(10)]
    >>> squares
    [0, 1, 4, 9, 16, 25, 36, 49, 64, 81]
\end{minted}
This is also equivalent to \mintinline{Python}{squares = map(lambda x : x ** 2, range(10))}, but
it's more concise and readable.

A list comprehension consists of brackets containing an expression followed by a \texttt{for}
clause, then zero or more \texttt{for} or \texttt{if} clauses. The result will be a new list
resulting from evaluating the expression in the context of the \texttt{for} and \texttt{if} clauses
which follow it. For example, this listcomp combines the elements of two lists if they are not
equal:
\begin{minted}{bash}
    >>> [(x, y) for x in [1, 2, 3] for y in [3, 1, 4] if x != y]
    [(1, 3), (1, 4), (2, 3), (2, 1), (2, 4), (3, 1), (3, 4)]
\end{minted}
and it's equivalent to:
\begin{minted}{bash}
    >>> combs = []
    >>> for x in [1, 2, 3]:
    ...     for y in [3, 1, 4]:
    ...         if x != y:
    ...             combs.append((x, y))
    ...
    >>> combs
    [(1, 3), (1, 4), (2, 3), (2, 1), (2, 4), (3, 1), (3, 4)]
\end{minted}

If the expression is a tuple (e.g. the \texttt{(x, y)} in the previous example), it must be
parenthesized.
\begin{minted}{bash}
    >>> vec = [-4, -2, 0, 2, 4]
    >>> # create a new list with the values doubled
    ... [x*2 for x in vec]
    [-8, -4, 0, 4, 8]
    >>> # filter the list to exclude negative numbers
    ... [x for x in vec if x >= 0]
    [0, 2, 4]
    >>> # apply a function to all the elements
    ... [abs(x) for x in vec]
    [4, 2, 0, 2, 4]
    >>> # call a method on each element
    ... freshfruit = ['  banana', '  loganberry ', 'passion fruit   ']
    >>> [weapon.strip() for weapon in freshfruit]
    ['banana', 'loganberry', 'passion fruit']
    >>> # create a list of 2-tuples like (nubmer, square)
    ... [(x, x**2) for x in range(6)]
    [(0, 0), (1, 1), (2, 4), (3, 9), (4, 16), (5, 25)]
    >>> # the tuple must be parenthesized, otherwise an error is raised
    ... [x, x**2 for x in range(6)]
      File "<stdin>", line 2
        [x, x**2 for x in range(6)]
                   ^
    SyntaxError: invalid syntax
    >>> # flatten a list using a listcomp with two 'for'
    ... vec = [[1, 2, 3], [4, 5, 6], [7, 8, 9]]
    >>> [num for elem in vec for num in elem]
    [1, 2, 3, 4, 5, 6, 7, 8, 9]
\end{minted}

List comprehension can contain complex expressions and nested functions:
\begin{minted}{bash}
    >>> from math import pi
    >>> [str(round(pi, i)) for i in range(1, 6)]
    ['3.1', '3.14', '3.142', '3.1416', '3.14159']
\end{minted}

The initial expression in a list comprehension can be any arbitrary expression, including another
list comprehension.

Consider the following example of a 3$\times$4 matrix implemented as a list of 3 lists of length 4:
\begin{minted}{bash}
    >>> matrix = [
    ...     [1, 2, 3, 4],
    ...     [5, 6, 7, 8],
    ...     [9, 10, 11, 12]
    ... ]
\end{minted}

The following list comprehension will transpose rows and columns:
\begin{minted}{bash}
    >>> [[row[i] for row in matrix] for i in range(4)]
    [[1, 5, 9], [2, 6, 10], [3, 7, 11], [4, 8, 12]]
\end{minted}

As we saw in the previous section, the nested listcomp is evaluated in the context of the
\texttt{for} that follows it, so this example is equivalent to:
\begin{minted}{bash}
    >>> transposed = []
    >>> for i in range(4):
    ...     transposed.append([row[i] for row in matrix])
    ...
    >>> transposed
    [[1, 5, 9], [2, 6, 10], [3, 7, 11], [4, 8, 12]]
\end{minted}

which, in turn, is the same as:
\begin{minted}{bash}
    >>> transposed = []
    >>> for i in range(4):
    ...     # the following 3 lines implement the nested listcomp
    ...     transposed_row = []
    ...     for row in matrix:
    ...         transposed_row.append(row[i])
    ...     transposed.append(transposed_row)
    ...
    >>> transposed
    [[1, 5, 9], [2, 6, 10], [3, 7, 11], [4, 8, 12]]
\end{minted}

In the real world, you should prefer built-in functions to complex flow statements. The
\texttt{zip()} function would do a great job for this use case:
\begin{minted}{bash}
    >>> zip(*matrix)  # use * to unpack the arguments from matrix
    [(1, 5, 9), (2, 6, 10), (3, 7, 11), (4, 8, 12)]
\end{minted}

\subsection{The \texttt{del} Statement}
There is a way to remove an item from a list given its index instead of its value: the \texttt{del}
statement. This differs from the \texttt{pop()} method which returns a value. The \texttt{del}
statement can also be used to remove slices from a list or clear the entire list (which we did
earlier by assignment of an empty list to the slice). For example:
\begin{minted}{bash}
    >>> a = [-1, 1, 66.25, 333, 333, 1234.5]
    >>> del a[0]
    >>> a
    [1, 66.25, 333, 333, 1234.5]
    >>> del a[2:4]
    >>> a
    [1, 66.25, 1234.5]
    >>> del a[:]
    >>> a
    []
\end{minted}

\texttt{del} can also be used to delete entire variables. Referencing the name \texttt{a} hereafter
is an error (at least until another value is assigned to it):
\begin{minted}{bash}
    >>> del a
    >>> a
    Traceback (most recent call last):
      File "<stdin>", line 1, in <module>
    NameError: name 'a' is not defined
\end{minted}

\subsection{Tuples and Sequences}
A tuple consists of a number of values separated by commas, for instance:
\begin{minted}{bash}
    >>> t = 12345, 54321, 'hello!'
    >>> t[0]
    12345
    >>> t
    (12345, 54321, 'hello!')
    >>> # Tuples may be nested
    ... u = t, (1, 2, 3, 4, 5)
    >>> u
    ((12345, 54321, 'hello!'), (1, 2, 3, 4, 5))
    >>> # Tuples are immutable
    ... t[0] = 88888
    Traceback (most recent call last):
      File "<stdin>", line 2, in <module>
    TypeError: 'tuple' object does not support item assignment
    >>> # but they can contain mutable objects
    ... v = ([1, 2, 3], [3, 2, 1])
    >>> v
    ([1, 2, 3], [3, 2, 1])
\end{minted}

As you see, on output tuples are always enclosed in parentheses, so that nested tuples are
interpreted correctly; they may be input with or without surrounding parentheses, although often
parentheses are necessary anyway (if the tuple is part of a larger expression). It is not possible
to assign to the individual items of a tuple, however it is possible to create tuples which contain
mutable objects, such as lists.

Though tuples may seem similar to lists, they are often used in different situations and for
different purposes. Tuples are immutable, and usually contain a heterogeneous sequence of elements
that are accessed via unpacking or indexing (or even by attribute in the case of
\texttt{namedtuples}). Lists are mutable, and their elements are usually homogeneous and are
accessed by iterating over the list.

A special problem is the construction of tuples containing 0 or 1 items: the syntax has some extra
quirks to accommodate these. Empty tuples are constructed by an empty pair of parentheses; a tuple
with one item is constructed by following a value with a comma (it is not sufficient to enclose a
single value in parentheses). Ugly, but effective. For example:
\begin{minted}{bash}
    >>> empty = ()
    >>> singleton = 'hello',    # <-- note trailing comma
    >>> len(singleton)
    1
    >>> singleton
    ('hello',)
    >>> singleton = 'hello'     # the type of singleton here is 'str'
    >>> len(singleton)
    5
    >>> singleton
    'hello'
\end{minted}

The statement \texttt{t = 12345, 54321, 'hello!'} is an example of \emph{tuple packing}: the values
\texttt{12345}, \texttt{54321} and \texttt{'hello!'} are packed together in a tuple. The reverse
operation is also possible:
\begin{minted}{bash}
    >>> x, y, z = t
    >>> x
    12345
    >>> y
    54321
    >>> z
    'hello!'
\end{minted}

This is called, appropriately enough, \emph{sequence unpacking} and works for any sequence on the
right-hand side. Sequence unpacking requires the list of variables on the left to have the same
number of elements as the length of the sequence. Note that multiple assignment is really just a
combination of tuple packing and sequence unpacking.

\subsection{Sets}
Python also includes a data type for \emph{sets}. A set is an unsorted collection with no duplicate
elements. Basic uses include membership testing and eliminating duplicate entries. Set objects also
support mathematical operations like union, intersection, difference, and symmetric difference.

Curly braces or the \texttt{set()} function can be used to create sets. Note: to create an empty
set you have to use \texttt{set()}, not \texttt{\{\}}; the latter creates an empty dictionary.

Here is a brief demonstration:
    \begin{minted}{bash}
    >>> basket = ['apple', 'orange', 'apple', 'pear', 'orange', 'banana']
    >>> fruit = set(basket)     # create a set without duplicates
    >>> fruit
    set(['orange', 'pear', 'apple', 'banana'])
    >>> 'orange' in fruit       # fast membership testing
    True
    >>> 'crabgrass' in fruit
    False

    >>> # Demonstrate set operations on unique letters from two words
    ...
    >>> a = set('abracadabra')
    >>> b = set('alacazam')
    >>> a
    set(['a', 'r', 'b', 'c', 'd'])
    >>> b
    set(['a', 'c', 'z', 'm', 'l'])
    >>> a - b
    set(['r', 'b', 'd'])
    >>> a | b
    set(['a', 'c', 'b', 'd', 'm', 'l', 'r', 'z'])
    >>> a & b
    set(['a', 'c'])
    >>> a ^ b
    set(['b', 'd', 'm', 'l', 'r', 'z'])
\end{minted}

Similarly to list comprehensions, set comprehensions are also supported:
\begin{minted}{bash}
    >>> a = {x for x in 'abracadabra' if x not in 'abc'}
    >>> a
    set(['r', 'd'])
\end{minted}

\subsection{Dictionaries}
Another useful data type built into Python is the \emph{dictionary}. Dictionaries are sometimes
found in other languages as "associative memories" or "associative arrays". Unlike sequences, which
are indexed by a range of numbers, dictionaries are indexed by \emph{keys}, which can be any
immutable types; strings and numbers can always be keys. Tuples can be used as keys if they contain
only strings, numbers, or tuples; if a tuple contain any mutable object either directly or
indirectly, it cannot be used as a key. You can't use lists as keys, since lists can modified in
place using index assignments, slice assignments, or methods like \texttt{append()} and
\texttt{extend()}.

It is best to think of a dictionary an unsorted set of \emph{key: value} pairs, with the
requirement that the keys are unique (within one dictionary). A pair of braces creates an empty
dictionary: \texttt{\{\}}. Placing a comma-separated list of key:value pairs within the braces adds
initial key:value pairs to the dictionary; this is also the way dictionaries are written on output.

The main operations on a dictionary are storing a value with some key and extracting the value
given the key. It is also possible to delete a key:value pair with \texttt{del}. If you store using
a key that is already in use, the old value associated with that key is forgotten. It is an error
to extract a value using a non-existent key.

The \texttt{keys()} method of a dictionary object returns a list of all the keys used in the
dictionary, in arbitrary order (if you want it sorted, just apply the \texttt{sorted()} function to
it). To check whether a single key is in the dictionary, use \texttt{in} keyword.

Here is a small example using a dictionary:
\begin{minted}{bash}
    >>> tel = {'jack': 4098, 'sape': 4139}
    >>> tel['guido'] = 4127
    >>> tel
    {'sape': 4139, 'jack': 4098, 'guido': 4127}
    >>> tel['jack']
    4098
    >>> del tel['sape']
    >>> tel['irv'] = 4127
    >>> tel
    {'jack': 4098, 'irv': 4127, 'guido': 4127}
    >>> tel.keys()
    ['jack', 'irv', 'guido']
    >>> 'guido' in tel
    True
    >>> 'tom' in tel
    False
\end{minted}

The \texttt{dict()} constructor builds dictionaries from sequences of key-value pairs:
\begin{minted}{bash}
    >>> dict([('sape', 4139), ('guido', 4127), ('jack', 4098)])
    {'sape': 4139, 'jack': 4098, 'guido': 4127}
\end{minted}

In addition, dict comprehensions can be used to create dictionaries from arbitrary key and value
expressions:
\begin{minted}{bash}
    >>> {x: x**2 for x in (2, 4, 6)}
    {2: 4, 4: 16, 6: 36}
\end{minted}

When the keys are simple strings, it is sometimes easier to specify pairs using keyword arguments:
\begin{minted}{bash}
    >>> dict(sape=4139, guido=4127, jack=4098)
    {'sape': 4139, 'jack': 4098, 'guido': 4127}
\end{minted}

\subsection{Looping Techniques}
When looping through a sequence, the position index and corresponding value can be retrieved at the
same time using the \texttt{enumerate()} function.
\begin{minted}{bash}
    >>> for i, v in enumerate(['tic', 'tac', 'toe']):
    ...     print i, v
    ...
    0 tic
    1 tac
    2 toe
\end{minted}

To loop over two or more sequences at the same time, the entries can be paired with the
\texttt{zip()} function.
\begin{minted}{bash}
    >>> questions = ['name', 'quest', 'favorite color']
    >>> answers = ['lancelot', 'the holy grail', 'blue']
    >>> for q, a in zip(questions, answers):
    ...     print 'What is your {0}? It is {1}.'.format(q, a)
    ...
    What is your name? It is lancelot.
    What is your quest? It is the holy grail.
    What is your favorite color? It is blue.
\end{minted}

To loop over a sequence in reverse, first specify the sequence in a forward direction and then
call the \texttt{reversed()} fucntion.
\begin{minted}{bash}
    >>> for i in reversed(xrange(1, 10, 2)):
    ...     print i
    ...
    9
    7
    5
    3
    1
\end{minted}

To loop over a sequence in sorted order, use the \texttt{sorted()} function which returns a new
sorted list while leaving the source unaltered.
\begin{minted}{bash}
    >>> basket = ['apple', 'orange', 'apple', 'pear', 'orange', 'banana']
    >>> for f in sorted(set(basket)):
    ...     print f
    ...
    apple
    banana
    orange
    pear
    >>> basket
    ['apple', 'orange', 'apple', 'pear', 'orange', 'banana']
\end{minted}

When looping through dictionaries, the key and corresponding value can be retrieved at the same
time using the \texttt{iteritems()} method.
\begin{minted}{bash}
    >>> knights = {'gallahad': 'the pure', 'robin': 'the brave'}
    >>> for k, v in knights.iteritems():
    ...     print k, v
    ...
    gallahad the pure
    robin the brave
\end{minted}

It is sometimes tempting to change a list while you are looping over it; however, it is often
simpler and safer to create a new list instead.
\begin{minted}{bash}
    >>> import math
    >>> raw_data = [56.2, float('NaN'), 51,7, 55.3, 52.5, float('NaN'), 47.8]
    >>> filtered_data = []
    >>> for value in raw_data:
    ...     if not math.isnan(value):
    ...         filtered_data.append(value)
    ...
    >>> filtered_data
    [56.2, 51, 7, 55.3, 52.5, 47.8]
\end{minted}

\subsection{More on Conditions}
The conditions used in \texttt{while} and \texttt{if} statements can contain any operators, not
just comparisons.

The comparison operators \texttt{in} and \texttt{not in} check whether a value occurs (does not
occur) in a sequence. The operators \texttt{is} and \texttt{is not} compare whether two objects are
really the same object; this only matters for mutable objects like lists. All comparison operators
have the same priority, which is lower than that of all numerical operators.

Comparisons can be chained. For example, \texttt{a < b == c} tests whether \texttt{a} is less than
\texttt{b} and moreover \texttt{b} equals \texttt{c}.

Comparisons may be combined using the Boolean operators \texttt{and} and \texttt{or}, and the
outcome of a comparison (or of any other Boolean expression) may be negated with \texttt{not}.
These have lower priorities than comparison operators; between them, \texttt{not} has the highest
priority and \texttt{or} the lowest, so that \texttt{A and B not B or C} is equivalent to
\texttt{(A and (not B)) or C}. As always, parentheses can be used to express the desired
composition.

The Boolean operators \texttt{and} and \texttt{or} are so-called \emph{short-circuit} operators:
their arguments are evaluated from left to right, and evaluation stops as soon as the outcome is
determined. For example, if \texttt{A} and \texttt{C} are true but \texttt{B} is false,
\texttt{A and b and C} does not evaluate the expression \texttt{C}. When used as a general value
and not as a Boolean, the return value of a short-circuit operator is the last evaluted argument.

\begin{minted}{bash}
    >>> 3 and 5 and 6 and 7
    7
    >>> 3 or 4 or 5
    3
    >>> 0 or 6 or 7
    6
\end{minted}

It is possible to assign the result of a comparison or other Boolean expression to a variable. For
example,
\begin{minted}{bash}
    >>> string1, string2, string3 = '', 'Trondheim', 'Hammer Dance'
    >>> non_null = string1 or string2 or string3
    >>> non_null
    'Trondheim'
\end{minted}

Note that in Python, unlike C, assignment cannot occur inside expressions. C programmers may
grumble about this, but it avoids a common class of problems encountered in C programs: typing
\texttt{=} in an expression when \texttt{==} was intended.
\footnote{https://stackoverflow.com/questions/2603956/can-we-have-assignment-in-a-condition}

\subsection{Comparing Sequences and Other Types}
Sequence objects may be compared to other objects with the same sequence type. The comparison uses
\emph{lexicographical} ordering: first the first two items are compared, and if they differ this
determines the outcome of the comparison; if they are equal, the next two items are compared, and
so on, until either sequence is exhausted. If two items to be compared are themselves sequences of
the same type, the lexicographical comparison is carried out recursively. If all items of two
sequence compare equal, the sequences are considered equal. If one sequence is an initial
sub-sequence of the other, the shorter sequence is the smaller (lesser) one. Lexicographical
ordering for strings uses the ASCII ordering for individual characters. Some examples of
comparisons between sequences of the same type:
\begin{minted}{bash}
    >>> (1, 2, 3) < (1, 2, 4)
    >>> [1, 2, 3] < [1, 2, 4]
    >>> 'ABC' < 'C' < 'Pascal' < 'Python'
    >>> (1, 2, 3, 4) < (1, 2, 4)
    >>> (1, 2) < (1, 2, -1)
    >>> (1, 2, 3) == (1.0, 2.0, 3.0)
    >>> (1, 2, ('aa', 'ab')) < (1, 2, ('abc', 'a'), 4)
\end{minted}

Note that comparing objects of different types is leagal. The outcome is deterministic but
arbitrary: the types are ordered by their name. Thus, a list is always smaller than a string, a
string is always smaller than a tuple, etc. Mixed numeric types are compared according to their
numeric value, so 0 equals 0.0, etc.

\section{Modules}
If you quit from the Python interpreter and enter it again, the definitions you have made
(functions and variables) are lost. Therefore, if you want to write a somewhat longer program, you
are better off using a text editor to prepare the input for the interpreter and running it with
that file as input instead. This is known as creating a \emph{script}. As your program gets longer,
you may want to split it into several files for easier maintenance. You may also want to use a
handy function that you've written in several programs without copying its definitions into each
program.

To support this, Python has a way to put definitions in a file and use them in a script or in an
interactive instance of the interpreter. Such a file is called a \emph{module}; definitions from a
module can be \emph{imported} into other modules or into the \emph{main} module (the collection of
variables that you have access to in a script executed at the top level and in calculator mode).

A module is a file containing Python definitions and statements. The file name is the module name
with the suffix \texttt{.py} appended. Within a module, the module's name (as a string) is
available as the value of the global variable \texttt{\_\_name\_\_}. For instance, use your favorite
text editor to create a file called \texttt{fibo.py} in the current directory with the following
contents:
\begin{minted}{python}
    # Fibonacci numbers module

    def fib(n):     # write Fibonacci series up to n
        a, b = 0, 1
        while b < n:
            print b,
            a, b = b, a+b

    def fib2(n):    # return Fibonacci series up to n
        result = []
        a, b = 0, 1
        while b < n:
            result.append(b)
            a, b = b, a+b
        return result
\end{minted}

Now enter the Python interpreter and import this module with the following command:
\begin{minted}{bash}
    >>> import fibo
\end{minted}

This does not enter the names of the functions defined in \texttt{fibo} directly in the current
symbol table; it only enters the module name \texttt{fibo} there. Using the module name you can
access the functions:
\begin{minted}{bash}
    >>> fibo.fib(1000)
    1 1 2 3 5 8 13 21 34 55 89 144 233 377 610 987
    >>> fibo.fib2(100)
    [1, 1, 2, 3, 5, 8, 13, 21, 34, 55, 89]
    >>> fibo.__name__
    fibo
\end{minted}

If you intend to use a function often you can assign it to a local name:
\begin{minted}{bash}
    >>> fib = fibo.fib
    >>> fib(500)
    1 1 2 3 5 8 13 21 34 55 89 144 233 377
\end{minted}

\subsection{More on Modules}
A module can contain executable statements as well as function definitions. These statements are
intended to initialize the module. They are executed only the \emph{first} time the module name is
encountered in an import statement. (They are also run if the file is executed as a script.)

Each module has its own private symbol table, which is used as the global symbol table by all
functions defined in the module. Thus, the author of a module can use global variables in the
module without worrying about accidental clashes with a user's global variables. On the other hand,
if you know what you are doing you can touch a module's global variables with the name notation
used to refer to its functions, \texttt{modname.itemname}.

Modules can import other modules. It is customary but not required to place all \texttt{import}
statements at the beginning of a module (or script, for that matter). The imported module names are
placed in the importing modules's global symbol table.

There is a variant of the \texttt{import} statement that imports names from a module directly into
the importing module's symbol table. For example:
\begin{minted}{bash}
    >>> from fibo import fib, fib2
    >>> fib(500)
    1 1 2 3 5 8 13 21 34 55 89 144 233 377
\end{minted}
This does not introduce the module name from which the imports are taken in the local symbol table
(so in the example, \texttt{fibo} is not defined).

There is even a variant to import all names that a module defines:
\begin{minted}{bash}
    >>> from fibo import *
    >>> fib(500)
    1 1 2 3 5 8 13 21 34 55 89 144 233 377
\end{minted}
This imports all names except those beginning with an underscore(\texttt{\_}).

Note that in general the practice of importing \texttt{*} from a module or package is frowned upon,
since it often causes poorly readable code. However, it is okay to use it to save typing in
interactive sessions.

\textbf{Note:} For efficiency reasons, each module is only imported once per interpreter session.
Therefore, if you change your modules, you must restart the interpreter --- or, if it's just one
module you want to test interactively, use \texttt{reload()}, e.g. \texttt{reload(modulename)}.

\subsubsection{Executing modules as scripts}
When you run a Python module with \mintinline{Python}{python fibo.py <arguments>} the code in the
module will be executed, just as if you imported it, but with the \texttt{\_\_name\_\_} set to
"\_\_main\_\_". That means that by adding this code at the end of your module:
\begin{minted}{python}
    if __name__ == "__main__":
        import sys
        fib(int(sys.argv[1]))
\end{minted}
you can make the file usable as a script as well as an importable module, because the code that
parses the command line only runs if the module is executed as the "main" file:
\begin{minted}{bash}
    duangdeiMac:6 duang$ python fibo.py 50
    1 1 2 3 5 8 13 21 34
\end{minted}

If the module is imported, the code is not run:
\begin{minted}{bash}
    >>> import fibo
    >>>
\end{minted}
This is often used either to provide a convenient user interface to a module, or for testing
purposes (running the module as a script executes a test suite).

\subsubsection{The Module Search Path}
When a module named \texttt{spam} is imported, the interpreter first searches for a built-in module
with that name. If not found, it then searches for a file named \texttt{spam.py} in a list of
directories given by the variable \texttt{sys.path}. \texttt{sys.path} is initialized from these
locations:
\begin{itemize}
    \item the directory containing the input script (or the current directory)
    \item \texttt{PYTHONPATH} (a list a directory names, with the same syntax as the shell variable
    \texttt{PATH})
    \item the installation-dependent default
\end{itemize}

After initialization, Python programs can modify \texttt{sys.path}. The directory containing the
script being run is placed at the begining of the search path, ahead of the standard library path.
This means that scripts in that directory will be loaded instead of modules of the same name in the
library directory. This is an error unless the replacement is intended.

\subsubsection{``Compiled'' Python files}
As an important speed-up of the start-up time for short programs that use a lot of standard modules,
if a file called \texttt{spam.pyc} exists in the directory where \texttt{spam.py} is found, this is
assumed to contain an already-"byte-compiled" version of the module \texttt{spam}. The mofication
time of the version of \texttt{spam.py} used to create \texttt{spam.pyc} is recorded in
\texttt{spam.pyc}, and the \texttt{.pyc} file is ignored if these don't match.

Normally, you don't need to do anything to create the \texttt{spam.pyc} file. Whenever
\texttt{spam.py} is successfully compiled, an attempt is made to write the compiled version to
\texttt{spam.pyc}. It is not an error if this attempt fails; if for any reason the file is not
written completely, the resulting \texttt{spam.pyc} file will be recognized as invalid and thus
ignored later. The contents of the \texttt{spam.pyc} file are platform independent, so a Python
module directory can be shared by machines of different architectures.

Some tips for experts:
\begin{itemize}
    \item When the Python interpreter is invoked with the \texttt{-o} flag, optimized code is
    generated and stored in \texttt{.pyo} files. The optimizer currently doesn't help much; it only
    removes \texttt{assert} statements. When \texttt{-o} is used, all bytecode is optimized;
    \texttt{.pyc} files are ignored and \texttt{.py} files are compiled to optimized bytecode.
    \item Passing two \texttt{-o} flags to the Python interpreter (\texttt{-oo}) will cause the
    bytecode compiler to perform optimizations that could in some rare cases result in
    malfunctioning programs. Currently only \texttt{\_\_doc\_\_} strings are removed from the
    bytecode, resulting in more compact \texttt{.pyo} files. Since some programs may rely on having
    these available, you should only use this option if you know what you're doing.
    \item A program doesn't run any faster when it is read from a \texttt{.pyc} or \texttt{.pyo}
    file than when it is read from \texttt{.py} file; the only thing that's faster about
    \texttt{.pyc} or \texttt{.pyo} files is the speed which they are loaded.
    \item When a script is run by giving its name of the command line, the bytecode for the script
    is never written to a \texttt{.pyc} or \texttt{.pyo} file. Thus, the startup time of a script
    may be reduced by moving most of its code to a module and having a small bootstrap script that
    imports that module. It is also possible to name a \texttt{.pyc} or \texttt{.pyo} file directly
    on the command line.
    \item It is possible to have a file called \texttt{spam.pyc} (or \texttt{spam.pyo} when
    \texttt{-o} is used) without a file \texttt{spam.py} for the same module. This can be used to
    distribute a library of Python code in a form that is moderately hard to reverse engineer.
    \item The module \texttt{compileall} can create \texttt{.pyc} files (or \texttt{.pyo} files
    when \texttt{-o} is used) for all modules in a directory.
\end{itemize}

\subsection{Standard Modules}
Python comes with a library of standard modules, described in a separated document, the Python
Library Reference. Some modules are built into the interpreter; these provide access to operations
that are not part of the core of the language but are nevertheless built in, either for efficiency
or to provide access to operating system primitives such as system calls. The set of such modules
is a configuration option which also depends on the underlying platform. For example, the
\texttt{winreg} module is only provided on Windows systems. One particular module deserves some
attention: \texttt{sys}, which is built into every Python interpreter. The variables
\texttt{sys.ps1} and \texttt{sys.ps2} define the strings used as primary and secondary prompts:
\begin{minted}{bash}
    >>> import sys
    >>> sys.ps1
    '>>> '
    >>> sys.ps2
    '... '
    >>> sys.ps3
    >>> sys.ps1 = 'C> '
    C> print 'Yuck!'
    Yuck!
\end{minted}

These two variables are only defined if the interpreter is in interactive mode.

The variable \texttt{sys.path} is a list of strings that determines the interpreter's search path
for modules. It is initialized to a default path taken from the environment variable
\texttt{PYTHONPATH}, or from a built-in default if \texttt{PYTHONPATH} is not set. You can modify
it using standard list operations:
\begin{minted}{bash}
    >>> import sys
    >>> sys.path.append('/ufs/gudio/lib/python')
\end{minted}

\subsection{The \texttt{dir()} Function}
The built-in function \texttt{dir()} is used to find out which names a module defines. It returns a
sorted list of strings:
\begin{minted}{bash}
    >>> import fibo, sys
    >>> dir(fibo)
    ['__builtins__', '__doc__', '__file__', '__name__', '__package__', 'fib', 'fib2']
    >>> dir(sys)
    ['__displayhook__', '__doc__', '__egginsert', '__excepthook__', '__name__', '__package__',
    '__plen', '__stderr__', '__stdin__', '__stdout__', '_clear_type_cache', '_current_frames',
    '_getframe', '_mercurial', 'api_version', 'argv', 'builtin_module_names', 'byteorder',
    'call_tracing', 'callstats', 'copyright', 'displayhook', 'dont_write_bytecode',
    'exc_clear', 'exc_info', 'exc_type', 'excepthook', 'exec_prefix', 'executable', 'exit',
    'flags', 'float_info', 'float_repr_style', 'getcheckinterval', 'getdefaultencoding',
    'getdlopenflags', 'getfilesystemencoding', 'getprofile', 'getrecursionlimit', 'getrefcount',
    'getsizeof', 'gettrace', 'hexversion', 'last_traceback', 'last_type', 'last_value',
    'long_info', 'maxint', 'maxsize', 'maxunicode', 'meta_path', 'modules', 'path', 'path_hooks',
    'path_importer_cache', 'platform', 'prefix', 'ps1', 'ps2', 'py3kwarning', 'setcheckinterval',
    'setdlopenflags', 'setprofile', 'setrecursionlimit', 'settrace', 'stderr', 'stdin',
    'stdout', 'subversion', 'version', 'version_info', 'warnoptions']
\end{minted}

Without arguments, \texttt{dir()} lists the names you have defined currently:
\begin{minted}{bash}
    >>> a = [1, 2, 3, 4, 5]
    >>> import fibo
    >>> fib = fibo.fib
    >>> dir()
    ['__builtins__', '__doc__', '__name__', '__package__', 'a', 'fib', 'fibo']
\end{minted}

Note that it lists all types of names: variables, modules, functions, etc.

\texttt{dir()} does not list the names of built-in functions and variables. If you want a list of
those, they are defined in the standard module \texttt{\_\_builtin\_\_}:
\begin{minted}{bash}
>>> import __builtin__
>>> dir(__builtin__)
['ArithmeticError', 'AssertionError', 'AttributeError', 'BaseException', 'BufferError',
'BytesWarning', 'DeprecationWarning', 'EOFError', 'Ellipsis', 'EnvironmentError', 'Exception',
'False', 'FloatingPointError', 'FutureWarning', 'GeneratorExit', 'IOError', 'ImportError',
'ImportWarning', 'IndentationError', 'IndexError', 'KeyError', 'KeyboardInterrupt',
'LookupError', 'MemoryError', 'NameError', 'None', 'NotImplemented', 'NotImplementedError',
'OSError', 'OverflowError', 'PendingDeprecationWarning', 'ReferenceError', 'RuntimeError',
'RuntimeWarning', 'StandardError', 'StopIteration', 'SyntaxError', 'SyntaxWarning',
'SystemError', 'SystemExit', 'TabError', 'True', 'TypeError', 'UnboundLocalError',
'UnicodeDecodeError', 'UnicodeEncodeError', 'UnicodeError', 'UnicodeTranslateError',
'UnicodeWarning', 'UserWarning', 'ValueError', 'Warning', 'ZeroDivisionError', '_',
'__debug__', '__doc__', '__import__', '__name__', '__package__', 'abs', 'all', 'any', 'apply',
'basestring', 'bin', 'bool', 'buffer', 'bytearray', 'bytes', 'callable', 'chr', 'classmethod',
'cmp', 'coerce', 'compile', 'complex', 'copyright', 'credits', 'delattr', 'dict', 'dir',
'divmod', 'enumerate', 'eval', 'execfile', 'exit', 'file', 'filter', 'float', 'format',
'frozenset', 'getattr', 'globals', 'hasattr', 'hash', 'help', 'hex', 'id', 'input', 'int',
'intern', 'isinstance', 'issubclass', 'iter', 'len', 'license', 'list', 'locals', 'long',
'map', 'max', 'memoryview', 'min', 'next', 'object', 'oct', 'open', 'ord', 'pow', 'print',
'property', 'quit', 'range', 'raw_input', 'reduce', 'reload', 'repr', 'reversed', 'round',
'set', 'setattr', 'slice', 'sorted', 'staticmethod', 'str', 'sum', 'super', 'tuple', 'type',
'unichr', 'unicode', 'vars', 'xrange', 'zip']
\end{minted}

\subsection{Packages}
Packages are a way of structuring Python's module namespace by using "dotted module names". For
example, the module name \texttt{A.B} designates a submodule named \texttt{B} in a package named
\texttt{A}. Just like the use of modules saves the authors of different moduels from having to
worry about each other's global variable names, the use of dotted module names saves the authors of
multi-module package like NumPy or the Python Imaging Library from having to worry about each
other's moudule names.


Suppose you want to design a collection of modules (a ``package'') for the uniform handling of
sound files and sound data. There are many different sound file formats (usually recognized by
their extension, for example: \texttt{.wav}, \texttt{.aiff}, \texttt{.au}), so you may need to
create and maintain a growing collection of modules for the conversion between the various file
formats. There are also many different operations you might want to perform on sound data (such
as mixing, adding echo, applying an equalizer function, creating an artificial stereo effect), so
in addition you'll be writing a never-ending stream of modules to perform these operations. Here's
a possible structure for your package (expressed in terms of a hierarchical filesystem):
\begin{minted}{text}
    sound/                          Top-level package
          __init__.py               Initialize the sound package
          formats/                  Subpackage for file format conversions
                  __init__.py
                  wavread.py
                  wavwrite.py
                  aiffread.py
                  aiffwrite.py
                  auread.py
                  auwrite.py
                  ...
          effects/                  Subpackage for sound effects
                  __init__.py
                  echo.py
                  surround.py
                  reverse.py
                  ...
          filters/                  Subpackage for filters
                  __init__.py
                  equalizer.py
                  vocoder.py
                  karaoke.py
                  ...
\end{minted}

When importing the package, Python searches through the directories on \texttt{sys.path} looking
for the package subdirectory.

The \texttt{\_\_init\_\_.py} files are required to make Python treat the directories as containing
packages; this is done to prevent directories with a common name, such as \texttt{string}, from
unintentionally hiding valid modules taht occur later on the module search path. In the simplest
case, \texttt{\_\_init\_\_.py} can just be an empty file, but it can also execute initialization
code for the package or set the \texttt{\_\_all\_\_} variable, described later.

Users of the package can import individual modules from the package, for example:
\begin{minted}{python}
    import sound.effects.echo
\end{minted}

This loads the submodule \texttt{sound.effects.echo}. It must be referened with its full name.
\begin{minted}{python}
    sound.effects.echo.echofilter(input, output, delay = 0.7, atten = 4)
\end{minted}

An alternative way of importing the submodule is:
\begin{minted}{python}
    from sound.effects import echo
\end{minted}

This also loads the submodule \texttt{echo}, and makes it available without its package prefix, so
it can be used as follows:
\begin{minted}{python}
    echo.echofilter(input, output, delay = 0.7, atten = 4)
\end{minted}

Yet another variation is to import the desired function or variable directly:
\begin{minted}{python}
    from sound.effects.echo import echofilter
\end{minted}

Again, this loads the submodule \texttt{echo}, but this makes its function \texttt{echofilter()}
directly available:
\begin{minted}{python}
    echofilter(input, output, delay = 0.7, atten = 4)
\end{minted}

Note that when using \mintinline{python}{from package import item}, the item can be either a
submodule (or subpackage) of the package, or some other anme defined in the package, like a
function, class or variable. The \texttt{import} statement first tests whether the item is defined
in the package; if not, it assumes it is a module and attempts to load it. If it fails to find it,
an \texttt{ImportError} exception is raised.

Contrarily, when using syntax like \mintinline{python}{import item.subitem.subsubitem}, each item
except for the last must be a package; the last item can be a module or a package but can't be a
class or a function or variable defined in the previous item.

\subsubsection{Importing \texttt{*} From a Package}
Now what happens when the user writes \mintinline{python}{from sound.effects import *}? Ideally,
one would hope that this somehow goes out the filesystem, finds which submodules are present in the
package, and imports them all. This could take a long time and importing sub-modules might have
unwanted side-effects that should only happen when the sub-module is explicitly imported.

The only solution is for the package author to provide an explicit index of the package. The
\mintinline{python}{import} statement uses the following convention: if a package's
\mintinline{python}{__init.py__} code defines a list named \mintinline{python}{__all__}, it is
taken to be the list of module names that should be imported when
\mintinline{python}{from package import *} is encountered. It is up to the package author to keep
this list up-to-date when a new version of the package is released. Package authors may also decide
not to support it, if they don't see a use for importing \texttt{*} from their package. For example,
the file \texttt{sound/effects/\_\_init\_\_.py} could contain the following code:
\begin{minted}{python}
    __all__ = ["echo", "surround", "reverse"]
\end{minted}

This would mean that \mintinline{python}{from sound.effects import *} would import the three named
submodules of the \texttt{sound} package.

If \texttt{\_\_all\_\_} is not defined, the statement \mintinline{python}{from sound.effects import
*} does \emph{not} import all submodules from the package \texttt{sound.effects} into the current
namespace; it only ensures that the package \texttt{sound.effects} has been imported (possibly
running any initialization code in \texttt{\_\_init.py\_\_}) and then imports whatever names are
defined in the package. This includes any names defined (and submodules explicitly loaded) by
\texttt{\_\_init\_\_.py}. It also includes any submodules of the package that were explicitly
loaded by previous \texttt{import} statements. Consider this code:
\begin{minted}{python}
    import sound.effects.echo
    import sound.effects.surround
    from sound.effects import *
\end{minted}
In this example, the \texttt{echo} and \texttt{surround} modules are imported in the current
namespace because they are defined by in the \texttt{sound.effects} package when the
\mintinline{python}{from...import} statement is executed. (This also works when
\texttt{\_\_all\_\_} is defined.)

Although certain modules are designed to export only names that follow certain patterns when you
use \mintinline{python}{import *}, it is still considered bad practice in production code.

Remember, there is nothing wrong with using \mintinline{python}{from package import
specific_submodule}! In fact, this the recommended notation unless the importing module needs to
use submodules with the same name from different packages.

\subsubsection{Intra-package References}
The submodules often needs to refer to each other. For example, the \texttt{surround} module might
use the \texttt{echo} module. In fact, such references are so common that the \texttt{import}
statement first looks in the containing package before looking in the standard module search path.
Thus, the \texttt{surround} module can simply use \mintinline{python}{imort echo} or
\mintinline{python}{from echo import echofilter}. If the imported module is not found in the current
package (the package of which the current module is a submodule), the \texttt{import} statement
looks for top-level module with the given name.

When packages are structured into subpackages (as with teh \texttt{sound} package in the example),
you can use absolute imports to refer to submodules of siblings packages. For example, if the
module \texttt{sound.filters.vocoder} needs to use the \texttt{echo} module in the
\texttt{sound.effects} package, it can use \mintinline{python}{from sound.effects import echo}.

Starting with Python 2.5, in addition to the implicit relative imports described above, you can
write explicit relative imports with the \mintinline{python}{from module import name} form of
import statement. These explicit relative imports use leading dots to indicate the current and
parent packages involed in the relative import. From the \texttt{surround} module for example, you
might use:
\begin{minted}{python}
    from . import echo
    from .. import formats
    from ..filters import equalizer
\end{minted}

Note that both explicit and implicit relative imports are bases on the name of the current module.
Since the name of the main module is always ``\texttt{\_\_main\_\_}'', modules intended for use as
the main module of a Python application should always use absolute imports.

\subsubsection{Packages in Multiple Directories}
Packages support one more special attribute, \texttt{\_\_path\_\_}. This is initialized to be a
list containing the name of the diretory holding the package's \texttt{\_\_init\_\_.py} before the
code in that file is executed. This variable can be modified; doing so affects future searches for
modules and subpackages contained in the package.

While this feature is not often needed, it can be used to extend the set of modules found in a
package.

\section{Input and Output}
\subsection{Fancier Output Formatting}
So far we've encountered two ways of writing values: \emph{expression statements} and the
\texttt{print} statement. (A third way is using the \texttt{write()} method of file objects; the
standard output file can be referenced as \texttt{sys.stdout}.)

Often you'll want more control over the formatting of your output than simply printing
space-separated values. There are two ways to format your output; the first way is to do all the
string handling yourself; using string slicing and concatenation operations you can create any
layout you can imagine. The string types have some methods that perform useful operations for
padding strings to a given colums width; these will be discussed shortly. The second way is to use
the \texttt{str.format()} method.

The \texttt{string} module contains a \texttt{Template} class which offers yet another way to
substitute values into strings.

One question remains, of course: how do you convert values to strings? Luckily, Python has ways to
convert any value to string: pass it to \texttt{repr()} or \texttt{str()} functions.

The \texttt{str()} function is meant to return representations of values which are fairly
human-readable, which \texttt{repr()} is meant to generate representations which can be read by the
interpreter (or will force a \texttt{SyntaxError} if there is no equivalent syntax). For objects
which don't have a particular representation for human consumption, \texttt{str()} will return the
same value as \texttt{repr()}. Many values, such as numbers or structures like lists and
dictionaries, have the same representation using either function. Strings and floating point
numbers, in particular, have two distinct representations.

Some examples:
\begin{minted}{bash}
    >>> s = 'Hello, world.'
    >>> str(s)
    'Hello, world.'
    >>> repr(s)
    "'Hello, world.'"
    >>> str(1.0/7.0)
    '0.142857142857'
    >>> repr(1.0/7.0)
    '0.14285714285714285'
    >>> x = 10 * 3.25
    >>> y = 200 * 200
    >>> s = 'The value of x is ' + repr(x) + ', and y is ' + repr(y) + '...'
    >>> print s
    The value of x is 32.5, and y is 40000...
    >>> # The repr() of a string adds string quotes and backslashes:
    ... hello = 'hello, world\n'
    >>> hellos = repr(hello)
    >>> print hellos
    'hello, world\n'
    >>> # The argument to repr() may be any Python object:
    ... repr((x, y, ('spam', 'eggs')))
    "(32.5, 40000, ('spam', 'eggs'))"
\end{minted}

Here are two ways to write a table of squares and cubes:
\begin{minted}{bash}
    >>> for x in range(1, 11):
    ...     # S.rjust(width) returns S right-justified in a string of length width
    ...     print repr(x).rjust(2), repr(x*x).rjust(3),
    ...     # Note trailing comma on previous line
    ...     print repr(x*x*x).rjust(4)
    ...
     1   1    1
     2   4    8
     3   9   27
     4  16   64
     5  25  125
     6  36  216
     7  49  343
     8  64  512
     9  81  729
    10 100 1000

    >>> for x in range(1, 11):
    ...     print '{0:2d} {1:3d} {2:4d}'.format(x, x*x, x*x*x)
    ...
     1   1    1
     2   4    8
     3   9   27
     4  16   64
     5  25  125
     6  36  216
     7  49  343
     8  64  512
     9  81  729
    10 100 1000
\end{minted}
(Note that in the first example, one space between each column was added by the way \texttt{print}
works: it always adds spaces between its arguments.)

This example demonstrates the \texttt{str.rjust()} method of strign objects, which right-justifies
a string in a field of a given width by padding it with spaces on the left. There are similar
methods \texttt{str.ljust()} and \texttt{str.center()}. These methods do not write anything, they
just return a new string. If the input string is too long, they don't truncate it, but return it
unchanged; this will mess up your column layout but that's usually better than the alternative,
which would be lying about a value. (If you really want truncation you can always add a slice
operation, as in \texttt{x.ljust(n)[:n]}.)

There is another method, \texttt{str.zfill(width)}, which pads a numeric string on the left with
zeros. It understands about plus and minus signs. The original string is returned if width is less
than or equal to \texttt{len(str)}.
\begin{minted}{bash}
    >>> '12'.zfill(5)
    '00012'
    >>> '-3.14'.zfill(7)
    '-003.14'
    >>> '3.14159265359'.zfill(5)
    '3.14159265359'
\end{minted}

Basic usage of \texttt{str.format()} method looks like this:
\begin{minted}{bash}
    >>> print 'We are the {} whos say "{}!"'.format('knights', 'Ni')
    We are the knights whos say "Ni!"
\end{minted}

The brackets and characters within them (called format fields) are replaced with the objects passed
into the \texttt{str.format()} method. A number in the brackets refers to the position of the
object passed into the \texttt{str.format()} method.
\begin{minted}{bash}
    >>> print '{0} and {1}'.format('spam', 'eggs')
    spam and eggs
    >>> print '{1} and {0}'.format('spam', 'eggs')
    eggs and spam
\end{minted}

If keyword arguments are used in the \texttt{str.format()} method, their values are referred to by
using the name of the arguments.
\begin{minted}{bash}
    >>> print 'This {food} is {adjective}.'.format(
    ...       food = 'spam', adjective = 'absolutely horrible')
    This spam is absolutely horrible.
\end{minted}

Positional and keyword arguments can be arbitrary combined:
\begin{minted}{bash}
    >>> print 'The story of {0}, {1}, and {other}.'.format('Bill', 'Manfred', other = 'Georg')
    The story of Bill, Manfred, and Georg.
    >>> print 'The story of {0}, {1}, and {other}.'.format('Bill', other = 'Georg', 'Manfred')
      File "<stdin>", line 1
    SyntaxError: non-keyword arg after keyword arg
\end{minted}

\texttt{'!s'} (apply \texttt{str()}) and \texttt{'!r'} (apply \texttt{repr()}) can be used to
convert the value before it is formatted.
\begin{minted}{bash}
    >>> import math
    >>> print 'The value of PI is approximately {}.'.format(math.pi)
    The value of PI is approximately 3.14159265359.
    >>> print 'The value of PI is approximately {!s}.'.format(math.pi)
    The value of PI is approximately 3.14159265359.
    >>> print 'The value of PI is approximately {!r}.'.format(math.pi)
    The value of PI is approximately 3.141592653589793.
\end{minted}

An optional \texttt{':'} and format specifier can follow the field name. This allows greater
control over how the value is formatted. The following example rounds Pi to three palces after the
decimal.
\begin{minted}{bash}
    >>> print 'The value of PI is approximately {0:.3f}.'.format(math.pi)
    The value of PI is approximately 3.142.
\end{minted}

Passing an integer after the \texttt{':'} will cause that field to be a minimum number of
characters wide. This is useful for making tables pretty.
\begin{minted}{bash}
    >>> for name, phone in table.items():
    ...     print '{0:10} ==> {1:10d}'.format(name, phone)
    ...
    Dcab       ==>       7678
    Sjored     ==>       4127
    Jack       ==>       4098
\end{minted}

If you have a really long format string that you don't want to split up, it would be nice if you
could reference the variables to be formatted by name instead of by position. This can be done by
simply passing the dict and using square brackets \texttt{'[]'} to access the keys.
\begin{minted}{bash}
    >>> table = {'Sjoerd': 4127, 'Jack': 4098, 'Dcab': 8637678}
    >>> print ('Jack: {0[Jack]:d}; Sjoerd: {0[Sjoerd]:d}; '
    ...        'Dcab: {0[Dcab]:d}'.format(table))
    Jack: 4098; Sjoerd: 4127; Dcab: 8637678
\end{minted}

This could also be done by passing the table as keyword arguments with the \texttt{**} notation.
\begin{minted}{bash}
    >>> table = {'Sjoerd': 4127, 'Jack': 4098, 'Dcab': 8637678}
    >>> print 'Jack: {Jack:d}; Sjoerd: {Sjoerd:d}; Dcab: {Dcab:d}'.format(**table)
    Jack: 4098; Sjoerd: 4127; Dcab: 8637678
\end{minted}

This is particularly useful in combination with the built-in function \texttt{vars()}, which
returns a dictionary containing all local variables.

\subsubsection{Old string formatting}
The \texttt{\%} operator can also be used for string formatting. It interprets the left argument
much like a \texttt{sprintf()}-style format string to be applied to the right argument, and returns
the string resulting from this formatting operation. For example:
\begin{minted}{bash}
    >>> import math
    >>> print 'The value of PI is approximately %5.3f.' % math.pi
    The value of PI is approximately 3.142.
\end{minted}

\subsection{Reading and Writing Files}
\mintinline{python}{open()} returns a file object, and is most commonly used with two arguments:
\mintinline{python}{open(filename, mode)}.
\begin{minted}{bash}
    >>> f = open('workfile', 'w')
    >>> print f
    <open file 'workfile', mode 'w' at 0x107dfc5d0>
\end{minted}
The first argument is a string containing the filename. The second argument is another string
containing a few characters  describing the way in which the file will be used. \emph{mode} can be
\texttt{'r'} when the file will only be read, \texttt{'w'} for only writing (an exsisting file with
the same name will be erased), and \texttt{'a'} opens the file for appending; any data wirtten to
the file is automatically added to the end. \texttt{'r+'} opens the file for both reading and
writing. The \emph{mode} argument is optional; \texttt{'r'} will be assumed if it's omitted.

On Windows, \texttt{'b'} appended to the mode opens the file in binary mode, so there are also
modes like \texttt{'rb'}, \texttt{'wb'}, and \texttt{'r+b'}. Python on Windows makes a distinction
between text and binary files; the end-of-line character in text files are automatically altered
slightly when data is read or written. This behind-the-scenes modification to file data is fine for
ASCII text files, but it'll corrupt binary data like that in \texttt{JPEG} or \texttt{EXE} files.
Be very careful to use binary mode when reading and writing such files. On Unix, it dosen't hurt to
append a \texttt{'b'} to the mode, so you can use it platform-independently for all binary files.

\subsubsection{Methods of File Objects}
The rest of the examples in this section will assume that a file object called \texttt{f} has
already been created.

To read a file's contents, call \mintinline{python}{f.read(size)}, which reads some quantity of
data and returns it as a string. \emph{size} is an optional numeric argument. When \emph{size} is
omitted or negative, the entire contents of the file will be read and returned; it's your problem
if the file is twice as large as your machine's memory. Otherwise, at most \emph{size} bytes are
read and returned. If the end of the file has been reached, \mintinline{python}{f.read()} will
return an empty string (\texttt{""}).
\begin{minted}{bash}
    >>> f.read()
    'This is the entire file.\n'
    >>> f.read()
    ''
\end{minted}

\mintinline{python}{f.readline()} reads a single line from the file; a newline character
\mintinline{python}{'\n'} is left at the end of the string, and is only omitted on the last line of
the file if the file doesn't end in a newline. This makes the return value unambiguous; if
\texttt{f.readline()} returns an empty string, the end of the file has been reached, while a blank
line is represented by \mintinline{python}{'\n'}, a string containing only a single newline.
\begin{minted}{bash}
    >>> # the file doesn't end in a newline
    ... f.readline()
    'This is the first line of the file.\n'
    >>> f.readline()
    'Second line of the file.\n'
    >>> f.readline()
    ''

    >>> # the file ends in a newline
    ... f.readline()
    'This is the first line of the file.\n'
    >>> f.readline()
    'Second line of the file.\n'
    >>> f.readline()
    '\n'
    >>> f.readline()
    ''
\end{minted}

For reading lines from a file, you can loop over the file object. This is memory efficient, fast,
and leads to simple code:
\begin{minted}{bash}
    >>> for line in f:
    ...     print line,
    ...
    This is the first line of the file.
    Second line of the file.
\end{minted}

If you want to read all the lines of a file in a list you can also use \mintinline{python}{list(f)}
or \mintinline{python}{f.readline()}.

\mintinline{python}{f.write(string)} writes the contents of \emph{string} to the file, returning
\mintinline{python}{None}.
\begin{minted}{bash}
    >>> f.write('This is a test\n')
\end{minted}

To write something other than a string, it needs to be converted to a string first:
\begin{minted}{bash}
    >>> value = ('the answer', 42)
    >>> s = str(value)
    >>> f.write(s)
\end{minted}

\mintinline{python}{f.tell()} returns an integer giving the file object's current position in the
file, measured in bytes from the beginning of the file. To change the file object's position, use
\mintinline{python}{f.seek(offset, from_what)}. The position is computed from adding \emph{offset}
to a reference point; the reference point is selected by the \emph{from\_what} argument. A
\emph{from\_what} value of 0 measures from the beginning of the file, 1 uses the current file
position, and 2 uses the end of the file as the reference point. \emph{from\_what} can be omitted
and defaults to 0, using the beginning of the file as the reference point.
\begin{minted}{bash}
    >>> f = open('workfile', 'r+')
    >>> f.write('0123456789abcdef')
    >>> f.seek(5)
    >>> f.read(1)
    '5'
    >>> f.seek(-3, 2)
    >>> f.read(1)
    'd'
\end{minted}

When you're done with a file, call \mintinline{python}{f.close()} to close it and free up any
system resources taken up by the open file. After calling \mintinline{python}{f.close()}, attempts
to use the file object will automatically fail.
\begin{minted}{bash}
    >>> f.close()
    >>> f.read()
    Traceback (most recent call last):
      File "<stdin>", line 1, in <module>
    ValueError: I/O operation on closed file
\end{minted}

It is a good practice to use the \mintinline{python}{with} keyword when dealing with file objects.
This has the advantage that the file is properly closed after its suite finishes, even if an
exception is raised on the way. It is also much shorter than writing equivalent
\mintinline{python}{try}-\mintinline{python}{finally} blocks:
\begin{minted}{bash}
    >>> with open('workfile', 'r') as f:
    ...     read_data = f.read()
    ...
    >>> f.closed
    True
    >>> read_data
    '0123456789abcdef'
\end{minted}

File objects have some additional methods, such as \texttt{isatty()} and \texttt{truncate()} which
are less frequently used; consult the Library Reference for a complete guide to file objects.

\subsubsection{Saving structured data with \texttt{json}}
Strings can easily be written to and read from a file. Numbers take a bit more effort, since the
\texttt{read()} method only returns strings, which will have to be passed to a function like
\texttt{int()}, which takes a string like '123' and returns its numeric value 123. When you want to
save more complex data types like nested lists and dictionaries, parsing and serializing by hand
becomes complicated.

Rather than having users constantly writing and debugging code to save complicated data types to
files, Python allows you to use the popular data interchange format called JSON (JavaScript Object
Notation). The standard module called \texttt{json} can take Python data hierachies, and convert
them to string representations; this process is called \emph{serializing}. Reconstructing the data
from the string representation is called \emph{deserializing}. Between serializing and
deserializing, the string representing the object may have been stored in a file or data, or sent
over a network connection to some distant machine.

\textbf{Note:} The JSON format is commonly used by modern applications to allow for data exchange.
Many programmers are already familiar with it, which makes it a good choice for interoperability.

If you have an object \texttt{x}, you can view its JSON string representation with a simple line of
code:
\begin{minted}{bash}
    >>> import json
    >>> json.dumps([1, 'simple', 'list'])
    '[1, "simple", "list"]'
\end{minted}

Another variant of the \texttt{dumps()} function, called \texttt{dump()}, simply serializes the
object to a file. So if \texttt{f} is a file object opened for writing, we can do this:
\begin{minted}{bash}
    >>> x = json.dumps([1, 'simple', 'list'])
    >>> f = open('workfile', 'r+')
    >>> json.dump(x, f)
    >>> f.seek(0)
    >>> f.readlines()
    ['"[1, \\"simple\\", \\"list\\"]"']
    >>> f.close()
\end{minted}

To decode the object again, if \texttt{f} is a file object which has been opened for reading:
\begin{minted}{bash}
    >>> with open('workfile', 'r') as f:
    ...     x = json.load(f)
    ...
    >>> x
    u'[1, "simple", "list"]'
\end{minted}

This simple serialization technique can handle lists and dictionaries, but serializing arbitrary
class instances in JSON requires a bit of extra effort. The reference for the \texttt{json} module
contains an explanation of this.

\textbf{See also:} \texttt{pickle} - the pickle module \\
Contrary to JSON, \emph{pickle} is a protocol which allows the serialization of arbitrary complex
Python objects. As such, it is specific to Python and cannot be used to communicate with
applications written in other languages. It is also insecure by default: deserializing pickle data
coming from an untrusted source can execute arbitrary code, if the data was crafted by a skilled
attacker.

\section{Errors and Exceptions}
Until now error messages haven't been more than mentioned, but if you have tried out the examples
you have probably seen some. There are (at least) two distinguishable kinds of errors:
\emph{syntax errors} and \emph{exceptions}.

\subsection{Syntax Errors}
Syntax errors, also known as parsing errors, are perhaps the most common kind of complaint you get
while you are still learning Python:
\begin{minted}{bash}
    >>> while True print 'Hello world'
      File "<stdin>", line 1
        while True print 'Hello world'
                       ^
    SyntaxError: invalid syntax
\end{minted}

The parser repeats the offending line and displays a little `arrow' pointing at the earliest point
in the line where the error was detected. The error iss caused by (or at least detected at) the
token \emph{preceding} the arrow: in the example, the error is detected at the keyword
\mintinline{python}{print}, since a colon (\texttt{':'}) is missing before it. File name and line
number are printed so you know where to look in case the input came from a script.

\subsection{Exceptions}
Even if statement or expression is syntactically correct, it may cause an error when an attempt is
made to execute it. Errors detected during execution are called \emph{exceptions} and are not
unconditionally fatal: you will soon learn how to handle them in Python programs. Most exceptions
are not handled by programs, however, and result in error messages as shown here:
\begin{minted}{bash}
    >>> 10 * (1 / 0)
    Traceback (most recent call last):
      File "<stdin>", line 1, in <module>
    ZeroDivisionError: integer division or modulo by zero
    >>> 4 + spam * 3
    Traceback (most recent call last):
      File "<stdin>", line 1, in <module>
    NameError: name 'spam' is not defined
    >>> '2' + 2
    Traceback (most recent call last):
      File "<stdin>", line 1, in <module>
    TypeError: cannot concatenate 'str' and 'int' objects
\end{minted}

The last line of the error message indicates what happened. Exceptions come in different types, and
the type is printed as part of the message: the types in the example are \texttt{ZeroDivisionError},
\texttt{NameError} and \texttt{TypeError}. The string printed as the exception type is the name of
the built-in exception that occurred. This is true for all built-in exceptions, but need not be
true for userdefined exceptions (although it is a useful convention). Standard exception names are
built-in indetifiers (not reserved keywords).

The rest of the line provides detail based on the type of exception and what caused it.

The preceding part of the error message shows the context where the exception happened, in the form
of a stack traceback. In general it contains a stack traceback listing source lines; however, it
will not display lines read from standard input.

Built-in Exceptions lists the built-in exceptions and their meanings.

\subsection{Handling Exceptions}
It is possible to write programs that handle selected exceptions. Look at the following example,
which asks the user for input until a valid integer has been entered, but allows the user to
interrupt the program (using \texttt{Control-C} or whatever the operating system supports); note
that a user-generated interruption is signalled by raising the \texttt{KeyboardInterrupt} exception.
\begin{minted}{bash}
    >>> while True:
    ...     try:
    ...         x = int(raw_input("Please enter a number: "))
    ...         break
    ...     except ValueError:
    ...         print "Oops! That was no valid number. Try again..."
    ...
    Please enter a number: j
    Oops! That was no valid number. Try again...
\end{minted}

The \texttt{try} statement works as follows:
\begin{itemize}
     \item First, the \emph{try clause} (the statement(s) between the \texttt{try} and
     \texttt{except} keywords) is executed.
     \item If no exception occurs during execution of the try clause, the rest of the clause is
     skipped. Then if its type matches the exception named after the \texttt{except} keyword, the
     except clause is executed, and then execution continues after the \texttt{try} statement.
     \item If an exception occurs which does not match the exception named in the except clause, it
     is passed on to outer \texttt{try} statement; if no handler is found, it is an \emph{unhandled
     exception} and execution stops with a message as shown above.
\end{itemize}

A \texttt{try} statement may have more than one except clause, to specify handles for different
exceptions. At most one handler will be executed. Handlers only handle exceptions that occur in the
corresponding try clause, not in other handlers of the same \texttt{try} statement. An except
clause may name multiple exceptions as a parenthesized tuple, for example:
\begin{minted}{bash}
    ... except (RuntimeError, TypeError, NameError):
    ...     pass
\end{minted}

Note that the parentheses aroundt this tuple are required, because
\mintinline{python}{except ValueError, e:} was the syntax used for what is normally written as
\mintinline{python}{except ValueError as e:} in modern Python (described below). The old syntax is
still supported for backwards compatibility. This means
\mintinline{python}{except RuntimeError, TypeError} is not equivalent to
\mintinline{python}{except (RuntimeError, TypeError):} but to
\mintinline{python}{except RuntimeError as TypeError:} which is not what you want.

The last except clause may omit exceptions name(s), to serve as a wildcard. Use this with extreme
caution, since it is easy to mask a real programming error in this way! It can also be used to
print an error message and then re-raise exception (allowing a caller to handle the exception as
well):
\begin{minted}{python}
    import sys

    try:
        f = open('myfile.txt')
        s = f.readline()
        i = int(s.strip())
    except IOError as e:
        print "I/O error({0}): {1}".format(e.errno, e.strerror)
    except ValueError:
        print "Could not convert data to an integer."
    except:
        print "Unexpected error:", sys.exc_info()[0]
        raise
\end{minted}

The \texttt{try} ... \texttt{except} statement has an optional \emph{else clause}, which, when
present, must follow all except clauses. It it useful for code that must be executed if the try
clause does not raise an exception. For example:
\begin{minted}{python}
    for arg in sys.argv[1:]:
        try:
            f = open(arg, 'r')
        except IOError:
            print 'cannot open', arg
        else:
            print arg, 'has', len(f.readlines()),  'lines'
            f.close()
\end{minted}

The use of the \texttt{else} clause is better than adding additional code to the \texttt{try}
clause because it avoids accidentally catching an exception that wasn't raised by the code being
protected by the \texttt{try} ... \texttt{except} statement.

When an exception occurs, it may have an associated value, also known as the exception's
\emph{argument}. The presence and type of the argument depend on the exception type.

The except clause may specify a variable after the exception name (or tuple). The variable is bound
to an exception instance with the arguments stored in \texttt{instance.args}. For convenience, the
exception instance defines \texttt{\_\_str\_\_()} so the arguments can be printed directly without
having to reference \texttt{.args}.

One may also instantiate an exception first before raising it and add any attributes to it as
desired.
\begin{minted}{bash}
    >>> try:
    ...     raise Exception('spam', 'eggs')
    ... except Exception as inst:
    ...     print type(inst)    # the exception instance
    ...     print inst.args     # arguments stored in .args
    ...     print inst          # __str__ allows args to be printed directly
    ...     x, y = inst.args
    ...     print 'x =', x
    ...     print 'y =', y
    ...
    <type 'exceptions.Exception'>
    ('spam', 'eggs')
    ('spam', 'eggs')
    x = spam
    y = eggs
\end{minted}

If an exception has an argument, it is printed as the last part (`detail') of the message for
unhandled exceptions.

Exception handlers don't just handle exceptions if they occur immediately in the try clause, but
also if they occur inside functions that are called (even indirectly) in the try clause. For
example:
\begin{minted}{bash}
    >>> def this_fails():
    ...     x = 1 / 0
    ...
    >>> try:
    ...     this_fails()
    ... except ZeroDivisionError as detail:
    ...     print 'Handling run-time error:', detail
    ...
    Handling run-time error: integer division or modulo by zero
\end{minted}

\subsection{Raising Exceptions}
The \mintinline{python}{raise} statement allows the programmers to force a specified exception to
occur. For example:
\begin{minted}{bash}
    >>> raise NameError('HiThere')
    Traceback (most recent call last):
      File "<stdin>", line 1, in <module>
    NameError: HiThere
\end{minted}

The sole argument to \mintinline{python}{raise} indicates the exception to be raised. This must be
either an exception instance or an exception (a class that derives from \texttt{Exception}):
\begin{minted}{bash}
    >>> try:
    ...     raise NameError('HiThere')
    ... except NameError:
    ...     print 'An exception flew by!'
    ...     raise   # re-raise the exception to see if there is a exception
    ...
    An exception flew by!
    Traceback (most recent call last):
      File "<stdin>", line 2, in <module>
    NameError: HiThere
\end{minted}

\subsection{User-defined Exceptions}
Programs may name their own exceptions by creating a new exception class (see Classes for more
about Python classes). Exceptions should typically be derived from the \texttt{Exceptions} class,
either directly or indirectly. For example:
\begin{minted}{bash}
    >>> class MyError(Exception):
    ...     def __init__(self, value):
    ...         self.value = value
    ...     def __str__(self):
    ...         return repr(self.value)
    ...
    >>> try:
    ...     raise MyError(2*2)
    ... except MyError as e:
    ...     print 'My exception occurred, value:', e.value
    ...
    My exception occurred, value: 4
    >>> raise MyError('oops!')
    Traceback (most recent call last):
      File "<stdin>", line 1, in <module>
    __main__.MyError: 'oops!'
\end{minted}

In this example, the default \texttt{\_\_init()\_\_} of \texttt{Exception} has been overridden.
The new behavior simply creates the \emph{value} attribute. This replaces the default behavior of
creating the \emph{args} attribute.

Exception classes can be defined which do anything any other class can do, but are usually kept
simple, often only offering a nubmer of attributes that allow information about the error to be
extracted by handles for the exception. When creating a module that can raise several distinct
errors, a common practice is to create a base class for exceptions defined by that module, and
subclass that to create specific exception classes for different error conditions:
\begin{minted}{python}
    class Error(Exception):
        """Base class for exceptions in this module."""
        pass

    class InputError(Error):
        """Exceptions raised for errors in the input.

        Attributes:
            expr -- input expression in which the error occurred
            msg  -- explanation of the error
        """

        def __init__(self, expr, msg):
            self.expr = expr
            self.msg = msg

    class TransitionError(Error):
        """Raised when an operation attempts a state transition that's not allowed.

        Attributes:
            prev -- state at beginning of transition
            next -- attempted new state
            msg  -- explanation of why the specific transition is not allowed
        """

        def __init__(self, prev, next, msg):
            self.prev = prev
            self.next = next
            self.msg = msg
\end{minted}

Most exceptions are defined with names that end in ``Error'', similar to the naming of the standard
exceptions.

Many standard modules define their own exceptions to report errors that may occur in functions they
define.

\subsection{Defining Clean-up Actions}
The \texttt{try} statement has another optional clause which is intended to define clean-up actions
that must be executed under all circumstances. For example:
\begin{minted}{bash}
    >>> try:
    ...     raise KeyboardInterrupt
    ... finally:
    ...     print 'Goodbye, world!'
    ...
    Goodbye, world!
    Traceback (most recent call last):
      File "<stdin>", line 2, in <module>
    KeyboardInterrupt
\end{minted}

A \emph{finally clause} is always executed before leaving the \texttt{try} statement, whether an
exception has occurred or not. When an exception has occurred in the \texttt{try} clause and has
not been handled by an \texttt{except} clause (or it has occurred in an \texttt{except} or
\texttt{else} clause), it is re-raised after the \texttt{finally} clause has been executed. The
\texttt{finally} clause is also executed ``on the way out'' when any other clause of the
\texttt{try} statement is left via a \texttt{break}, \texttt{continue} or \texttt{return} statement.
A more complicated example (having \texttt{except} and \texttt{finally} clauses in the same
\texttt{try} statement works as of Python 2.5):
\begin{minted}{bash}
    >>> def divide(x, y):
    ...     try:
    ...         result = x / y
    ...     except ZeroDivisionError:
    ...         print "division by zero!"
    ...     else:
    ...         print "result is", result
    ...     finally:
    ...         print "executing finally clauses"
    ...
    >>> divide(2, 1)
    result is 2
    executing finally clauses
    >>> divide(2, 0)
    division by zero!
    executing finally clauses
    >>> divide("2", "1")
    executing finally clauses
    Traceback (most recent call last):
      File "<stdin>", line 1, in <module>
      File "<stdin>", line 3, in divide
    TypeError: unsupported operand type(s) for /: 'str' and 'str'
\end{minted}

As you can see, the \texttt{finally} clause is executed in any event. The \texttt{TypeError} raised
by dividing two strings is not handled by the \texttt{except} clause and therefore re-raised after
the \texttt{finally} clause has been executed.

In real world applications, the \texttt{finally} clause is useful for releasing external resources
(such as files or network connections), regardless of whether the use of the resource was
successful.

\subsection{Predefined Clean-up Actions}
Some objects define standard clean-up actions to be undertaken when the object is no longer needed,
regardless of whether or not the operation using the object succeeded or failed. Look at the
following example, which tries to open a file and print its contents to the screen.
\begin{minted}{python}
    for line in open("myfile.txt"):
        print line,
\end{minted}

The problem with this code is that it leaves the file open for an interminate amount of time after
the code has finished executing. This is not an issue in simple scripts, but can be a problem for
larger applications. The \texttt{with} statement allows objects like files to be used in a way that
ensures they are always cleaned up promptly and correctly.
\begin{minted}{python}
    with open("myfile.txt") as f:
        for line in f:
            print line,
\end{minted}

After the statement is executed, the file \texttt{f} is always closed, even if a problem was
encountered while processing the lines. Other objects which provide predefined clean-up actions
will indicate this in their documentation.








\end{document}
