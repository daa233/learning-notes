%-*- coding: UTF-8 -*-
% Notes.tex
%
\documentclass[UTF8]{ctexart}
\usepackage{geometry}
\geometry{a4paper, centering, scale=0.8}
\usepackage{minted}

\title{\heiti 《程序设计与算法》——《C++ 程序设计》学习笔记}
\author{\kaishu Du Ang \\ \texttt{du2ang233@gmail.com} }
\date{\today}

\begin{document}
\maketitle

\section{函数指针}
\subsection{函数指针的定义}
程序运行期间,每个函数都会占用一段连续的内存空间。而函数名就是该函数所占内存区域的起始地址(也称“入口地址”)。将入口地
址赋给一个指针变量,那该指针就指向了这个函数,这种指向函数的指针变量称为“函数指针”。我们可以通过函数指针来调用函数。

定义函数指针:类型名 (*指针变量名)(参数类型1, 参数类型2, ...)

例如:\mintinline{C++}{int (*pf)(int, char);} 表示 \texttt{pf} 是一个函数指针,它所指向的函数返回类型是
\texttt{int},该函数应有两个参数,第一个是 \texttt{int} 类型,第二个是 \texttt{char} 类型。

\subsection{函数指针和 qsort 库函数}
为什么要定义函数指针呢?干嘛不直接用函数名来调用?

如果要对数组排序,需要知道:
\begin{itemize}
    \item 数组的起始地址
    \item 数组元素的个数
    \item 每个元素的大小(由此可以算出每个元素的地址)
    \item 元素谁在前谁在后的规则
\end{itemize}

C 语言快速排序库函数可以对任意类型的数组进行排序:
\begin{minted}{C++}
    void qsort(void *base, int nelem, usigned int width,
        int (*pfCompare)(const viod *, cont void *));
\end{minted}

\begin{itemize}
    \item \texttt{base}: 待排数组的起始地址
    \item \texttt{nelem}: 带排序数组的元素个数
    \item \texttt{width}: 待排序数组的每个元素的大小(以字节为单位)
    \item \texttt{pfCompare}: 比较函数(程序员自己编写)的地址
\end{itemize}

\begin{minted}{C++}
    int compare(const void *elem1, const void *elem2);    // 比较函数
\end{minted}
比较函数的编写规则:
\begin{enumerate}
    \item 如果 \texttt{*elem1} 应该排在 \texttt{*elem2} 前面,则函数返回值是负整数
    \item 如果 \texttt{*elem1} 和 \texttt{*elem2} 哪个排在前面都行,那么函数返回 0
    \item 如果 \texttt{*elem1} 应该排在 \texttt{*elem2} 后面,则函数返回值是正整数
\end{enumerate}

一个比较函数的示例:
\begin{minted}{C++}
    int myCompare(const void *elem1, const void *elem2)
    {
        unsigned int *p1, *p2;
        p1 = (unsigned int *) elem1;    // 只写“*elem1” 非法,编译器不知道其类型和大小
        p2 = (unsigned int *) elem2;
        return (*p1 % 10) - (*p2 % 10); // 个位数小的排在前面
    }
\end{minted}

\section{命令行参数}
将用户在命令行窗口输入可执行文件名的方式启动程序时,跟在可执行文件后面的那些字符串,称为“命令行参数”。命令行参数可以有
多个,以空格分隔。

\begin{minted}{C++}
    int main(int argc, char *argv[])
    {
        ...
    }
\end{minted}

\texttt{argc}: 代表程序启动时,命令行参数的个数。执行程序本身的文件名是第一个命令行的参数。
\texttt{argv}: 指针数组,其中的每个元素都是一个 \texttt{char *} 类型的指针,该指针指向存放命令行参数的字符串。

\begin{minted}{C++}
    int main(int argc, char const *argv[]) {
        for (int i = 0; i < argc; i++) {
            cout << argv[i] << " ";
        }
        cout << endl;
        return 0;
    }
\end{minted}

如果参数中包含空格,需要将参数用 \texttt{"..."} 引起来

输入:\mintinline{bash}{./CmdArguments hello world "hello world"}

输出:\mintinline{bash}{./CmdArguments hello world hello world}

\section{位运算}
位运算:对于整数类型(\texttt{int}、\texttt{char}、\texttt{long} 等)变量中的某一位(bit),或者若干位进行操作。

C/C++ 语言提供了六种位运算符来进行位运算操作:
\begin{itemize}
    \item \texttt{\&} 按位与(双目)
    \item \texttt{|} 按位或(双目)
    \item \texttt{\^{}} 按位异或(双目)
    \item \texttt{$\sim$} 按位非(取反)(单目)
    \item \texttt{<<} 左移(双目)
    \item \texttt{>>} 右移(双目)
\end{itemize}

\subsection{按位与 \texttt{\&}}
通常用按位与来将某变量中的某些位清 0 且同时保留其他位不变。
\begin{minted}{C++}
    n = n & 0xffffff00; // 将 int 类型的 n 的低 8 位全部置 0
    n &= 0xffffff00;
    n &= 0xff00;        // 将 short 类型的 n 的低 8 位全部置 0
\end{minted}

也可以用按位与来获取某变量中的某一位。例如我们想判断 \texttt{int} 型变量 \texttt{n} 的右数第 7 位是否为 1,只需看
表达式 \mintinline{C++}{n & 0x80} 的值是否等于 \texttt{0x80} 即可。

\subsection{按位或 \texttt{|}}
按位或通常用来将某变量中的某些位置 1 且保持其他位不变。
\begin{minted}{C++}
    n |= 0xff;  // 将 int 型变量 n 的低 8 位全置成1
\end{minted}

\subsection{按位异或 \texttt{\^{}}}
按位异或运算通常用来将某变量中的某些位取反,且保持其他位不变。
\begin{minted}{C++}
    n ^= 0xff;  // 将 int 型变量 n 的低 8 位取反
\end{minted}

异或运算的特点:如果 $a$ \^{} $b = c$,那么 $c$ \^{} $b = a$,$c$ \^{} $a = b$。(穷举法可证)

可以利用此规律用来进行最简单的加密和解密。

另外异或运算还能实现不通过临时变量,就能交换两个变量的值:
\begin{minted}{C++}
    int a = 5, b = 7;
    a = a ^ b;
    b = b ^ a;
    a = a ^ b;
\end{minted}

\subsection{按位非 \texttt{$\sim$}}
按位非运算符 \texttt{$\sim$} 是单目运算符,其功能是将操作数中的二进制位 0 变成 1,1 变成 0。

\subsection{左移 \texttt{<<}}
表达式 \mintinline{C++}{a << b} 的值是将 \texttt{a} 各二进制位全部左移 \texttt{b} 位后得到的值。左移时,高位
丢弃,低位补 0。注意 \texttt{<<} 是双目运算符,这里是一个表达式,\texttt{a} 的值不因运算而改变。

实际上,左移 1 位,就等于是乘 2,左移 n 位,就等于乘 $2^n$。而左移操作比乘法操作快得多。

\subsection{右移\texttt{>>}}
表达式 \mintinline{C++}{a >> b} 的值是:将 \texttt{a} 各二进制位全部右移 \texttt{b} 位后得到的值。右移时,移
出最右边的位就被丢弃。同样地,\texttt{a} 的值不因右移运算而改变。

对于无符号数,右移时高位补 0。对于有符号数,如 \texttt{long}、\texttt{int}、\texttt{short}、\texttt{char}
类型变量,在右移时,符号位(即最高位)将一起移动,并且大多数 C/C++ 编译器规定,如果原符号位为 1,则右移时最高位就补充
1,原符号位为 0,则右移时高位就补充 0。

实际上,右移 n 位,就相当于左操作数除以 $2^n$。但是有时候不能除尽,需要将结果往小里取整。

思考题:有两个 \texttt{int} 型的变量 \texttt{a} 和 \texttt{n}(\texttt{0 <= n <= 31}),要求写一个表达式,
使该表达式的值和 \texttt{a} 的第 \texttt{n} 位相同。

答案:\mintinline{C++}{(a >> n) & 1} \qquad 如果 \texttt{n} 不为 31,答案还可以是
\mintinline{C++}{(a & (1 << n)) >> n}







\end{document}
