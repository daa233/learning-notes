%-*- coding: UTF-8 -*-
% TypesettingText.tex
%
\documentclass[UTF8]{ctexart}
\usepackage{geometry}
\geometry{a4paper, centering, scale=0.8}
\usepackage{minted}
\usepackage{textcomp}
\usepackage[gen]{eurosym}

\title{\heiti Chapter 2 Typesetting Text}
\author{\kaishu Du Ang \\ \texttt{du2ang233@gmail.com} }
\date{\today}

\begin{document}
\maketitle

\section{文章和语言的结构}
写文章最重要的一点是要把想法、信息、知识传达给读者,而好的文章结构能够帮助读者更好地查看、感受和理解我们想传达的东西。

\LaTeX 中最重要的文本单位是段(paragraph)。一段文字应该只包含一个思想或一种想法。写文章时什么时候分段?应该怎么分段呢?如果要写一个新的想法了,
那就另起一段,对应在源码中空一行);否则,可以用换行符(line breaking)来继续写原来的想法,对应在源码中使用
\mintinline{LaTeX}{\\} 或 \mintinline{LaTeX}{\newline}。

很多人都低估了合理分段的重要性。在\LaTeX 中,很多人甚至都不知道什么是分段,自己已经新起了一段都不知道。按照我的理解,一般情况下是不需要使用换
行符的,在该分段的时候在代码里空一行另起一段就行了。但是在使用公式(equation)的时候需要考虑好该使用什么,这时候很容易犯上述的错误。下面是说明
该换行还是该另起一段的三个正确示例:
\begin{minted}{LaTeX}
    % Example 1
    \ldots when Einstein introduced his formula
    \begin{equation}
        e = m \cdot c^2 \; ,
    \end{equation}
    which is at the same time the most widely known
    and the least well understood physical formula.

    % Example 2
    \ldots from which follows Kirchhoff’s current law:
    \begin{equation}
        \sum_{k=1}^{n} I_k = 0 \; .
    \end{equation}

    Kirchhoff’s voltage law can be derived \ldots

    % Example 3
    \ldots which has several advantages.
    \begin{equation}
        I_D = I_F - I_R
    \end{equation}
    is the core of a very different transistor model. \ldots
\end{minted}

\section{断行和断页}
\subsection{合理分段}
\begin{itemize}
    \item \mintinline{LaTeX}{\\} 或 \mintinline{LaTeX}{\newline}:断行、不另起一段。
    \mintinline{LaTeX}{\\} 也在表格、公式等地方用于分行,而 \mintinline{LaTeX}{\newline} 只用于文本段落中。
    \item \mintinline{LaTeX}{\\*}:断行、不另起一段、不断页。
    \item \mintinline{LaTeX}{\newpage} 或 \mintinline{LaTeX}{\clearpage}:断页。二者有些细微的区别:一是在双栏排版
    中 \mintinline{LaTeX}{\newpage} 只起到另起一栏的作用;二是涉及到浮动体的排版上行为不同。
    \item \mintinline{LaTeX}{\linebreak[n]}、\mintinline{LaTeX}{\nolinebreak[n]}、\mintinline{LaTeX}{\pagebreak[n]}、
    \mintinline{LaTeX}{\nopagebreak[n]}:向 \LaTeX 建议哪些地方适合断行、断页,
    哪些地方不适合断行、断页。n是数字,代表适合/不适合的程度,取值范围为0到4,默认为4。数字越大代表程度越高。
\end{itemize}

一般\LaTeX 都会努力找到最适合断行的地方。但是有些时候——比如它不知道该如何用连字符分割单词的时候,它可能会让这一行文字从右边伸出这一段,然后
报出\texttt{overfull hbox}的警告。这时使用 \mintinline{LaTeX}{\sloppy} 命令可以使单词之间的间距增大来避免这个问题,但是这时会报出
\texttt{underfull hbox}的警告。大多数情况下这样排版出来的都不太好看。可以使用 \mintinline{LaTeX}{\fussy} 命令恢复 \LaTeX 的默认方式。

\section{连字符(Hyphenation)}
对于绝大部分单词,\LaTeX 都能够找到合适的断词位置,在断开的行尾加上连字符 \texttt{-}。如果一些单词没能自动断词,我们可以在单词内手动使用
\mintinline{LaTeX}{\-} 命令指定断词的位置。另外,也可以使用 \mintinline{LaTeX}{\hyphenation{word list}} 命令来指定使用连字符
的位置,例如 \mintinline{LaTeX}|\hyphenation{FORTRAN Hy-phen-a-tion}|,其中的 \texttt{word list}是不区分大小写的。

\mintinline{LaTeX}|\mbox{text}| 或 \mintinline{LaTeX}|\fbox{text}|:会避免 \texttt{text}被连字符分开。
\mintinline{LaTeX}|\fbox| 比 \mintinline{LaTeX}|\mbox| 多了个可见的框。

\section{预定义好的字符串}
\begin{itemize}
    \item \mintinline{LaTeX}{\today}:\today(打印当天日期)
    \item \mintinline{LaTeX}{\TeX}:\TeX
    \item \mintinline{LaTeX}{\LaTeX}:\LaTeX
    \item \mintinline{LaTeX}{\LaTeXe}:\LaTeXe
\end{itemize}

\section{特殊符号}
\subsection{引号(Quotation Marks)}
\begin{itemize}
    \item 双引号:\mintinline{LaTeX}{``...text...''}
    \item 单引号:\mintinline{LaTeX}{`...text...'}
\end{itemize}

\subsection{短划线(Dashes)和连字符(Hyphens)}
在 \LaTeX 中有下面四种横杠:
\begin{itemize}
    \item \mintinline{LaTeX}{-}:-,连字符(hyphen),用于连接词语
    \item \mintinline{LaTeX}{--}:--,短破折号(en-dash),常用于连接数字表示起止范围
    \item \mintinline{LaTeX}{---}:---,长破折号(em-dash),常用于表示意思的转换
    \item \mintinline{LaTeX}{$-$}:$-$,减号(minus sign)
\end{itemize}

\subsection{波浪线(Tilde)}
\begin{itemize}
    \item \mintinline{LaTeX}{\~{}}:\~{}
    \item \mintinline{LaTeX}{$\sim$}:$\sim$
\end{itemize}

\subsection{斜杠(Slash)}
\begin{itemize}
    \item \mintinline{LaTeX}{read/write}:read/write(不允许用连字符拆分)
    \item \mintinline{LaTeX}{read\slash write}:read\slash write(允许用连字符拆分)
\end{itemize}

\subsection{度(Degree Symbol)}
\begin{itemize}
    \item \mintinline{LaTeX}{$30\,^{\circ}\mathrm{C}$}\footnote{这里的 \mintinline{LaTeX}{\,} 会输出空格}
    :$30\,^{\circ}\mathrm{C}$
    \item \mintinline{LaTeX}{30 \textcelsius{}}:30 \textcelsius{}
    \item \mintinline{LaTeX}{86 \textdegree{}F}:86 \textdegree{}F
\end{itemize}

\subsection{欧元符号}
要想使用欧元符号,需要先在导言区通过 \mintinline{LaTeX}{\usepackage{textcomp}} 命令导入宏包,然后再使用\\
\mintinline{LaTeX}{\texteuro} 命令输出欧元符号。如果所用的字体不包含欧元符号或者想用别的字体的欧元符号,可以通过
\mintinline{LaTeX}{\usepackage[official]{eurosym}} 导入 \texttt{eurosym} 宏包,然后用 \mintinline{LaTeX}{\euro} 输出官方的
欧元符号。用\texttt{gen}来替换\texttt{official}参数可以使用和当前字体匹配的欧元符号。
\begin{itemize}
    \item \mintinline{LaTeX}{\texteuro}:\texteuro
    \item \mintinline{LaTeX}{\euro}:\euro
\end{itemize}

\subsection{省略号(Ellipsis)}
\LaTeX 提供了命令 \mintinline{LaTeX}{\ldots} 来生成省略号,相对于直接输入三个点的方式更为合理。\mintinline{LaTeX}{\ldots} 和 \mintinline{LaTeX}{\dots}
是两个等效的命令。
\begin{itemize}
    \item \mintinline{LaTeX}{Apples, bananas, ...}:Apples, bananas, ...
    \item \mintinline{LaTeX}{Apples, bananas, \ldots}:Apples, bananas, \ldots
    \item \mintinline{LaTeX}{Apples, bananas, \dots}:Apples, bananas, \dots
\end{itemize}

\subsection{连字(Ligatures)}
有些相邻的字母在排版时会连接起来,可以通过 \mintinline{LaTeX}{\mbox{}} 命令避免它们相连。
\begin{itemize}
    \item \mintinline{LaTeX}{ffshfilfluffia}:ffshfilfluffia(相连的情况)
    \item \mintinline{LaTeX}{f\mbox{}fshf\mbox{}ilf\mbox{}luf\mbox{}f\mbox{}ia}:
    f\mbox{}fshf\mbox{}ilf\mbox{}luf\mbox{}f\mbox{}ia(没有相连的情况)
\end{itemize}

\subsection{重音符号(Accents)和特殊符号}
原始代码如下:
\begin{minted}{LaTeX}
    H\^otel, na\"\i ve, \'el\`eve, \\
    sm\o rrebr\o d, !'Se\~norita!, \\
    Sch\"onbrunner Schlo\ss{}
    Stra\ss e
\end{minted}

输出结果如下:
\begin{quote}
    H\^otel, na\"\i ve, \'el\`eve, \\
    sm\o rrebr\o d, !`Se\~norita!, \\
    Sch\"onbrunner Schlo\ss{}
    Stra\ss e
\end{quote}


\end{document}
