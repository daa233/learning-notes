%-*- coding: UTF-8 -*-
% Specialities.tex
%
\documentclass[UTF8]{ctexart}
\usepackage{geometry}
\geometry{a4paper, centering, scale=0.8}
\usepackage{minted}

\title{\heiti 第4章 \quad 特色功能}
\author{\kaishu Du Ang \\ \texttt{du2ang233@gmail.com} }
\date{\today}

\begin{document}
\maketitle

\tableofcontents

\newpage

在编写大型文档的时候,\LaTeX 还提供了一些像创建索引、管理参考文献等一些特色功能,详情见 \emph{\LaTeX Manual} 和
\emph{The \LaTeX Companion}。

\section{包含 Encapsulated PostScript}
借助 \texttt{figure} 和 \texttt{table} 环境,\LaTeX 能够支持一些像图像、图形这种简单的浮动体。

在基本的 \LaTeX 中或 \LaTeX 的扩展包中,有很多种方法能够生成一些实际的图形。一种比较简单的方法是,通过一些专业软件生
成这些图形,然后将它们包含到文档中。这里我们仅讨论使用 Encapsulated PostScript(EPS)来生
成图形,因为这种方法非常简单,而且获得了广泛的应用。为了使用 EPS 格式的图片,必须要有 PostScript 打印机来输出。

D. P. Carlisle 开发的 \texttt{graphicx} 宏包提供了很多包含图片的命令,这个宏包属于“graphics”宏集。

假设现在的系统有可以输出 PostScript 打印机,也安装好了 \texttt{graphicx} 宏包,可以根据下面的步骤在文档中包含图
片:
\begin{enumerate}
    \item 通过画图程序输出 EPS 格式的图片。
    \item 通过 \mintinline{LaTeX}|\usepackage[driver]{graphicx}| 命令,在导言区引入 \texttt{graphicx}
    宏包。 \\
    其中 \emph{driver} 是 dvi 转 PostScript 的转换程序,最常用的一种叫 \texttt{dvips}。知道
    \emph{driver} 的名字后,\texttt{graphicx} 宏包就可以选择正确的方法来将图形信息插入到 \texttt{.dvi} 文件
    中,然后打印机就能理解它并且正确地包含 \texttt{.eps} 文件。
    \item 在文档中使用 \mintinline{LaTeX}|\includegraphics[key=value, ...]{file}| 来包含 \emph{file}。
    \\ 命令中的可选参数允许有多个,之间用逗号隔开。\emph{key} 可以是 \texttt{width}、\texttt{height}、
    \texttt{angle}、\texttt{scale} 等参数,用于对包含的图形进行调整。
\end{enumerate}

示例代码:
\begin{minted}{LaTeX}
    \begin{figure}
        \centering
        \includegraphics[angle=90, width=0.5\textwidth]{test}
        \caption{This is a test.}
    \end{figure}
\end{minted}

上面的代码包含了存储好的 \texttt{test.eps} 图片。图片旋转了90度,并且图片的宽度缩放到了标准图片的0.5倍。由于没有
指定高度,所以默认宽高比是1。宽度和高度也可以指定为具体的长度。

\section{参考文献(Bibliography)}
通过 \texttt{thebibliography} 环境来生成参考文献。每一个条目都以 \mintinline{LaTeX}|\bibitem[lable]{marker}|
开头,再通过 \mintinline{LaTeX}|\cite{marker}| 命令,就可以用来在文档中引用书籍、文章、论文等。

如果不指定 \emph{label} 参数,所有的参考文献条目会自动编号。\mintinline{LaTeX}|\begin{thebibliography}| 命令
后的参数用来定义应该为条目编号预留多少空隙。在下面的示例中,该参数为 \texttt{\{99\}},表示所有的参考文献条目编号都不
能比数字99更宽。

示例代码:
\begin{minted}{LaTeX}
    Partl~\cite{pa} has proposed that \ldots
    \begin{thebibliography}{99}
    \bibitem{pa} H.~Partl: \emph{German \TeX}, TUGboat Volume~9, Issue~1 (1988)
    \end{thebibliography}
\end{minted}

示例输出:
Partl~\cite{pa} has proposed that \ldots
\begin{thebibliography}{99}
\bibitem{pa} H.~Partl: \emph{German \TeX}, TUGboat Volume~9, Issue~1 (1988)
\end{thebibliography}

对于更大型的项目,使用 Bib\TeX 是更好的选择。可以利用 Bib\TeX 建立一个参考文献数据库,然后再在文档中引用相关的文献。
Bib\TeX 产生的参考文献格式是通过样式文件定义的,网上有很多现成的样式文件可供选择。




\end{document}
