%-*- coding: UTF-8 -*-
% Specialities.tex
%
\documentclass[UTF8]{ctexart}
\usepackage{geometry}
\geometry{a4paper, centering, scale=0.8}
\usepackage{minted}
\usepackage{float}

\title{\heiti 第4章 \quad 特色功能}
\author{\kaishu Du Ang \\ \texttt{du2ang233@gmail.com} }
\date{\today}


\begin{document}
\maketitle

\tableofcontents

\newpage

在编写大型文档的时候,\LaTeX 还提供了一些像创建索引、管理参考文献等一些特色功能,详情见 \emph{\LaTeX Manual} 和
\emph{The \LaTeX Companion}。

\section{包含 Encapsulated PostScript}
借助 \texttt{figure} 和 \texttt{table} 环境,\LaTeX 能够支持一些像图像、图形这种简单的浮动体。

在基本的 \LaTeX 中或 \LaTeX 的扩展包中,有很多种方法能够生成一些实际的图形。一种比较简单的方法是,通过一些专业软件生
成这些图形,然后将它们包含到文档中。这里我们仅讨论使用 Encapsulated PostScript(EPS)来生
成图形,因为这种方法非常简单,而且获得了广泛的应用。为了使用 EPS 格式的图片,必须要有 PostScript 打印机来输出。

D. P. Carlisle 开发的 \texttt{graphicx} 宏包提供了很多包含图片的命令,这个宏包属于“graphics”宏集。

假设现在的系统有可以输出 PostScript 打印机,也安装好了 \texttt{graphicx} 宏包,可以根据下面的步骤在文档中包含图
片:
\begin{enumerate}
    \item 通过画图程序输出 EPS 格式的图片。
    \item 通过 \mintinline{LaTeX}|\usepackage[driver]{graphicx}| 命令,在导言区引入 \texttt{graphicx}
    宏包。 \\
    其中 \emph{driver} 是 dvi 转 PostScript 的转换程序,最常用的一种叫 \texttt{dvips}。知道
    \emph{driver} 的名字后,\texttt{graphicx} 宏包就可以选择正确的方法来将图形信息插入到 \texttt{.dvi} 文件
    中,然后打印机就能理解它并且正确地包含 \texttt{.eps} 文件。
    \item 在文档中使用 \mintinline{LaTeX}|\includegraphics[key=value, ...]{file}| 来包含 \emph{file}。
    \\ 命令中的可选参数允许有多个,之间用逗号隔开。\emph{key} 可以是 \texttt{width}、\texttt{height}、
    \texttt{angle}、\texttt{scale} 等参数,用于对包含的图形进行调整。
\end{enumerate}

示例代码:
\begin{minted}{LaTeX}
    \begin{figure}
        \centering
        \includegraphics[angle=90, width=0.5\textwidth]{test}
        \caption{This is a test.}
    \end{figure}
\end{minted}

上面的代码包含了存储好的 \texttt{test.eps} 图片。图片旋转了90度,并且图片的宽度缩放到了标准图片的0.5倍。由于没有
指定高度,所以默认宽高比是1。宽度和高度也可以指定为具体的长度。

\section{参考文献(Bibliography)}
通过 \texttt{thebibliography} 环境来生成参考文献。每一个条目都以 \mintinline{LaTeX}|\bibitem[lable]{marker}|
开头,再通过 \mintinline{LaTeX}|\cite{marker}| 命令,就可以用来在文档中引用书籍、文章、论文等。

如果不指定 \emph{label} 参数,所有的参考文献条目会自动编号。\mintinline{LaTeX}|\begin{thebibliography}| 命令
后的参数用来定义应该为条目编号预留多少空隙。在下面的示例中,该参数为 \texttt{\{99\}},表示所有的参考文献条目编号都不
能比数字99更宽。

示例代码:
\begin{minted}{LaTeX}
    Partl~\cite{pa} has proposed that \ldots
    \begin{thebibliography}{99}
    \bibitem{pa} H.~Partl: \emph{German \TeX}, TUGboat Volume~9, Issue~1 (1988)
    \end{thebibliography}
\end{minted}

示例输出:
Partl~\cite{pa} has proposed that \ldots
\begin{thebibliography}{99}
\bibitem{pa} H.~Partl: \emph{German \TeX}, TUGboat Volume~9, Issue~1 (1988)
\end{thebibliography}

对于更大型的项目,使用 Bib\TeX 是更好的选择。可以利用 Bib\TeX 建立一个参考文献数据库,然后再在文档中引用相关的文献。
Bib\TeX 产生的参考文献格式是通过样式文件定义的,网上有很多现成的样式文件可供选择。

\section{索引}
索引是一个非常有用的功能,在很多书中都能看到它。\LaTeX 中有个 \texttt{makeindex} 程序,可以方便地建立索引。这里仅
介绍基本的索引生成命令,更多内容见 \emph{The \LaTeX Companion}。

为了建立索引,需要在导言区通过 \mintinline{LaTeX}|\usepackage{makeindex}| 命令来引入 \texttt{makeidx} 宏
包,然后再在导言区中加入 \mintinline{LaTeX}{\makeindex} 命令来开启这项功能。

在正文中需要建立索引的地方,通过 \mintinline{LaTeX}|\index{key@formatted_entry}| 命令添加索引项。
可选参数 \emph{formatted\_entry} 会出现在建立索引的地方;参数 \emph{key} 用来排序。表~\ref{tab:index}
是索引项的写法示例。

\begin{table}[H]
\caption{索引项写法示例}
\label{tab:index}
\begin{center}
\begin{tabular}{@{}lll@{}}
  \textbf{Example} &\textbf{Index Entry} &\textbf{Comment}\\\hline
  \rule{0pt}{1.05em}\verb|\index{hello}| &hello, 1 &Plain entry\\
\verb|\index{hello!Peter}|   &\hspace*{2ex}Peter, 3 &Subentry under `hello'\\
\verb|\index{Sam@\textsl{Sam}}|     &\textsl{Sam}, 2& Formatted entry\\
\verb|\index{Lin@\textbf{Lin}}|     &\textbf{Lin}, 7& Formatted entry\\
\verb|\index{Kaese@K\"ase}|     &\textbf{K\"ase}, 33& Formatted entry\\
\verb.\index{ecole@\'ecole}.     &\'ecole, 4& Formatted entry\\
\verb.\index{Jenny|textbf}.     &Jenny, \textbf{3}& Formatted page number\\
\verb.\index{Joe|textit}.     &Joe, \textit{5}& Formatted page number
\end{tabular}
\end{center}
\end{table}

\LaTeX 在编译输入文件(\texttt{.tex} 文件)时,每一个 \mintinline{LaTeX}{\index} 命令都会把相应的索引项和当前页码写入和输入文件同名的
\texttt{.idx} 文件。\texttt{makeindex} 程序会对 \texttt{.idx} 文件进行处理,生成 \texttt{.ind} 文件。当再次
编译输入文件时,遇到 \mintinline{LaTeX}{\printindex} 命令时会在文档中输出索引。

\LaTeXe{} 中的 \texttt{showidx} 宏包将所有的索引项打印在相应的文本左侧,这在验证索引时十分有用。


\end{document}
